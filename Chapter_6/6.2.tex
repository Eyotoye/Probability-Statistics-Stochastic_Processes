\documentclass[../main.tex]{subfiles}
\begin{document}

设 $(X_1,\cdots,X_n)$ 的联合分布(PMF 或 PDF)为 $f(x_1,\cdots,x_n;\theta)$,其中 $\theta$ 为未知参数。

对于观测 $(X_1,\cdots,X_n)$,定义\emph{似然函数}(likelihood function)为 $L(\theta)=f(X_1,\cdots,X_n;\theta)$。

对于离散情形,$L(\theta)$ 就是当参数为 $\theta$ 时出现观测 $(X_1,\cdots,X_n)$ 的概率。

随机变量 $X_1,\cdots,X_n$ 的一个\emph{实现}是指一次观测到的具体数据,记为 $x_1,\cdots,x_n$。

若 $X_1,\cdots,X_n$ 独立同分布,来自总体 $f_1(x;\theta)$(PMF 或 PDF),则 $f(x_1,\cdots,x_n;\theta)=\prod_{i=1}^nf_1(x_i;\theta)$,似然函数 $L(\theta)=\prod_{i=1}^nf_1(X_i;\theta)$。

\begin{example}
    设 $X_1,\cdots,X_n$ 独立同分布,$X_i\sim N(\mu,\sigma^2)$,$\mu$ 和 $\sigma^2$ 未知,则 $f_1(x;\mu,\sigma^2)=\frac1{\sqrt{2\pi}\sigma}e^{-\frac{(x-\mu)^2}{2\sigma^2}}$,似然函数 $L(\theta)=L(\mu,\sigma^2)=\prod_{i=1}^n\frac1{\sqrt{2\pi}\sigma}e^{-\frac{(X_i-\mu)^2}{2\sigma^2}}$。
\end{example}

\begin{definition}\label{def:6.2.1}
    $\theta^*=\underset{\theta}{\operatorname{argmax}}L(\theta)$ 称为 $\theta$ 的\emph{极大似然估计}(MLE)。
\end{definition}

注意 $\theta^*=\theta^*(X_1,\cdots,X_n)$ 是一个随机变量,因为它是 $X_1,\cdots,X_n$ 的函数。

\begin{example}
    上例中,解方程 $\frac{\partial\log L}{\partial\mu}=0$ 和 $\frac{\partial\log L}{\partial(\sigma^2)}=0$(称它们为\emph{似然方程}),得 $\mu^*=\bar X$ 和 $(\sigma^2)^*=\frac1n\sum_{i=1}^n(X_i-\bar X)^2$。
\end{example}

此处 MLE 的结果与矩估计一致,这是偶然现象,对于一般分布不总成立。

\begin{proposition}
    MLE 有重要的所谓不变性:设 $\theta^*$ 是 $\theta$ 的 MLE,$g(\theta)$ 是 $\theta$ 的可测函数,则 $g(\theta^*)$ 是 $g(\theta)$ 的 MLE。例如,如果上例中选择 $\theta=(\mu,\sigma)$,则 $\sigma^*=\sqrt{(\sigma^2)^*}$ 是 $\sigma$ 的 MLE。
\end{proposition}

\begin{example}
    设 $X_1,\cdots,X_n$ 独立同分布,$X_i\sim U(0,\theta),\theta>0$ 未知,$L(\theta)=\left\{
        \begin{aligned}
            \frac1{\theta^n} & , & X_i\in (0,\theta),\forall i, \\
            0                & , & \text{其他},
        \end{aligned}\right.$ 则 $\theta^*=\max\{X_1,\cdots,X_n\}$。
\end{example}

\begin{example}
    设 $X_1,\cdots,X_n$ 独立同分布,$X_i$ 的 PDF 为 $f_1(x;\theta)=\frac1{\pi(1+(x-\theta)^2)}(x\in\mathbb R)$,$\theta$ 未知,即 $X_1,\cdots,X_n$ 服从 Cauchy 分布。
    \begin{itemize}
        \item 由于 Cauchy 分布的任意阶矩都不存在,故不能用矩估计。
        \item 若采用 MLE 方法,似然方程为 $\sum_{i=1}^{n}\frac{X_i-\theta}{1+(X_i-\theta)^2}=0$,当 $n$ 较大时,此方程有很多的根且无显式解,故 MLE 方法也不理想。
        \item 一种可能的对 $\theta$ 的估计:由于 $\theta$ 为中位数,因此用样本中位数作为 $\theta$ 的估计。
    \end{itemize}
\end{example}

这个例子告诉我们,统计方法不是唯一的,也没有绝对的优劣。

需要指出,MLE 不一定是唯一的。

MLE 的另一局限性是它需要分布的具体函数形式,而矩估计不需要。

此外,如果似然函数在最大值点附近变化过于平缓,则可能不利于通过迭代等方法有效计算。

\end{document}
