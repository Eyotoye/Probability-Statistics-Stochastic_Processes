\documentclass[../main.tex]{subfiles}
\begin{document}

本节开始讨论\emph{简单线性回归},其具体形式为 $Y=\beta_0+\beta_1X+\epsilon$,该式称为理论模型,其中 $\beta_0$ 为\emph{截距},$\beta_1$ 为\emph{斜率}或\emph{回归系数},二者统称\emph{回归参数},均未知且待定。

这里的“简单”指的是只有一个自变量 $X$,“线性”指的是 $f$ 关于 $\beta_0,\beta_1$ 线性。

设对 $(X,Y)$ 进行 $n$ 次独立观测,得到样本 $(x_1,y_1),\cdots,(x_n,y_n)$,即
\begin{align*}
    \left\{
    \begin{aligned}
         & y_i=\beta_0+\beta_1x_i+\epsilon_i(i=1,2,\cdots,n)                                       \\
         & \epsilon_i\text{ 独立同分布,}\mathrm E(\epsilon_i)=0,\mathrm{Var}(\epsilon_i)=\sigma^2 \\
    \end{aligned}
    \right.
\end{align*}
称之为简单线性回归模型。由此,$\mathrm E(y_i)=\beta_0+\beta_1x_i,\mathrm{Var}(y_i)=\sigma^2$。

\end{document}
