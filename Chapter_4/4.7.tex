\documentclass[../main.tex]{subfiles}
\begin{document}

我们定义\emph{条件期望} $\mathrm E(Y|X\in A)=\left\{
    \begin{aligned}
        \sum_iy_iP(Y=y_i|X\in A) \\
        \int_{-\infty}^{+\infty}yf_{Y|X}(y|X\in A)\mathrm dy
    \end{aligned}\right.$,两种定义分别针对 $Y$ 为离散型和连续型随机变量。

进而,我们定义 $\mathrm E(Y|x)=\mathrm E(Y|X=x)=\left\{
    \begin{aligned}
        \sum_iy_iP(Y=y_i|X=x) \\
        \int_{-\infty}^{+\infty}yf_{Y|X}(y|x)\mathrm dy
    \end{aligned}\right.$
,注意到这是一个 $x$ 的函数,记作 $h(x)$。将其作用在 $X$ 上,得到 $h(X)=\mathrm E(Y|X)$,这是一个 $X$ 的函数(称为 $Y$ 对 $X$ 的回归函数),因此是一个新的随机变量。

\begin{example}
    $(X,Y)\sim N(\mu_1,\mu_2,\sigma_1^2,\sigma_2^2,\rho)$,则 $\mathrm E(Y|x)=\mu_2+\rho\frac{\sigma_2}{\sigma_1}(x-\mu_1)$。
\end{example}

\begin{example}
    甲、乙两种同类产品,平均使用寿命分别为 $10$ 年和 $15$ 年,市场占有率分别为 $60\%$ 和 $40\%$,随机买一个,则期望寿命是 $10\times 60\%+15\times 40\%=12$ 年,我们发现这个计算过程可以表示为 $\mathrm E(Y)=\mathrm E(Y|X=1)P(X=1)+\mathrm E(Y|X=2)P(X=2)=h(1)P(X=1)+h(2)P(X=2)=\mathrm E(h(X))=\mathrm E(\mathrm E(Y|X))$,其中 $X=1$ 表示抽到甲产品,$X=0$ 表示抽到乙产品,$Y$ 表示抽到的产品的寿命。
\end{example}

一般地,我们有以下定理:

\begin{theorem}\label{thm:4.7.1}
    (全期望公式)\\
    对于随机向量 $(X,Y)$,有 $\mathrm E(Y)=\mathrm E(\mathrm E(Y|X))$。
\end{theorem}

\begin{proof}
    以连续型为例。设 $(X,Y)$ 的联合 PDF 为 $f(x,y)$,有 $\mathrm E(Y|x)=\int_{-\infty}^{+\infty}yf_{Y|X}(y|x)\mathrm dy=\int_{-\infty}^{+\infty}y\frac{f(x,y)}{f_X(x)}\mathrm dy$,故 $\mathrm E(Y)=\int_{-\infty}^{+\infty}yf_Y(y)\mathrm dy=\int_{-\infty}^{+\infty}\int_{-\infty}^{+\infty}yf(x,y)\mathrm dx\mathrm dy=\int_{-\infty}^{+\infty}\mathrm E(Y|x)f_X(x)\mathrm dx=\mathrm E(\mathrm E(Y|X))$。
\end{proof}

一般地,对于可测函数 $g$,我们有 $\mathrm E(g(X,Y))=\mathrm E(\mathrm E(g(X,Y)|X))$。

\begin{theorem}\label{thm:4.7.2}
    对于随机向量 $(X,Y)$ 和任意可测函数 $g:\mathbb R\rightarrow\mathbb R$,都有 $\mathrm E((Y-g(X))^2)\geq \mathrm E((Y-\mathrm E(Y|X))^2)$,即条件期望是均方误差意义下的最优预测。
\end{theorem}

\begin{proof}
    类比期望的性质 $\mathrm E((Y-c)^2)\geq\mathrm E((Y-\mathrm E(Y))^2),\forall c\in\mathbb R$,我们有 $\mathrm E((Y-g(X))^2|X)\geq\mathrm E((Y-\mathrm E(Y|X))^2|X),\forall g:\mathbb R\rightarrow\mathbb R$ 可测,两边对 $X$ 求期望即得。
\end{proof}

我们经常用到最优线性预测,即 $\min\limits_{a,b}\mathrm E((Y-(aX+b))^2)$,这种“均方意义上的最优”称之为\emph{最小二乘}(least square)。

% \begin{proposition}
% 记 $\hat Y=\mathrm E(Y|X)$ 为已知 $X$ 的条件下对 $Y$ 的最优估计,记估计误差 $\hat Y-Y$ 为 $\tilde Y$,则 $\mathrm E(\tilde Y)=0,\mathrm E(\tilde Y\hat Y)=0$,进而有 $\mathrm{Cov}(\hat Y,\tilde Y)=0,\mathrm{Var}(Y)=\mathrm{Var}(\hat Y)+\mathrm{Var}(\tilde Y)$。
% \end{proposition}

\end{document}
