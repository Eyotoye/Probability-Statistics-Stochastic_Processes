\documentclass[../main.tex]{subfiles}
\begin{document}

\begin{theorem}\label{thm:5.3.1}
    (中心极限定理,CLT)\\
    设随机变量 $X_1,\cdots,X_n$ 独立同分布,均值 $\mathrm E(X_i)=\mu$,方差 $\mathrm{Var}(X_i)=\sigma^2>0$,则 $\forall x\in\mathbb R,\lim_{n\rightarrow+\infty}P\left(\frac{\overline X-\mu}{\sigma/\sqrt n}\leq x\right)=\Phi(x)$,其中 $\Phi(x)$ 为标准正态分布的 CDF。
\end{theorem}

\begin{proof}
    只对 $X_i$ 的 MGF 存在的情形给出证明。\\
    令 $M(t)=\mathrm E(e^{tX_i})$,不失一般性,假设 $\mu=0,\sigma^2=1$,则 $M(0)=1,M'(0)=0,M''(0)=1$,于是 $\mathrm E(e^{t\frac{X_1+\cdots+X_n}{\sqrt n}})=\left(M\left(\frac t{\sqrt n}\right)\right)^n$,而 $\lim_{n\rightarrow +\infty}n\ln M(\frac t{\sqrt n})=\lim_{y\rightarrow 0^+}\frac{\ln M(yt)}{y^2}(y=\frac1{\sqrt{n}})=\frac{t^2}2$,即 $\frac{X+\cdots+X_n}{\sqrt n}$ 的 MGF 的极限为 $e^{\frac{t^2}2}$,这说明 $\frac{X+\cdots+X_n}{\sqrt n}$ 的极限分布为标准正态分布。
\end{proof}

上述定理通常称为 Lindeberg CLT,可推广至不同分布的情形。

\begin{definition}\label{def:5.3.1}
    (依分布收敛)\\
    我们称 $Y_n$ \emph{依分布收敛}于 $Y$,记作 $Y_n\overset{d}{\rightarrow}Y$,如果 $\lim_{n\rightarrow+\infty}F_n(y)=F(y),\forall y\text{ 为 $F$ 的连续点}$,其中 $F_n(y)$ 和 $F(y)$ 分别为 $Y_n$ 和 $Y$ 的 CDF。
\end{definition}

用上述定义,CLT 可以表述为 $\frac{\overline X-\mu}{\sigma/\sqrt n}\overset{d}{\rightarrow}N(0,1)$。

如果将上式中的 $\frac{\overline X-\mu}{\sigma/\sqrt n}$ 理解为标准化的过程,则不难得出 $\overline X$ 近似服从 $N(\mu,\frac{\sigma^2}n)$,$X_1+\cdots+X_n$ 近似服从 $N(n\mu,n\sigma^2)$。

\begin{example}
    (De Moivre-Laplace CLT)\\
    设 $X_i\sim B(p)$,则 $\sum_{i=1}^nX_i\sim B(n,p)$,当 $n$ 充分大时,可以近似地认为 $\sum_{i=1}^nX_i\sim N(np,np(1-p))$,于是我们可近似计算 $P(t_1\leq \sum_{i=1}^nX_i\leq t_2)\approx \Phi(y_2)-\Phi(y_1)$,其中 $y_1=\frac{t_1-np-\frac12}{\sqrt{np(1-p)}},y_2=\frac{t_2-np+\frac12}{\sqrt{np(1-p)}}$,其中 $\frac12$ 是连续性修正项。
\end{example}

\begin{example}
    (选举问题)\\
    设 $p$ 为选民支持度(未知),随机抽样调查 $n$ 人,其中支持比例 $P_n=\frac1n\sum_{i=1}^nX_i$,其中 $X_i\sim B(p)$ 且独立。\\
    设置精度 $\epsilon=0.03$,置信度 $1-\alpha=95\%$,则至少需要 $n$ 为多少,才能保证 $P(|P_n-p|<\epsilon)\geq1-\alpha$?\\
    根据 CLT,我们有 $P(P_n-p\geq\epsilon)\approx2\left(1-\Phi(\frac\epsilon{\sqrt{p(1-p)/n}})\right)\leq\alpha$,于是 $n\geq\frac{z_{1-\alpha/2}^2p(1-p)}{\epsilon^2}$,其中 $z_{1-\alpha/2}$ 为标准正态分布的上 $1-\alpha/2$ 分位数,代入最大值点 $p=\frac12$,我们得到 $n\geq\frac{z_{1-\alpha/2}^2}4\epsilon^2$,代入 $\epsilon=0.03,\alpha=0.05$,我们得到 $n\geq1068$。
\end{example}

\end{document}
