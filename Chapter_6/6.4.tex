\documentclass[../main.tex]{subfiles}
\begin{document}

\begin{definition}\label{def:6.4.1}
    $\forall\alpha\in(0,1)$,$\hat\theta_i=\hat\theta_i(X_1,\cdots,X_n)(i=1,2)$ 为统计量,若 $P(\hat\theta_1<\theta<\hat\theta_2)\geq1-\alpha$,则称 $(\hat\theta_1,\hat\theta_2)$ 为 $\theta$ 的一个 $(1-\alpha)$-置信的\emph{(双侧)区间估计}。
\end{definition}

$(1-\alpha)$ 称为\emph{置信水平},\emph{置信系数}或\emph{置信度}是指置信水平中的最大者,这三个术语都是针对方法而言的。$\alpha$ 通常取 $0.05,0.01,0.1$ 等。

通常用 $\mathrm E(\hat\theta_2-\hat\theta_1)$ 来刻画区间估计的精度。我们遵循可靠度优先原则,即先保证置信水平,然后再提升精度。

\begin{example}
    设 $X_1,\cdots,X_n$ 独立同分布,$X_i\sim N(\mu,\sigma^2)$,$\mu$ 未知,$\sigma^2$ 已知,则由 $\bar X\sim N(\mu,\frac{\sigma^2}n)$,有 $\bar X-\mu\sim N(0,\frac{\sigma^2}n)$。为给出 $\mu$ 的区间估计,我们的目标是寻找 $c_1,c_2$ 使得 $P(\bar X-c_1<\mu<\bar X+c_2)\geq1-\alpha$,这等价于 $P(-c_2<\bar X-\mu<c_1)\geq1-\alpha$。设 $\alpha_1=P(\bar X-\mu\leq-c_2),\alpha_2=P(\bar X-\mu\geq c_1)$,一个自然的选择是令 $\alpha_1=\alpha_2=\alpha/2$(事实上这也是能够使精度最高的选择)。记 $z_{\frac\alpha2}$ 为 $N(0,1)$ 的上 $\frac\alpha2$-分位数,即 $\Phi(z_{\frac\alpha2})=1-\frac\alpha2$,则 $P(\left|\frac{\bar X-\mu}{\sigma/\sqrt n}\right|\leq z_{\frac\alpha2})=1-\alpha$,从而 $P(\bar X-z_{\frac\alpha2}\frac\sigma{\sqrt n}<\mu<\bar X+z_{\frac\alpha2}\frac\sigma{\sqrt n})=1-\alpha$,故 $(\bar X-z_{\frac\alpha2}\frac\sigma{\sqrt n},\bar X+z_{\frac\alpha2}\frac\sigma{\sqrt n})$ 是 $\mu$ 的一个 $(1-\alpha)$-置信的区间估计。
\end{example}

若 $\alpha=0.05$,则 $z_{\frac\alpha2}\approx1.96\approx2$。

上述区间估计的一种理解是:若用 $\bar X$ 来估计 $\mu$,则绝对误差 $|\bar X-\mu|$ 在 $(1-\alpha)$-置信下不超过 $z_{\frac\alpha2}\frac\sigma{\sqrt n}$。

区间的半长度为 $z_{\frac\alpha2}\frac\sigma{\sqrt n}$,如果给定精度,例如取 $\epsilon>0$,要求 $z_{\frac\alpha2}\frac\sigma{\sqrt n}\leq\epsilon$,则 $n\geq(\frac{z_{\frac\alpha2}\sigma}\epsilon)^2$,即样本容量至少为 $(\frac{z_{\frac\alpha2}\sigma}\epsilon)^2$ 时有 $(1-\alpha)-$置信使绝对误差不超过 $\epsilon$。这一推理可以理解为 $(\alpha,\epsilon,n)$ 三个变量之间存在的关系。

\begin{example}
    设 $X_1,\cdots,X_n$ 独立同分布,$X_i\sim N(\mu,\sigma^2)$,$\sigma^2$ 未知,首先估计 $\sigma^2$。注意到,$\frac{(n-1)S^2}{\sigma^2}=\sum_{i=1}^n\left(\frac{X_i-\bar X}{\sigma}\right)^2=\sum_{i=1}^n\left(\frac{X_i-\mu}{\sigma}\right)^2-\left(\frac{\bar X-\mu}{\frac\sigma{\sqrt{n}}}\right)^2\sim\chi^2(n-1)$,同样令 $\alpha_1=\alpha_2=\alpha/2$,有 $(\frac{(n-1)S^2}{\chi^2_{\frac\alpha2}(n-1)},\frac{(n-1)S^2}{\chi^2_{1-\frac\alpha2}(n-1)})$ 是 $\sigma^2$ 的一个 $(1-\alpha)$-置信的区间估计,其中 $\chi^2_{\frac\alpha2}(n-1)$ 和 $\chi^2_{1-\frac\alpha2}(n-1)$ 分别为 $\chi^2(n-1)$ 的上 $\frac\alpha2$-分位数和下 $\frac\alpha2$-分位数。\\
    接下来估计未知参数 $\mu$,可以证明,$\frac{\bar X-\mu}{\frac\sigma{\sqrt{n}}}\sim N(0,1)$ 且与 $\frac{(n-1)S^2}{\sigma^2}$ 独立,从而 $\frac{\frac{\bar X-\mu}{\frac\sigma{\sqrt{n}}}}{\sqrt{\frac{\frac{(n-1)S^2}{\sigma^2}}{n-1}}}=\frac{\bar X-\mu}{\frac S{\sqrt{n}}}\sim t(n-1)$,故 $(\bar X-t_{\frac\alpha2}(n-1)\frac S{\sqrt{n}},\bar X+t_{\frac\alpha2}(n-1)\frac S{\sqrt{n}})$ 是 $\mu$ 的一个 $(1-\alpha)$-置信的区间估计,其中 $t_{\frac\alpha2}(n-1)$ 为 $t(n-1)$ 的上 $\frac\alpha2$-分位数。
\end{example}

\begin{example}
    若 $X\sim N(\mu_1,\sigma^2),Y\sim N(\mu_2,\sigma^2)$,且 $X,Y$ 独立,下面估计均值差 $\mu_1-\mu_2$。设随机样本为 $X_1,\cdots,X_n$ 和 $Y_1,\cdots,Y_m$,则 $\bar X-\bar Y\sim N(\mu_1-\mu_2,\frac{\sigma^2}n+\frac{\sigma^2}m)$,有 $\frac{(\bar X-\bar Y)-(\mu_1-\mu_2)}{\sigma\sqrt{\frac1n+\frac1m}}\sim N(0,1)$。同时,由 $\frac{\sum_{i=1}^n(X_i-\bar X)^2}{\sigma^2}=\frac{(n-1)S_1^2}{\sigma^2}\sim\chi^2(n-1)$ 和 $\frac{\sum_{i=1}^m(Y_i-\bar Y)^2}{\sigma^2}= \frac{(m-1)S_2^2}{\sigma^2}\sim\chi^2(m-1)$,且 $\frac{(n-1)S_1^2}{\sigma^2}$ 与 $\frac{(m-1)S_2^2}{\sigma^2}$ 独立,有 $\frac{(n-1)S_1^2}{\sigma^2}+\frac{(m-1)S_2^2}{\sigma^2}\sim\chi^2(n+m-2)$,故 $\frac{\frac{(\bar X-\bar Y)-(\mu_1-\mu_2)}{\sigma\sqrt{\frac1n+\frac1m}}}{\sqrt{\frac{\frac{(n-1)S_1^2}{\sigma^2}+\frac{(m-1)S_2^2}{\sigma^2}}{n+m-2}}}=\frac{(\bar X-\bar Y)-(\mu_1-\mu_2)}{S\sqrt{\frac1n+\frac1m}}\sim t(n+m-2)$,其中 $S^2=\frac{(n-1)S_1^2+(m-1)S_2^2}{n-m+2}$,于是 $(\bar X-\bar Y-t_{\frac\alpha2}(n+m-2)S\sqrt{\frac1n+\frac1m},\bar X-\bar Y+t_{\frac\alpha2}(n+m-2)S\sqrt{\frac1n+\frac1m})$ 是 $\mu_1-\mu_2$ 的一个 $(1-\alpha)$-置信的区间估计。
\end{example}

\end{document}
