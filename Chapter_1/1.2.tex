\documentclass[../main.tex]{subfiles}
\begin{document}

\begin{definition}\label{def:1.2.1}
    概率论中的\emph{随机试验}指的是符合下面两个特点的试验:
    \begin{enumerate}
        \item 不能预先确知结果
        \item 可以预测所有可能的结果
    \end{enumerate}
\end{definition}

\begin{definition}\label{def:1.2.2}
    \emph{样本空间}是指一个试验的所有可能结果的集合,常用 $\Omega$ 表示。
\end{definition}

\begin{definition}\label{def:1.2.3}
    \emph{事件}是样本空间的一个良定义的子集。
\end{definition}

一次随机试验中,一个事件可能发生或不发生。

下面是一些常见的事件:
\begin{enumerate}
    \item 全事件 $\Omega$(必然事件)
    \item 空事件 $\varnothing$(不可能事件)
    \item 基本事件 $\{a\}$,其中 $a\in\Omega$,即仅包含单一试验结果的事件
\end{enumerate}

\end{document}
