\documentclass[../main.tex]{subfiles}
\begin{document}

\begin{definition}\label{def:6.2.1}
    设 $T=\mathbb N^*$,$\{X_n\}_{n=1}^\infty$ 是一列独立同分布的随机变量,且 $P(X_n=1)=p,P(X_n=0)=1-p$,则称 $\{X_n\}_{n=1}^\infty$ 为参数为 $p$ 的 \emph{Bernoulli 过程}。
\end{definition}

定义中提到的 $\{X_n\}_{n=1}^\infty$ 相互独立,指的是 $\forall n\in\mathbb N^*$,都有 $X_1,\cdots,X_n$ 相互独立。

如果在每个离散时刻 $n$,将事件 $X_n=1$ 即“第 $n$ 次试验成功”理解为该时刻有一个顾客到达某商店,则 Bernoulli 过程属于一种\emph{到达过程}。

本章的主要内容 Poisson 过程也是一种到达过程。

\end{document}
