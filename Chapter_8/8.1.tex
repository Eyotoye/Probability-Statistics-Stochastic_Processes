\documentclass[../main.tex]{subfiles}
\begin{document}

本章的主题 Brown 运动是一种随机连续运动,其指标集 $T=[0,+\infty)$,状态空间 $S=\mathbb R$。

为理解 Brown 运动,先来考虑熟悉的随机游走。假设 $X(0)=0$,各 $X_i$ 独立同分布且 $P(X_i=1)=P(X_i=-1)=\frac12$,以 $\Delta x$ 为步长,$\Delta t$ 为时间跨度,令 $t$ 时刻的粒子位置为 $X(t)=\Delta x\sum_{i=1}^{\lfloor t/\Delta t\rfloor}X_i$,则 $\mathrm E(X(t))=0,\mathrm{Var}(X(t))=(\Delta x)^2\lfloor t/\Delta t\rfloor$。令 $(\Delta x)^2=\sigma^2\Delta t$,则 $\mathrm{Var}(X(t))\rightarrow \sigma^2t(\Delta t\rightarrow0)$,且 $\{X(t)\}$ 具有平稳增量性和独立增量性。

\begin{definition}\label{def:8.1.1}
    称一个随机过程 $\{X(t),t\geq 0\}$ 为 \emph{Brown 运动},若其满足:
    \begin{enumerate}
        \item $X(0)=0$(这一要求并不本质)
        \item $\{X(t),t\geq 0\}$ 有平稳增量性和独立增量性
        \item $\forall t>0$,$X(t)\sim N(0,\sigma^2t)$
    \end{enumerate}
    若 $\sigma^2=1$,称之为\emph{标准 Brown 运动},记为 $\{B(t),t\geq0\}$。
\end{definition}

历史上,Brown 运动是由 Brown 于 1827 年发现的,Einstein 于 1905 年给出了其数学解释,而 Wiener 于 1918 年给出了精确的数学刻画。

根据平稳增量性,$\forall0<s<t,B(t+s)-B(s)\sim N(0,\sigma^2t)$。而根据独立增量性,$\forall s,t>0,a,x\in\mathbb R,P(B(t+s)\leq x|B(s)=a,B(u)(0\leq u<s))=P(B(t+s)-B(s)\leq x-a|B(s)=a,B(u)(0\leq u<s))=P(B(t+s)-B(s)\leq x-a)=P(B(t+s)\leq x|B(s)=a)$,这说明 Brown 运动有 Markov 性(是 Markov 过程)。

可以证明,$B(t)$ 以概率 $1$ 关于 $t$ 连续且处处不可微。直观地,$\mathrm E((B(t+\Delta t)-B(t))^2)=\Delta t$,故 $B(t+\Delta t)-B(t)\sim\sqrt{\Delta t}$,因此 $\lim_{\Delta t\rightarrow0}\frac{B(t+\Delta t)-B(t)}{\Delta t}$ 不存在。

\end{document}
