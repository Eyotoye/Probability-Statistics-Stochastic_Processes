\documentclass[../main.tex]{subfiles}
\begin{document}

对于随机变量 $X$ 和可测函数 $g$,$Y=g(X)$ 也是随机变量。特别地,若 $X$ 为离散型随机变量,则 $Y$ 也离散。但若 $X$ 为连续型随机变量,$Y$ 未必连续。

\begin{example}
    $X\sim Exp(\lambda)$,$Y=\left\{
        \begin{aligned}
            0 & , & X\leq t_0, \\
            1 & , & X>t_0,
        \end{aligned}
        \right.$ 其中 $t_0>0$ 为常数,则 $Y\sim B(e^{-\lambda t_0})$。
\end{example}

\begin{example}
    设 $X$ 为连续型随机变量,PDF 为 $f(x)$,考虑 $Y=X^2$。\\
    从 CDF 入手,$\forall y>0,P(Y\leq y)=P(X^2\leq y)=P(-\sqrt{y}\leq X\leq \sqrt{y})=\int_{-\sqrt{y}}^{\sqrt{y}}f(x)\mathrm{d}x$,有 $Y$ 的 PDF 为 $l(y)=\frac{\mathrm{d}}{\mathrm{d}y}P(Y\leq y)=\frac{1}{2\sqrt{y}}(f(\sqrt{y})+f(-\sqrt{y}))(y>0)$。\\
    特别地,若 $X\sim N(0,1)$,称 $Y$ 服从自由度为 $1$ 的 $\chi^2$-分布,读作“卡方分布”。
\end{example}

若 $Y=g(X)$ 为随机变量,可以计算 $Y$ 的分布如下:
\begin{itemize}
    \item $P(Y=y)=P(g(X)=y)=P(X\in g^{-1}(y))$
    \item $P(Y\leq y)=P(g(X)\leq y)=P(X\in g^{-1}((-\infty,y]))$
\end{itemize}

\end{document}
