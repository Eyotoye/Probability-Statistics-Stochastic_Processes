\documentclass[../main.tex]{subfiles}
\begin{document}

统计学是一门从数据中获得信息的学问。根据 Claude Shannon 的信息论,所谓的信息就是不确定性的分解。

数理统计通常包括数据收集、数据分析和统计推断三部分。

% 其中数据分析这一步要依附统计模型,而统计推断这一步涉及从样本推断总体的问题。

\begin{example*}
    检测某厂的一大批电子元件产品的寿命,我们关注的问题是“判断产品是否合格”。这个问题的“总体”就是所需检测的这批元件的寿命,更具体地说,是元件寿命这一随机变量 $X$ 的分布。
\end{example*}

统计学上所谓\emph{总体},就是指一个概率分布。而统计分析问题就是研究对象全体所服从的分布的某个数字特征,来了解总体变量 $X$ 的分布。

总体可以分为有限总体、无限总体等,其中有限总体在个体数量很多时可以近似看作无限总体。

所谓的“虚拟总体”是一种无限总体,并无实际存在的个体集合,而是一个假想的、潜在的无限个体集合,如测量讲桌的长度所得到的测量值,可以视为来自一个虚拟总体。

将一族概率分布称为一个\emph{统计模型}。

\begin{example*}
    正态分布族 $\{N(\mu,\sigma^2):\mu\in\mathbb R,\sigma^2>0\}$ 就是一个统计模型。
\end{example*}

模型可以分为参数模型和非参数模型,正态分布族就是一个参数模型。非参数模型是指不能用少数几个参数决定的模型,例如对某总体 $X$,限定 $X$ 连续,$\mathrm E(X)$ 存在或属于某个取值范围等条件,但不用具体的若干参数去精确描述 $X$ 的分布,这就是一个非参数模型。

\emph{样本}是指从总体中抽取的一组观测值 $X_1,\cdots,X_n$,其中每个 $X_i$ 来自总体 $X$,而 $n$ 称为\emph{样本容量}。

抽样方式分为\emph{试验}与\emph{观测},后者又可以分为完全观测和不完全观测。

% 抽样调查是观测还是试验?

若 $X_1,\cdots,X_n$ 独立同分布,且 $X_i\sim X$,则称 $X_1,\cdots,X_n$ 为来自总体 $X$ 的一个\emph{随机样本}。对于有限总体,这需要有放回地抽样。

\emph{简单随机抽样}是指当总体个数 $N$ 有限,从中无放回地抽取 $n$ 个个体,每个个体被抽取的概率相同。这种情况下,任意容量为 $n$ 的样本都有相同的出现概率,为 $\frac1{\tbinom Nn}$。

抽样方式的选择有很多需要注意的地方,否则可能属于不当抽样。

\begin{definition*}\label{def:6.0.1}
    \emph{统计量}定义为样本的函数,即 $T(X_1,\cdots,X_n)$。
\end{definition*}

统计量是完全由样本决定的量,因此也是随机变量。统计量可以看作一种对数据进行简化的方式。

\begin{example*}
    设 $X_1,\cdots,X_n$ 独立同分布,均值 $\mathrm E(X_i)=\mu$,则以下是一些常用的统计量:
    \begin{enumerate}
        \item 样本均值 $\bar X=\frac1n\sum_{i=1}^nX_i$;
        \item 样本方差 $S^2=\frac1{n-1}\sum_{i=1}^n(X_i-\bar X)^2$;
        \item 当 $\mu$ 已知时,$\bar X-\mu$ 是统计量;当 $\mu$ 未知时,$\bar X-\mu$ 不是统计量。
    \end{enumerate}
\end{example*}

总体决定样本,故可以通过样本来推断总体的性质,这就是\emph{统计推断}。统计推断又可以分为经典方法(频率学派的)以及 Bayes 方法。

\begin{example*}
    设总体满足 $Y=aX+\epsilon$,其中 $X$ 为自变量,$Y$ 为因变量,$\epsilon$ 为误差。这是一个参数模型。\\
    假设抽取的样本为 $(X_1,Y_1),\cdots,(X_n,Y_n)$,则:
    \begin{itemize}
        \item 若 $a$ 未知,可通过观测各 $(X_i,Y_i)$ 来估计 $a$,这属于模型推断、参数估计的范畴;
        \item 若 $a$ 已知,可通过观测 $Y_i$ 来估计 $X_i$,这属于变量推断、模型应用的范畴。
    \end{itemize}
\end{example*}

\begin{example*}
    假设元件寿命 $X\sim Exp(\lambda)$,如何通过样本估计 $\lambda$ 的值?这是一个参数估计问题。\\
    假设元件的合格标准是 $\mathrm E(X)\geq L$,但 $\mathrm E(X)$ 未知。考虑制定一种可操作的检验标准,当 $\bar X\geq l$ 时,就认为元件合格。这种标准如何制定?这是一个假设检验问题。
\end{example*}

\end{document}
