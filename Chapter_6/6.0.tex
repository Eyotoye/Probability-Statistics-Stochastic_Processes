\documentclass[../main.tex]{subfiles}
\begin{document}

统计学是一门从数据中获得信息的学问。

数理统计通常包括数据收集、数据分析和统计推断等部分。其中数据分析这一步要依附统计模型,而统计推断这一步涉及从样本推断总体的问题。

\begin{example}
    检测某厂大批电子元件产品寿命,我们关注的问题是“如何判断产品是否合格”。这个问题的“总体”就是所要检测的这批元件的寿命,更具体地说,是元件寿命这一随机变量 $X$ 的分布。
\end{example}

统计学上所谓\emph{总体},就是指一个概率分布,而统计问题往往就是研究对象全体所服从的分布的某个数字特征。

总体可以分为无限总体、有限总体等,其中有限总体在个体数量很多时可以近似看作无限总体。

% 虚拟总体?

我们将一族概率分布称为一个\emph{统计模型}。

\begin{example}
    正态分布族 $\{N(\mu,\sigma^2):\mu\in\mathbb R,\sigma^2>0\}$ 就是一个统计模型。
\end{example}

模型可以分为参数模型和非参模型,正态分布族是一个参数模型。非参模型是指不能用有限个参数来刻画的模型,例如对某总体 $X$,我们限定 $X$ 连续,$\mathrm E(X)$ 存在等条件,但不假设 $X$ 服从某个特定的分布,这就是一个非参模型。

\emph{样本}是指从总体中抽取的一组观测值 $(X_1,\cdots,X_n)$,其中每个 $X_i$ 来自总体 $X$,而 $n$ 称为\emph{样本容量}。、

样本的来源分为\emph{观测}与\emph{试验},前者又可以分为完全观测和不完全观测。

% 抽样调查是观测还是试验?

\emph{简单随机抽样}是指当总体个数 $N$ 有限,从中无放回地抽取 $n$ 个个体,每个个体被抽取的概率相同。这种情况下,任意容量为 $n$ 的样本都有相同的出现概率。

若 $X_1,\cdots,X_n$ 独立同分布,且 $X_i\sim X$,则称 $(X_1,\cdots,X_n)$ 为来自总体 $X$ 的一个\emph{随机样本}。对于有限总体,这往往需要有放回,或近似有放回地抽样(例如总体中个体数量很大时)。

\begin{definition}\label{def:6.0.1}
    \emph{统计量}定义为样本的函数,即 $T=T(X_1,\cdots,X_n)$。
\end{definition}

统计量是完全由样本决定的量,因此也是随机变量。统计量可以看作一种对数据进行简化的方式。

\begin{example}
    设 $X_1,\cdots,X_n$ 独立同分布,均值 $\mathrm E(X_i)=\mu$,则以下是一些常用的统计量:
    \begin{enumerate}
        \item 样本均值 $\overline X=\frac1n\sum_{i=1}^nX_i$;
        \item 样本方差 $S^2=\frac1{n-1}\sum_{i=1}^n(X_i-\overline X)^2$;
        \item 当 $\mu$ 已知时,$\overline X-\mu$ 是统计量;当 $\mu$ 未知时,$\overline X-\mu$ 不是统计量。
    \end{enumerate}
\end{example}

由于样本(果)是来自总体(因)的,我们可以通过样本来推断总体的性质,这就是\emph{统计推断}。统计推断又可以分为经典方法(频率学派的)以及 Bayes 方法。

\begin{example}
    假设元件寿命 $X\sim Exp(\lambda)$,如何通过样本估计 $\lambda$ 的值?(参数估计)\\
    假设元件的合格标准是 $\mathrm E(X)\geq L$,但 $\mathrm E(X)$ 未知,如果制定一种检验标准,当 $\overline X\geq L$ 时,我们认为元件合格,这种标准是否合理?(假设检验)
\end{example}

\begin{example}
    设我们抽取的样本有如下关系:\\
    $Y_i=aX_i+\epsilon_i$,其中 $X_i$ 为自变量,$Y_i$ 为因变量,$\epsilon_i$ 为误差。\\
    若 $a$ 未知,通过观测 $X_i,Y_i$ 来估计 $a$,属于模型推断、参数估计的范畴;\\
    若 $a$ 已知,通过观测 $Y_i$ 来估计 $X_i$,属于变量推断的范畴。
\end{example}

\end{document}
