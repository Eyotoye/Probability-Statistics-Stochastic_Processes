\documentclass[../main.tex]{subfiles}
\begin{document}

\begin{definition}\label{def:7.3.1}
    $\forall i,j\in S$,称状态 $i$ \emph{可达}状态 $j$,若 $\exists n\geq0,p_{ij}^{(n)}>0$,记作 $i\rightarrow j$。若 $i\rightarrow j$ 且 $j\rightarrow i$,则称 $i$ 和 $j$ 是\emph{互通}的,记作 $i\leftrightarrow j$。
\end{definition}

\begin{proposition}
    $\leftrightarrow$ 是 $S$ 上的等价关系,即其具有
    \begin{enumerate}
        \item 自反性:$i\leftrightarrow i,\forall i\in S$
        \item 对称性:$i\leftrightarrow j\Rightarrow j\leftrightarrow i,\forall i,j\in S$
        \item 传递性:$i\leftrightarrow j,j\leftrightarrow k\Rightarrow i\leftrightarrow k,\forall i,j,k\in S$
    \end{enumerate}
\end{proposition}

\begin{proof}
    仅证明传递性。若 $i\rightarrow j,j\rightarrow k$,则 $\exists m,n\geq0,p_{ij}^{(m)}>0,p_{jk}^{(n)}>0$,由 C-K 方程可得 $p_{ik}^{(m+n)}=\sum_{l\in S}p_{il}^{(m)}p_{lk}^{(n)}\geq p_{ij}^{(m)}p_{jk}^{(n)}>0$,故 $i\rightarrow k$。同理可证 $k\rightarrow i$,故 $i\leftrightarrow k$。
\end{proof}

若 $i\leftrightarrow j$,则称 $i,j$ 归属一类。

\begin{definition}\label{def:7.3.2}
    若某 Markov 链的所有状态只有一类,则称该链\emph{不可约},否则称为\emph{可约}。
\end{definition}

\begin{example}
    下图中的三个 Markov 链分别是不可约、可约、可约的。
    \begin{center}
        \begin{tikzpicture}[>=stealth,thick, node distance=.5cm,baseline=(current bounding box.north)]
            \tikzstyle{state} = [circle, draw, minimum size=1cm]
            \node[state] (1) at (0,0) {1};
            \node[state] (2) [below left=of 1] {2};
            \node[state] (3) [below right=of 1] {3};
            \node[state] (5) [below=of 2] {5};
            \node[state] (4) [below=of 3] {4};
            \draw[->] (1) -- (2);
            \draw[->] (2) -- (3);
            \draw[->] (3) -- (1);
            \draw[->] ([yshift=2pt]5.east) -- ([yshift=2pt]4.west);
            \draw[->] ([yshift=-2pt]4.west) -- ([yshift=-2pt]5.east);
            \draw[->] ([xshift=-2pt]3.south) -- ([xshift=-2pt]4.north);
            \draw[->] ([xshift=2pt]4.north) -- ([xshift=2pt]3.south);
        \end{tikzpicture}
        \qquad\qquad
        \begin{tikzpicture}[>=stealth,thick, node distance=.5cm,baseline=(current bounding box.north)]
            \tikzstyle{state} = [circle, draw, minimum size=1cm]
            \node[state] (1) at (0,0) {1};
            \node[state] (2) [below left=of 1] {2};
            \node[state] (3) [below right=of 1] {3};
            \node[state] (5) [below=of 2] {5};
            \node[state] (4) [below=of 3] {4};
            \draw[->] (1) -- (2);
            \draw[->] (2) -- (3);
            \draw[->] (3) -- (1);
            \draw[->] ([yshift=2pt]5.east) -- ([yshift=2pt]4.west);
            \draw[->] ([yshift=-2pt]4.west) -- ([yshift=-2pt]5.east);
        \end{tikzpicture}
        \qquad\qquad
        \begin{tikzpicture}[>=stealth,thick, node distance=.5cm,baseline=(current bounding box.north)]
            \tikzstyle{state} = [circle, draw, minimum size=1cm]
            \node[state] (1) at (0,0) {1};
            \node[state] (2) [below left=of 1] {2};
            \node[state] (3) [below right=of 1] {3};
            \node[state] (5) [below=of 2] {5};
            \node[state] (4) [below=of 3] {4};
            \draw[->] (1) -- (2);
            \draw[->] (2) -- (3);
            \draw[->] (3) -- (1);
            \draw[->] (3) -- (4);
            \draw[->] ([yshift=2pt]5.east) -- ([yshift=2pt]4.west);
            \draw[->] ([yshift=-2pt]4.west) -- ([yshift=-2pt]5.east);
        \end{tikzpicture}
    \end{center}
    \bigskip
\end{example}

\begin{definition}\label{def:7.3.3}
    记 $f_{ij}^{(n)}$ 为从 $i$ 出发,经 $n$ 步首达 $j$ 的概率,即 $f_{ij}^{(n)}=P(X_n=j,X_k\neq j,k=1,\cdots,n-1|X_0=i)$。记 $f_{ij}=\sum_{n=1}^\infty f_{ij}^{(n)}$ 为从 $i$ 出发经有限步到达 $j$ 的概率。约定 $f_{ij}^{(0)}=\delta_{ij}$。
\end{definition}

\begin{definition}\label{def:7.3.4}
    若 $f_{ii}=1$,则称状态 $i$ 为\emph{常返}(recurrent)的。若 $f_{ii}<1$,则称状态 $i$ 为\emph{非常返}的或\emph{瞬态}(transient)的。
\end{definition}

\begin{theorem}\label{thm:7.3.1}
    $i$ 常返等价于 $\sum_{n=0}^\infty p_{ii}^{(n)}=+\infty$。$i$ 非常返等价于 $\sum_{n=0}^\infty p_{ii}^{(n)}=\frac1{1-f_{ii}}$。
\end{theorem}

先证明一个引理。

\begin{lemma}
    $\forall i,j\in S,n\in\mathbb N^*$,有 $p_{ij}^{(n)}=\sum_{k=1}^nf_{ij}^{(k)}p_{jj}^{(n-k)}$。
\end{lemma}

\begin{proof}
    用数学归纳法。对 $n=1$,由 $p_{ij}^{(1)}=f_{ij}^{(1)},p_{jj}^{(0)}=1$ 显然成立。假设对 $n-1$ 成立,即 $p_{ij}^{(n-1)}=\sum_{k=1}^{n-1}f_{ij}^{(k)}p_{jj}^{(n-1-k)}$,则
    \begin{equation*}
        \begin{aligned}
              & p_{ij}^{(n)}                                                                                        \\
            = & \sum_{l\in S}p_{il}^{(1)}p_{lj}^{(n-1)}                                                             \\
            = & p_{ij}^{(1)}p_{jj}^{(n-1)}+\sum_{l\neq j}p_{il}^{(1)}p_{lj}^{(n-1)}                                 \\
            = & p_{ij}^{(1)}p_{jj}^{(n-1)}+\sum_{l\neq j}p_{il}^{(1)}(\sum_{k=1}^{n-1}f_{lj}^{(k)}p_{jj}^{(n-1-k)}) \\
            = & p_{ij}^{(1)}p_{jj}^{(n-1)}+\sum_{k=1}^{n-1}(\sum_{l\neq j}p_{il}^{(1)}f_{lj}^{(k)})p_{jj}^{(n-1-k)} \\
            = & p_{ij}^{(1)}p_{jj}^{(n-1)}+\sum_{k=1}^{n-1}f_{ij}^{(k+1)}p_{jj}^{(n-1-k)}                           \\
            = & p_{ij}^{(1)}p_{jj}^{(n-1)}+\sum_{k=2}^nf_{ij}^{(k)}p_{jj}^{(n-k)}                                   \\
            = & \sum_{k=1}^nf_{ij}^{(k)}p_{jj}^{(n-k)}.
        \end{aligned}
    \end{equation*}
\end{proof}

下面证明定理~\ref{thm:7.3.1}。

\begin{proof}
    $\sum_{n=0}^\infty p_{ii}^{(n)}=1+\sum_{n=1}^\infty p_{ii}^{(n)}=1+\sum_{n=1}^\infty\sum_{k=1}^nf_{ii}^{(k)}p_{ii}^{(n-k)}=1+\sum_{k=1}^\infty f_{ii}^{(k)}\sum_{n=k}^\infty p_{ii}^{(n-k)}=1+\sum_{k=1}^\infty f_{ii}^{(k)}\sum_{n=0}^\infty p_{ii}^{(n)}=1+f_{ii}\sum_{n=0}^\infty p_{ii}^{(n)}$,故若 $i$ 非常返,则 $\sum_{n=0}^\infty p_{ii}^{(n)}=\frac1{1-f_{ii}}<+\infty$。而若 $i$ 常返,则 $\sum_{n=0}^\infty p_{ii}^{(n)}=+\infty$。
\end{proof}

若令 $I_n=\left\{\begin{aligned}
        1 & , & X_n=i,     \\
        0 & , & X_n\neq i,
    \end{aligned}\right.$ 则 $\sum_{n=0}^\infty I_n$ 为经过状态 $i$ 的次数,于是 $\mathrm E(\sum_{n=0}^\infty I_n|X_0=i)=\sum_{n=0}^\infty\mathrm E(I_n|X_0=i)=\sum_{n=0}^\infty p_{ii}^{(n)}$ 为链从 $i$ 出发经过 $i$ 的期望次数。若 $i$ 常返,从 $i$ 出发以概率 $1$ 经过 $i$ 无穷多次。若 $i$ 非常返,从 $i$ 出发有概率 $(1-f_{ii})$ 回不到 $i$(逃离 $i$),经过 $i$ 的次数为 $k$ 的概率为 $f_{ii}^{k-1}(1-f_{ii})$,即经过 $i$ 的次数服从几何分布,链以概率 $1$ 最终逃离 $i$。

\begin{theorem}\label{thm:7.3.2}
    关于常返性,有如下性质:
    \begin{enumerate}
        \item 若 $i\leftrightarrow j$,则 $i,j$ 同为常返态或同为非常返态
        \item 有限链至少存在一个常返态
        \item 若链有限且不可约,则所有状态均为常返态
    \end{enumerate}
\end{theorem}

\begin{proof}
    仅证明性质 1。由 $i\leftrightarrow j$,知 $\exists m,n,p_{ij}^{(m)}>0,p_{ji}^{(n)}>0$,且由 C-K 方程知 $p_{ii}^{(m+n+k)}\geq p_{ij}^{(m)}p_{jj}^{(k)}p_{ji}^{(n)}$,故 $\sum_{k=0}^\infty p_{ii}^{(k)}\geq\sum_{k=0}^\infty p_{ii}^{(m+n+k)}\geq p_{ij}^{(m)}p_{ji}^{(n)}\sum_{k=0}^\infty p_{jj}^{(k)}$。交换 $i$ 与 $j$ 可得类似不等式,故 $\sum_{k=0}^\infty p_{ii}^{(k)}$ 与 $\sum_{k=0}^\infty p_{jj}^{(k)}$ 相互控制,二者同为发散或同为收敛,故 $i,j$ 同为常返态或同为非常返态。
\end{proof}

\begin{example}
    一维随机游走,状态空间 $S=\mathbb Z$,转移概率为 $p_{i,i+1}=p\in(0,1),p_{i,i-1}=q=1-p$,则显然其不可约。可以证明,若 $p=\frac12$,则该链常返,否则非常返。
\end{example}

\begin{proposition}
    若 $i\rightarrow j$ 且 $i$ 常返,则 $f_{ji}=1$。
\end{proposition}

\begin{proof}
    若 $f_{ji}<1$,则从 $j$ 出发以概率 $1-f_{ji}>0$ 不能到 $i$。又 $i\rightarrow j$,则从 $i$ 出发有正概率回不到 $i$,与 $i$ 常返矛盾。
\end{proof}

上述命题说明,从常返态出发不能到达非常返态。对于有限链,若从非常返态出发,最终一定会到达某个常返态。

\begin{definition}\label{def:7.3.5}
    若集合 $\{n|p_{ii}^{(n)}>0\}$ 非空,则其所有元素的最大公约数 $d(i)$ 称为 $i$ 的\emph{周期}。若 $d(i)>1$,则称 $i$ 为\emph{周期}的,否则称为\emph{非周期}的。若链的所有状态都是非周期的,则称链是\emph{非周期}的,否则称链是\emph{周期}的。
\end{definition}

\begin{example}
    下图中的 Markov 链满足 $d(1)=d(2)=d(3)=3,d(4)=d(5)=1$,因此是周期的。
    \begin{center}
        \begin{tikzpicture}[>=stealth,thick, node distance=.5cm,baseline=(current bounding box.north)]
            \tikzstyle{state} = [circle, draw, minimum size=1cm]
            \node[state] (1) at (0,0) {1};
            \node[state] (2) [below left=of 1] {2};
            \node[state] (3) [below right=of 1] {3};
            \node[state] (5) [below=of 2] {5};
            \node[state] (4) [below=of 3] {4};
            \draw[->] (1) -- (2);
            \draw[->] (2) -- (3);
            \draw[->] (3) -- (1);
            \draw[->] (3) -- (4);
            \draw[->] ([yshift=2pt]5.east) -- ([yshift=2pt]4.west);
            \draw[->] ([yshift=-2pt]4.west) -- ([yshift=-2pt]5.east);
            \draw[->] (4) edge [loop right] ();
        \end{tikzpicture}
    \end{center}
    \bigskip
\end{example}

需注意,一般情况下 $p_{ii}^{(nd(i))}(n\in\mathbb N^*)$ 不一定大于 $0$,如下例。

\begin{example}
    下图中的 Markov 链满足 $p_{11}^{(n)}>0,n=4,6,8,10,12,\cdots$,故 $d(1)=2$,但 $p_{11}^{(2)}=0$。
    \begin{center}
        \begin{tikzpicture}[>=stealth,thick, node distance=.5cm,baseline=(current bounding box.north)]
            \tikzstyle{state} = [circle, draw, minimum size=1cm]
            \node[state] (1) {1};
            \node[state] (2) [above right=of 1] {2};
            \node[state] (3) [below right=of 2] {3};
            \node[state] (4) [below left=of 3] {4};
            \node[state] (5) [below left=of 1] {5};
            \node[state] (6) [left=of 5] {6};
            \node[state] (7) [above left=of 6] {7};
            \node[state] (9) [above left=of 1] {9};
            \node[state] (8) [left=of 9] {8};
            \draw[->] (1) -- (2);
            \draw[->] (2) -- (3);
            \draw[->] (3) -- (4);
            \draw[->] (4) -- (1);
            \draw[->] (1) -- (5);
            \draw[->] (5) -- (6);
            \draw[->] (6) -- (7);
            \draw[->] (7) -- (8);
            \draw[->] (8) -- (9);
            \draw[->] (9) -- (1);
        \end{tikzpicture}
    \end{center}
    \bigskip
\end{example}

但可以证明,$\forall i\in S,\exists N\in\mathrm N^*,\forall n\geq N,p_{ii}^{(nd(i))}>0$。

\begin{theorem}\label{thm:7.3.3}
    若 $i\leftrightarrow j$,则 $d(i)=d(j)$。
\end{theorem}

\begin{proof}
    由 $i\leftrightarrow j$,知 $\exists m,n,p_{ij}^{(m)}>0,p_{ji}^{(n)}>0$,且由 C-K 方程知 $p_{ii}^{(m+n)}\geq p_{ij}^{(m)}p_{ji}^{(n)}>0$。对于所有满足 $p_{jj}^{(s)}>0$ 的 $s\in\mathbb N^*$,都有 $p_{ii}^{(m+n+s)}\geq p_{ij}^{(m)}p_{jj}^{(s)}p_{ji}^{(n)}>0$,则 $d(i)|m+n+s$ 且 $d(i)|m+n$,故 $d(i)|s$,于是 $d(i)|d(j)$。同理可证 $d(j)|d(i)$,故 $d(i)=d(j)$。
\end{proof}

一般地,可根据状态的分类对状态重新编号,实现状态空间的分解,形如 $S=C_1\cup\cdots\cup C_r\cup C_0$,其中 $C_i\cap C_j=\varnothing,\forall i\neq j$,且 $C_0$ 为非常返状态集,而 $C_1,\cdots,C_r$ 为常返互通类,此时转移概率矩阵可写为
$\left[\begin{matrix}
            P_1    & \cdots & O      &   \\
            \vdots & \ddots & \vdots & O \\
            O      & \cdots & P_r    &   \\
                   & W      &        & Q
        \end{matrix}\right]$。
其中,$P_1,\cdots,P_r,Q$ 分别为 $C_1,\cdots,C_r,C_0$ 内部的转移概率矩阵。

\end{document}
