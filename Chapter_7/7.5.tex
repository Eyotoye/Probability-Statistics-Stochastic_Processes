\documentclass[../main.tex]{subfiles}
\begin{document}

\begin{example}
    设连续掷一个六面骰 $60$ 次,得到结果如下:

    \bigskip
    \begin{tabular}{|c|c|c|c|c|c|c|c|}
        \hline
        点数     & 1  & 2  & 3  & 4  & 5  & 6  & 总计 \\
        \hline
        观测频数 & 4  & 6  & 17 & 16 & 8  & 9  & 60   \\
        \hline
        期望频数 & 10 & 10 & 10 & 10 & 10 & 10 & 60   \\
        \hline
    \end{tabular}
    \bigskip

    检验 $H_0:\text{骰子是均匀的},H_1:\text{骰子不均匀}$。下面介绍 \emph{Pearson 拟合优度检验}。取检验统计量 $\chi^2=\sum_{i=1}^k\frac{(O_i-E_i)^2}{E_i}$,其中 $k=6$,$O_i$ 为观测频数,$E_i$ 为期望频数。

    有如下一般性的定理:设 $k$ 为单元数,检验 $H_0:P(X\in\text{第 $i$ 单元})=p_i^0(i=1,\cdots,k)$,若 $H_0$ 为真,则当 $n\rightarrow\infty$ 时,上述检验统计量 $\chi^2$ 的分布收敛于 $\chi^2(k-1)$。

    代入表中数据,得检验统计量的观测值为 $\frac{(4-10)^2}{10}+\frac{(6-10)^2}{10}+\cdots=14.2$,故 $p$ 值为 $P(\chi^2\geq14.2)\approx0.014$,其中近似有 $\chi^2\sim\chi^2(5)$。
\end{example}

应用中,需期望频数 $E_i=np_i^0\geq 5$,才可以保证应用上述定理的效果较好。

下面介绍列联表(独立性)检验。

\begin{example}
    设希望研究对某项议题的态度于年龄段是否相互独立,调查结果如下:

    \bigskip
    \begin{tabular}{|c|c|c|c|c|}
        \hline
             & 青 & 中 & 老 & 总计 \\
        \hline
        支持 & 20 & 40 & 20 & 80   \\
        \hline
        反对 & 30 & 30 & 10 & 70   \\
        \hline
        总计 & 50 & 70 & 30 & 150  \\
        \hline
    \end{tabular}
    \bigskip

    检验 $H_0:\text{年龄段与态度相互独立},H_1:\text{年龄段与态度不相互独立}$。设 $p_{ij}$ 表示样本属于 $(i,j)$ 单元的概率,用 $p_{i+},p_{+j}$ 分别表示第 $i$ 行和第 $j$ 列的边际概率,则 $H_0$ 下,$p_{ij}=p_{i+}p_{+j}$。取检验统计量 $\chi^2=\sum_{i=1}^a\sum_{j=1}^b\frac{(O_{ij}-E_{ij})^2}{E_{ij}}$,其中 $a,b$ 分别为行数和列数,$O_{ij}$ 为观测频数,$E_{ij}$ 为期望频数。在 $H_0$ 为真时,可给出 $p_ij$ 的估计 $p_ij^*=(p_{i+}p_{+j})^*=p_{i+}^*p_{+j}^*$,其中 $p_{i+}^*=\frac{O_{i+}}n,p_{+j}^*=\frac{O_{+j}}n$,$O_{i+}$ 为第 $i$ 行的观测频数之和,$O_{+j}$ 为第 $j$ 列的观测频数之和,且 $E_{ij}=np_{ij}\approx np_{ij}^*$。

    当 $n\rightarrow\infty$ 时,上述统计量 $\chi^2$ 的分布收敛于 $\chi^2((a-1)(b-1))$。代入表中数据,得检验统计量的观测值为 $\frac{(20-26.67)^2}{26.67}+\cdots\approx6.12$,故 $p$ 值为 $P(\chi^2\geq6.12)\approx0.0469$,其中近似有 $\chi^2\sim\chi^2(2)$。
\end{example}

\end{document}
