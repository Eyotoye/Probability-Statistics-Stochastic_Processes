\documentclass[../main.tex]{subfiles}
\begin{document}

首先简单介绍一下 Bernoulli 过程。

\begin{definition}\label{def:6.1.1}
    设 $T=\mathbb N^*$,$\{X_n\}_{n=1}^\infty$ 是一列独立同分布的随机变量,且 $P(X_n=1)=p,P(X_n=0)=1-p$,则称 $\{X_n\}_{n=1}^\infty$ 为参数为 $p$ 的\emph{Bernoulli 过程}。
\end{definition}

定义中提到的 $\{X_n\}_{n=1}^\infty$ 相互独立,指的是 $\forall n\in\mathbb N^*$,都有 $X_1,\cdots,X_n$ 相互独立。

如果在每个离散时刻 $n$,事件 $X_n=1$ 即“第 $n$ 次试验成功”对应的是该时刻有一个顾客到达某商店,则 Bernoulli 过程可以理解为一种\emph{到达过程}。本章的主要内容 Poisson 过程也是一种到达过程。

\begin{definition}\label{def:6.1.2}
    称一个随机过程 $\{N(t),t\geq0\}$ 为\emph{计数过程},若满足:
    \begin{enumerate}
        \item $N(0)=0$,且 $N(t)\in\mathbb N$
        \item $\forall t>s\geq0,N(t)\geq N(s)$
        \item $N(t)-N(s)$ 为 $(s,t]$ 时间内发生的事件数
    \end{enumerate}
\end{definition}

\begin{definition}\label{def:6.1.3}
    称一个计数过程 $\{N(t),t\geq0\}$ 为\emph{Poisson 过程},若满足:
    \begin{enumerate}
        \item $N(0)=0$
        \item $\{N(t),t\geq0\}$ 有平稳增量和独立增量
        \item $\exists\lambda>0$,当 $h\rightarrow0$ 时,$P(N(h)=1)=\lambda h+o(h)$
        \item 当 $h\rightarrow0$ 时,$P(N(h)\geq2)=o(h)$
    \end{enumerate}
\end{definition}

利用平稳增量性可知,$\forall t\geq0,P(N(t+h)-N(t)=1)=\lambda h+o(h),P(N(t+h)-N(t)\geq2)=o(h)$。

考虑 $[0,t]$ 时间段,将其分成 $n$ 个长度为 $\frac tn$ 的子区间,当 $n\gg1$ 时,每个小区间上发生 $2$ 次及以上事件的概率趋于 $0$,而发生 $1$ 次事件的概率 $p\approx \lambda\frac tn$,因此近似有 $N(t)\sim B(n,p)$,而我们知道当 $n\rightarrow\infty$ 时,$B(n,p)$ 会趋向于 Poisson 分布,即 $N(t)\sim P(\lambda t)$。稍后我们将严格证明这一结论。

称 $\lambda$ 为 Poisson 过程的\emph{强度}或\emph{到达率}。

\begin{proposition}
    若 $\{N(t),t\geq0\}$ 是 Poisson 过程,则 $\forall t\geq0,N(t)\sim P(\lambda t)$。
\end{proposition}

\begin{proof}
    记 $P_n(t)=P(N(t)=n)\ (n\in\mathbb N)$,则 $P_0(t+h)=P(N(t+h)=0)=P(N(t+h)-N(t)=0,N(t)=0)=P(N(t+h)-N(t)=0)P(N(t)=0)=(1-\lambda h+o(h))P_0(t)$,即 $\frac{P_0(t+h)-P_0(t)}h=-\lambda P_0(t)+\frac{o(h)}{h}$。当 $h\rightarrow0$ 时,$\frac{o(h)}{h}\rightarrow0$,因此 $\frac{P_0(t+h)-P_0(t)}{h}\rightarrow-\lambda P_0(t)$,即 $P_0'(t)=-\lambda P_0(t)$,由边界条件 $P_0(0)=P(N(0)=0)=1$ 解得 $P_0(t)=e^{-\lambda t}=\frac{(\lambda t)^0}{0!}e^{-\lambda t}$。同理,
    \begin{equation*}
        \begin{aligned}
              & P_n(t+h)                                                       \\
            = & P(N(t+h)=n)                                                    \\
            = & P(N(t+h)-N(t)=0,N(t)=n)+P(N(t+h)-N(t)=1,N(t)=n-1)              \\
            + & P(N(t+h)-N(t)\geq2,N(t+h)=n)                                   \\
            = & P(N(t+h)-N(t)=0)P(N(t)=n)+P(N(t+h)-N(t)=1)P(N(t)=n-1)          \\
            + & P(N(t+h)-N(t)\geq2,N(t+h)=n)                                   \\
            = & P(N(h)=0)P_n(t)+P(N(h)=1)P_{n-1}(t)+o(h)                       \\
              & (\because P(N(t+h)-N(t)\geq2,N(t+h)=n)\leq P(N(t+h)-N(t)\geq2) \\
              & =P(N(h)\geq2)=o(h))                                            \\
            = & (1-\lambda h+o(h))P_n(t)+(\lambda h+o(h))P_{n-1}(t)+o(h)       \\
            = & P_n(t)+\lambda h(P_{n-1}(t)-P_n(t))+o(h)
        \end{aligned}
    \end{equation*}
\end{proof}

\end{document}
