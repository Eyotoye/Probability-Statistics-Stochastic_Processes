\documentclass[../main.tex]{subfiles}
\begin{document}

\begin{definition}\label{def:4.6.1}
    记 $M_X(t)=\mathrm E(e^{tX})$,若 $M_X(t)$ 在 $t=0$ 的某邻域内存在,则称其为 $X$ 的\emph{矩母函数}(Moment Generating Function, MGF),否则称 $X$ 的矩母函数不存在。
\end{definition}

\begin{example}
    若 $X\sim Exp(\lambda)$,则 $M_X(t)=\mathrm E(e^{tX})=\int_0^{+\infty}e^{tx}\lambda e^{-\lambda x}\mathrm dx=\frac\lambda{\lambda-t},t<\lambda$。
\end{example}

\begin{example}
    若 $X\sim N(0,1)$,则 $M_X(t)=\mathrm E(e^{tX})=\frac{1}{\sqrt{2\pi}}\int_{-\infty}^{+\infty}e^{tx}e^{-\frac{x^2}2}\mathrm dx=e^{\frac{t^2}2},t\in\mathbb R$。
\end{example}

\begin{proposition}
    矩母函数有如下性质:
    \begin{enumerate}
        \item $M_X(0)\equiv1$
        \item $Y=aX+b$,则 $M_Y(t)=\mathrm E(e^{tY})=\mathrm E(e^{t(aX+b)})=e^{tb}M_X(at)$
    \end{enumerate}
\end{proposition}

\begin{example}
    若 $Y\sim N(\mu,\sigma^2)$,令 $Y=\sigma X+\mu$,则 $X\sim N(0,1)$,故 $M_Y(t)=e^{\mu t}M_X(\sigma t)=e^{\mu t}e^{\frac{(\sigma t)^2}2}=e^{\frac{\sigma^2t^2}2+\mu t},t\in\mathbb R$。
\end{example}

矩母函数可以用于确定矩。

\begin{theorem}\label{thm:4.6.1}
    随机变量 $X$ 的 $n$ 阶(原点)矩与其矩母函数有如下关系:$\mathrm E(X^n)=M_X^{(n)}(0)$。
\end{theorem}

\begin{proof}
    由 Taylor 展开有 $M_X(t)=\sum_{n=0}^{+\infty}M_X^{(n)}(0)\frac{t^n}{n!}$,又 $M_X(t)=\mathrm E(e^{tX})=\mathrm E(\sum_{n=0}^{+\infty}X^n\frac{t^n}{n!})=\sum_{n=0}^{+\infty}\mathrm E(X^n)\frac{t^n}{n!}$,得到结论。
\end{proof}

\begin{example}
    若 $X\sim N(0,1)$,则 $M_X(t)=e^{\frac{t^2}2}=\sum_{n=0}^{+\infty}\frac{(\frac{t^2}2)^n}{n!}=\sum_{n=0}^{+\infty}\frac{(2n)!}{2^nn!}\frac{t^{2n}}{(2n)!}$,因此 $\mathrm E(X^{2n})=\frac{(2n)!}{2^nn!},\mathrm E(X^{2n+1})\equiv0\ (n=0,1,\cdots)$。\\
    由此可以计算 $\mathrm{Var}(X)=\mathrm E(X^2)=1,\mathrm{Kurt}(X)=\mathrm E(X^4)=\frac{4!}{2^2\cdot2!}=3$。
\end{example}

矩母函数还可以用于确定分布。

\begin{theorem}\label{thm:4.6.2}
    若存在 $a>0$,使得 $M_X(t)=M_Y(t),\forall t\in(-a,a)$,则 $X,Y$ 同分布。
\end{theorem}

\begin{example}
    若随机变量 $X$ 的矩母函数 $M_X(t)=\frac12e^{-t}+\frac14+\frac18e^{4t}+\frac18e^{5t}$,则 $X$ 为离散型随机变量,分布表为

    \bigskip
    \begin{tabular}{|c|c|c|c|c|}
        \hline
        $X$ & -1  & 0   & 4   & 5   \\
        \hline
        $P$ & 1/2 & 1/4 & 1/8 & 1/8 \\
        \hline
    \end{tabular}
    \bigskip

    一般地,若离散型随机变量 $X$ 有 PMF $P(X=k)=p_k\ (\sum_k p_k\equiv1)$,则其 MGF 为 $M_X(t)=\mathrm E(e^{tX})=\sum_ke^{tk}p_k$。
\end{example}

注意,各阶矩均相同的随机变量未必同分布。

\begin{example}
    设连续型随机变量 $X_1$ 和 $X_2$ 的 PDF 分别为 $f_1(x)=\frac{1}{\sqrt{2\pi}x}e^{-\frac{(\log x)^2}{2}},x>0$ 和 $f_2(x)=f_1(x)(1+\sin(2\pi\log x)),x>0$($X_1$ 服从对数正态分布),则 $\mathrm E(X_2^n)=\mathrm E(X_1^n)+\int_0^{+\infty}x^nf_1(x)\sin(2\pi\log x)\mathrm dx$,其中后一项通过换元 $y=\log x-n$ 可以证明为 $0$,即 $X_1$ 和 $X_2$ 同矩但不同分布。
\end{example}

下面运用矩母函数,研究独立随机变量和的分布。

\begin{theorem}\label{thm:4.6.3}
    若随机变量 $X,Y$ 独立,$Z=X+Y$,则 $M_Z(t)=M_X(t)M_Y(t)$。
\end{theorem}

\begin{proof}
    $M_Z(t)=\mathrm E(e^{tZ})=\mathrm E(e^{t(X+Y)})=\mathrm E(e^{tX}e^{tY})=M_X(t)M_Y(t)$,其中最后一个等号利用了独立性。
\end{proof}

推而广之,若 $\{X_i\}_{i=1}^n$ 相互独立,$Z=X_1+\cdots+X_n$,则 $M_Z(t)=\prod_{i=1}^nM_{X_i}(t)$。

\begin{example}
    若 $\{X_i\}_{i=1}^n$ 相互独立且服从正态分布,则 $X_1+\cdots+X_n$ 也服从正态分布。\\
    以 $n=2$ 为例说明。设 $X_i\sim N(\mu_i,\sigma_i^2)\ (i=1,2)$,则 $M_{X_1+X_2}(t)=M_{X_1}(t)M_{X_2}(t)=e^{\frac{\sigma_1^2t^2}2+\mu_1t}e^{\frac{\sigma_2^2t^2}2+\mu_2t}=e^{\frac12(\sigma_1^2+\sigma_2^2)+(\mu_1+\mu_2)t}$,对应 $N(\mu_1+\mu_2,\sigma_1^2+\sigma_2^2)$ 的 MGF,再由 MGF 确定分布可得结论。
\end{example}

定义随机向量 $(X_1,\cdots,X_n)$ 的 MGF 为 $M_{X_1,\cdots,X_n}(t_1,\cdots,t_n)=\mathrm E(e^{t_1X_1+\cdots+t_nX_n})$。
% 类似地,若 $M_{X_1,\cdots,X_n}(t_1,\cdots,t_n)$ 与 $M_{Y_1,\cdots,Y_n}(t_1,\cdots,t_n)$ 在原点的某个邻域内相等,则 $(X_1,\cdots,X_n)$ 与 $(Y_1,\cdots,Y_n)$ 同分布。

以下简介类似 MGF 的其他函数:
\begin{enumerate}
    \item \emph{概率母函数}(Probability Generating Function, PGF),仅针对非负整数取值的离散型随机变量 $X$,设其 PMF 为 $P(X=k)=p_k$,则其 PGF 定义为 $\mathrm E(t^X)=\sum_{k=0}^{+\infty}p_kt^k,t\in[-1,1]$,或对于 $t\in(0,1]$,等于 $\mathrm E(e^{X\log t})=M_X(\log t)$。
    \item \emph{特征函数},定义为 $\mathrm E(e^{itX})$,其中 $i^2=-1$。
\end{enumerate}

\end{document}
