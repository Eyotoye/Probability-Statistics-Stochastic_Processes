\documentclass[../main.tex]{subfiles}
\begin{document}

\begin{definition}\label{def:3.3.1}
    对 $n$ 维随机向量 $(X_1,\cdots,X_n)$,若存在 $f:\mathbb R^n\rightarrow [0,+\infty)$,使得 $\forall$ 可测集 $Q\subset\mathbb R^n$,都有 $P((X_1,\cdots,X_n)\in Q)=\int_Qf(x_1,\cdots,x_n)\mathrm dx_1\cdots\mathrm dx_n$,则称 $(X_1,\cdots,X_n)$ 为\emph{连续型随机向量},$f$ 称为其\emph{(联合)概率密度函数}(PDF)。
\end{definition}

\begin{proposition}
    连续随机向量 $(X_1,\cdots,X_n)$ 的 PDF 具有如下性质:
    \begin{enumerate}
        \item $\int_{\mathbb R^n}f(x_1,\cdots,x_n)\mathrm dx_1\cdots\mathrm dx_n\equiv 1$
        \item 以 $n=2$ 为例,$F(x,y)=\int_{-\infty}^x\int_{-\infty}^yf(t,s)\mathrm ds\mathrm dt,f(a,b)=\frac{\partial^2F}{\partial x\partial y}(a,b),\mathrm{a.e.}$
    \end{enumerate}
\end{proposition}

其中 a.e. 表示“almost everywhere”。

\begin{example}
    矩形域上的均匀分布的 PDF:
    $f(x,y)=\left\{\begin{aligned}
            \frac{1}{(b-a)(d-c)} & , & (x,y)\in(a,b)\times(c,d), \\
            0                    & , & \text{其他}.
        \end{aligned}\right.$
\end{example}

\begin{example}\label{BVN}
    二元正态分布 $(X,Y)\sim N(\mu_1,\mu_2,\sigma_1^2,\sigma_2^2,\rho)$ 的 PDF:\\
    $f(x,y)=\frac{1}{2\pi\sigma_1\sigma_2}\frac{1}{\sqrt{1-\rho^2}}e^{-\frac{1}{2(1-\rho^2)}((\frac{x-\mu_1}{\sigma_1})^2+(\frac{y-\mu_2}{\sigma_2})^2-2\rho\frac{x-\mu_1}{\sigma_1}\frac{y-\mu_2}{\sigma_2})},\forall (x,y)\in\mathbb R^2,\sigma_1,\sigma_2>0,|\rho|<1$。\\
    令 $\boldsymbol{x}=\left[\begin{matrix}
                \frac{x-\mu_1}{\sigma_1} \\
                \frac{y-\mu_2}{\sigma_2}
            \end{matrix}\right],W=\frac1{1-\rho^2}\left[\begin{matrix}
                1     & -\rho \\
                -\rho & 1
            \end{matrix}\right]$,$W=A^\mathrm TA$ 为正定矩阵 $W$ 的 Cholesky 分解,则 $-\frac{1}{2(1-\rho^2)}((\frac{x-\mu_1}{\sigma_1})^2+(\frac{y-\mu_2}{\sigma_2})^2-2\rho\frac{x-\mu_1}{\sigma_1}\frac{y-\mu_2}{\sigma_2})=-\frac12\boldsymbol{x}^\mathrm TW\boldsymbol{x}=-\frac12\boldsymbol{x}^\mathrm TA^\mathrm TA\boldsymbol{x}=-\frac12 (A\boldsymbol x)^\mathrm T(A\boldsymbol x)$。\\
    上述 Cholesky 分解的结果为 $A=\frac{1}{\sqrt{1-\rho^2}}\left[\begin{matrix}
                1 & -\rho               \\
                0 & \pm \sqrt{1-\rho^2}
            \end{matrix}\right]
    $ 或 $A=\frac{1}{\sqrt{1-\rho^2}}\left[\begin{matrix}
                -1 & \rho                \\
                0  & \pm \sqrt{1-\rho^2}
            \end{matrix}\right]
    $。
\end{example}

\end{document}
