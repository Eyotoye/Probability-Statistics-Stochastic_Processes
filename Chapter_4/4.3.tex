\documentclass[../main.tex]{subfiles}
\begin{document}

离散型和连续型随机变量的方差分别参见定义~\ref{def:2.2.2} 和定义~\ref{def:2.4.2}。

方差的意义:若 $X$ 为收益率,则 $\mathrm{SD}(X)$ 称为\emph{波动率},刻画了风险的大小。我们定义\emph{变异系数} $\mathrm{CV}=\frac{\mathrm{SD}(X)}{\mu}$,其中 $\mu=\mathrm E(X)\neq0$。

\begin{proposition}
    方差有如下性质:
    \begin{enumerate}
        \item $\mathrm{Var}(C)\equiv 0,C\text{ 为常数}$
        \item $\mathrm{Var}(CX)=C^2\mathrm{Var}(X)$
        \item $\mathrm{Var}(X+Y)=\mathrm{Var}(X)+\mathrm{Var}(Y)+2\mathrm{E}((X-\mathrm{E}(X))(Y-\mathrm{E}(Y)))$,且若 $X,Y$ 独立,则 $\mathrm{E}((X-\mathrm{E}(X))(Y-\mathrm{E}(Y)))=0$
    \end{enumerate}
\end{proposition}

\end{document}
