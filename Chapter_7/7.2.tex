\documentclass[../main.tex]{subfiles}
\begin{document}

首先讨论假设检验中的两类错误。若原假设为真,但拒绝了原假设,则犯了\emph{第 I 类错误},又称\emph{弃真错误}。若原假设为假,但不拒绝原假设,则犯了\emph{第 II 类错误},又称\emph{取伪错误}。两类错误发生的概率分别记作 $P_\theta(\text I)(\theta\in\Theta_0)$ 和 $P_\theta(\text{II})(\theta\in\Theta_1)$。一次决策不会同时犯两种错误。

根据样本作决策,错误不可能根本避免。对于固定的 $n$,调整检验准则时,两种错误发生的概率此消彼长。

\begin{example}
    检验元件是否合格,$H_0$ 和 $H_1$ 分别表示合格与不合格。
    \begin{enumerate}
        \item 若从不拒绝 $H_0$,即总认为元件合格,则 $P_\theta(\text I)=0$,但 $P_\theta(\text{II})=1$。
        \item 一般地,当 $P_\theta(\text I)$ 变小,就意味着我们更不容易拒绝原假设(更谨慎地判断元件不合格),此时不合格元件就更不容易检出,因此 $P_\theta(\text{II})$ 变大。
    \end{enumerate}
\end{example}

进一步讨论两种错误的概率。对于 $\theta\in\Theta_0$,我们有 $P_\theta(\text I)=P_\theta((X_1,\cdots,X_n)\in R)$,将其记为 $\alpha(R)$,即调整拒绝域时,犯第 I 类错误的概率相应变化。对于 $\theta\in\Theta_1$,我们有 $P_\theta(\text{II})=P_\theta((X_1,\cdots,X_n)\in R^c)$,将其记为 $\beta(R)$,即调整拒绝域时,犯第 II 类错误的概率相应变化。若固定 $R$,则 $\alpha(R)$ 和 $\beta(R)$ 都是 $\theta$ 的函数。对于 $\theta\in\Theta_1$,我们将 $(1-\beta(R))$ 称为\emph{功效}(Power)。

利用上述概念,我们之前所做的假设检验“当 $T(X_1,\cdots,X_n)\geq c$ 时拒绝 $H_0$”需要满足的条件 $P_{H_0}(T(X_1,\cdots,X_n)\geq c)\leq\alpha$ 实际上就是犯第 I 类错误的概率不超过 $\alpha$。

在假设检验中,有所谓 \emph{Neyman-Pearson 范式}:固定 $n$,对于预先给定的检验水平 $\alpha\in(0,1)$,首先保证犯第 I 类错误的概率不超过 $\alpha$,再在此限制之下使 $P_\theta(\text{II})(\theta\in\Theta_1)$ 尽可能小。若 $\exists\alpha,\beta>0,\forall\theta\in\Theta_0,\alpha(R)\leq\alpha;\forall\theta\in\Theta_1,\beta(R)\leq\beta$,则 $\alpha,\beta$ 是检验程序的属性,即预先给定的可接受的长期错误率。

此种范式下,$H_0$ 与 $H_1$ 一般来说地位不对等。原假设 $H_0$ 通常是受保护的,若证据不充分则不能予以拒绝;备择假设 $H_1$ 往往是我们真正感兴趣的,又称\emph{研究假设}。

在~\ref{sec:7.1} 节中,我们强调过 Fisher 显著性检验中,“不拒绝”不等于“接受”。但在 Neyman-Perason 检验中,由于强调了两类错误,并量化了其概率,故若 $\beta(R)$ 足够小(即功效 $(1-\beta(R))$ 足够大),则可以接受 $H_0$。

\end{document}
