\documentclass[../main.tex]{subfiles}
\begin{document}

\begin{example}
    某女士声称自己可以区分奶茶的制作方法是先加奶还是先加茶。为检验她的话是否为真,Ronald Fisher 设计了如下实验:分别用两种方法制作各 $4$ 杯奶茶,以随机顺序让女士品尝并鉴别(女士知道两种奶茶各有 $4$ 杯),发现她全部说对了。用 $H$ 表示“该女士无鉴别能力”这一假设,则在 $H$ 成立的前提下,该女士只能随机猜测哪 $4$ 杯是先加奶的,能全猜对的概率为 $\frac1{\tbinom 84}=\frac 1{70}$。根据\emph{小概率事件原理},即小概率的事件不易发生,于是我们相信 $H$ 不成立,即该女士有鉴别能力。
\end{example}

那么一个自然而然的问题是:概率要多小才算小呢?通常,我们结合实际情况选取阈值 $\alpha=0.05,0.01,0.1$ 等,称之为\emph{显著性水平}。

上例中,若女士只说对了 $3$ 杯,那么 $H$ 成立的前提下,能猜对至少 $3$ 杯的概率为 $\frac{17}{70}\approx0.243$。形象地说,这一概率即“出现比实际结果更极端的结果的概率”,称为 \emph{$p$ 值}。由于 $p>\alpha$,因此不能轻易否定 $H$,即不能轻易认为女士有鉴别能力。

这种方法称为 \emph{Fisher 显著性检验}。注意到,若我们认可某组观测(样本)的效力,则用它来证实和证伪某个理论(断言)具有天然的不对等,因为即使 $p$ 值不小,我们也不能断言该理论(断言)成立,只能说该理论(断言)在这组观测下没有被证伪。因此,用 Fisher 显著性检验证伪比证实更容易。

通过这个例子我们看到,可以将假设 $H$ 模型化,计算出 $H$ 成立的前提下的各种情况的概率,如记女士猜对的杯数为随机变量 $X$,则 $P(X=k)=\frac{\tbinom 4k\tbinom 4{4-k}}{\tbinom 84}(k\in\{0,1,2,3,4\})$。

历史上,先后提出了 Fisher 显著性检验、Neyman-Pearson 检验和零假设显著性检验(NHST)。

统计学上的假设(\emph{统计假设})是对一个或多个总体的某种断言或猜测,分为 $H_0$ 和 $H_1$,分别称之为\emph{原假设}或\emph{零假设}(Null Hypothesis)和\emph{备择假设}(Alternative Hypothesis)。原假设 $H_0$ 是被检验的假设,而备择假设 $H_1$ 是拒绝 $H_0$ 后可供选择的假设。

一种常见情形是假设可表示为参数形式,即 $H_0:\theta\in\Theta_0,H_1:\theta\in\Theta_1,\Theta_0\cap\Theta_1=\varnothing$,且 $\Theta_0\cup\Theta_1$ 为 $\theta$ 的所有可能取值之集合。

\end{document}
