\documentclass[zihao=-4,linespread=1.5,a4paper,heading=true,oneside]{ctexbook}
% \usepackage[utf8]{inputenc}
\pagestyle{empty}
\setlength{\headheight}{14.5pt}
\usepackage{xcolor}
\usepackage{amsmath}
\usepackage{blkarray}
\usepackage{tikz}
\usetikzlibrary{automata, positioning, arrows}
\usepackage{amssymb}
\usepackage{bm}
\usepackage{calrsfs}
\usepackage{stmaryrd}
\usepackage{graphicx}
\graphicspath{ {./images/}{./images/} }
\usepackage{wrapfig}
\usepackage{subcaption}
\usepackage{capt-of}
\usepackage{cutwin}
\usepackage{geometry}
\geometry{
a4paper,
total={171.8mm,246.2mm},
left=19.1mm,
top=25.4mm,
}
\usepackage{fancyhdr}
\pagestyle{fancy}
\fancyhead{}
\fancyhead[R]{\leftmark}
\fancyhead[L]{\rightmark}
\fancyfoot{}
\fancyfoot[C]{\thepage}

\usepackage{csquotes}
\usepackage{array}
\usepackage{enumitem}
\setlist{nosep}

\usepackage{hyperref}
\hypersetup{colorlinks=true}

% \renewcommand{\thefootnote}{\fnsymbol{footnote}}
% \usepackage{perpage}
% \MakePerPage{footnote}
\usepackage[perpage,symbol*]{footmisc}

% \usepackage[
% backend=biber,
% style=gb7714-2015,
% gbpub=false,
% sorting=none
% ]{biblatex}
% \addbibresource{./ref/ref.bib}

\usepackage{amsthm}

\theoremstyle{definition}
\newtheorem{definition}{定义}
\newtheorem*{definition*}{定义}
\newtheorem{theorem}{定理}
\newtheorem*{theorem*}{定理}
\newtheorem{lemma}{引理}
\newtheorem{proposition}{命题}
\newtheorem{corollary}{推论}[theorem]
% \let\oldproof\proof
% \renewcommand{\proof}{\color{gray}\oldproof}
% \theoremstyle{plain}
\newtheorem{example}{例}
\newtheorem*{example*}{例}

\usepackage{chngcntr}
\counterwithin{figure}{chapter}
\counterwithin{equation}{chapter}
\counterwithin{definition}{chapter}
\counterwithin{theorem}{chapter}
\counterwithin{lemma}{chapter}
\counterwithin{example}{chapter}
\counterwithin{proposition}{chapter}
% \counterwithin{chapter}{part}
\setcounter{tocdepth}{2}

\usepackage{subfiles}

\usepackage{datetime2}
\title{概率论与随机过程}
\author{授课教师:唐宏岩}
\date{}

\begin{document}
\maketitle

\pagenumbering{roman}
\chapter*{前言}\label{chap:preface}
\subfile{preface.tex}
\addcontentsline{toc}{chapter}{\nameref{chap:preface}}
\tableofcontents

\newpage
\pagenumbering{arabic}
\part{初等概率论}\label{part:1}

\chapter{事件的概率}\label{chap:1}
\section{概率的发展史}\label{sec:1.1}
\subfile{Chapter_1/1.1.tex}
\section{随机试验与事件}\label{sec:1.2}
\subfile{Chapter_1/1.2.tex}
\section{事件的运算}\label{sec:1.3}
\subfile{Chapter_1/1.3.tex}
\section{概率的几种解释}\label{sec:1.4}
\subfile{Chapter_1/1.4.tex}
\section{概率的公理化定义}\label{sec:1.5}
\subfile{Chapter_1/1.5.tex}
\section{条件概率}\label{sec:1.6}
\subfile{Chapter_1/1.6.tex}
\section{事件的独立性}\label{sec:1.7}
\subfile{Chapter_1/1.7.tex}
\section{Bayes 公式}\label{sec:1.8}
\subfile{Chapter_1/1.8.tex}

\chapter{随机变量}\label{chap:2}
\section{一维随机变量}\label{sec:2.1}
\subfile{Chapter_2/2.1.tex}
\section{离散随机变量}\label{sec:2.2}
\subfile{Chapter_2/2.2.tex}
\section{常见离散分布}\label{sec:2.3}
\subfile{Chapter_2/2.3.tex}
\section{连续随机变量}\label{sec:2.4}
\subfile{Chapter_2/2.4.tex}
\section{常见连续分布}\label{sec:2.5}
\subfile{Chapter_2/2.5.tex}
\section{随机变量的函数}\label{sec:2.6}
\subfile{Chapter_2/2.6.tex}

\chapter{联合分布}\label{chap:3}
\section{随机向量}\label{sec:3.1}
\subfile{Chapter_3/3.1.tex}
\section{离散分布}\label{sec:3.2}
\subfile{Chapter_3/3.2.tex}
\section{连续分布}\label{sec:3.3}
\subfile{Chapter_3/3.3.tex}
\section{边际分布}\label{sec:3.4}
\subfile{Chapter_3/3.4.tex}
\section{条件分布}\label{sec:3.5}
\subfile{Chapter_3/3.5.tex}
\section{独立性}\label{sec:3.6}
\subfile{Chapter_3/3.6.tex}
\section{随机向量的函数}\label{sec:3.7}
\subfile{Chapter_3/3.7.tex}

\chapter{随机变量的数字特征}\label{chap:4}
\section{期望}\label{sec:4.1}
\subfile{Chapter_4/4.1.tex}
\section{分位数}\label{sec:4.2}
\subfile{Chapter_4/4.2.tex}
\section{方差}\label{sec:4.3}
\subfile{Chapter_4/4.3.tex}
\section{协方差与相关系数}\label{sec:4.4}
\subfile{Chapter_4/4.4.tex}
\section{矩}\label{sec:4.5}
\subfile{Chapter_4/4.5.tex}
\section{矩母函数}\label{sec:4.6}
\subfile{Chapter_4/4.6.tex}
\section{条件期望}\label{sec:4.7}
\subfile{Chapter_4/4.7.tex}

\chapter{不等式与极限定理}\label{chap:5}
\section{概率不等式}\label{sec:5.1}
\subfile{Chapter_5/5.1.tex}
\section{大数定律}\label{sec:5.2}
\subfile{Chapter_5/5.2.tex}
\section{中心极限定理}\label{sec:5.3}
\subfile{Chapter_5/5.3.tex}

\part{随机过程}\label{part:2}

\chapter*{随机过程引言}\label{chap:preface2}
\subfile{Chapter_6/6.0.tex}

\chapter{Poisson 过程}\label{chap:6}
\section{基本概念}\label{sec:6.1}
\subfile{Chapter_6/6.1.tex}
\section{到达时间与到达间隔}\label{sec:6.2}
\subfile{Chapter_6/6.2.tex}
\section{进一步性质}\label{sec:6.3}
\subfile{Chapter_6/6.3.tex}
\section{Poisson 过程推广}\label{sec:6.4}
\subfile{Chapter_6/6.4.tex}

\chapter{离散时间 Markov 链}\label{chap:7}
\section{基本概念}\label{sec:7.1}
\subfile{Chapter_7/7.1.tex}
\section{Chapman-Kolmogorov 方程}\label{sec:7.2}
\subfile{Chapter_7/7.2.tex}
\section{状态分类}\label{sec:7.3}
\subfile{Chapter_7/7.3.tex}
\section{稳态性质}\label{sec:7.4}
\subfile{Chapter_7/7.4.tex}
\section{可逆性}\label{sec:7.5}
\subfile{Chapter_7/7.5.tex}
\section{MCMC}\label{sec:7.6}
\subfile{Chapter_7/7.6.tex}

\chapter{Brown 运动}\label{chap:8}
\section{基本概念}\label{sec:8.1}
\subfile{Chapter_8/8.1.tex}
\section{简单性质}\label{sec:8.2}
\subfile{Chapter_8/8.2.tex}
\section{首中时与最大值}\label{sec:8.3}
\subfile{Chapter_8/8.3.tex}
\section{变形与推广}\label{sec:8.4}
\subfile{Chapter_8/8.4.tex}
\section{It\"o 积分}\label{sec:8.5}
\subfile{Chapter_8/8.5.tex}

\end{document}
