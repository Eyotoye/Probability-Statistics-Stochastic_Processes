\documentclass[../main.tex]{subfiles}
\begin{document}

\begin{definition}\label{def:1.6.1}
    若 $P(B)>0$,定义\emph{条件概率} $P(A|B)=\frac{P(AB)}{P(B)}$。
\end{definition}

通常,我们计算条件概率的方法有两种:
\begin{enumerate}
    \item 在缩小(受限)的样本空间(要求事件 $B$ 发生)上,考虑事件 $A$ 发生的概率
    \item 根据定义计算
\end{enumerate}

一种常用的形式是 $P(AB)=P(A|B)P(B)=P(B|A)P(A)$,这可以视作是求解两个事件的积的概率的方法(乘法法则)。

\begin{example}
    掷一个均匀六面骰,$\Omega=\{1,2,3,4,5,6\},A=\{2,3,4,5\},B=\{1,3,5\}$,则 $P(A)=4/6,P(B)=3/6,P(AB)=2/6,P(A|B)=\frac{P(AB)}{P(B)}=2/3$。
\end{example}

\begin{example}
    袋子中有 8 个红球和 4 个白球,无放回地取出两个球,利用组合数可知,两个都是红球的概率为 $\frac{\tbinom{8}{2}}{\tbinom{12}{2}}$。\\
    用条件概率可以简化计算:$P(R_1R_2)=P(R_1)P(R_2|R_1)=\frac{8}{12}\times\frac{7}{11}$。
\end{example}

更一般地,我们有 $P(A_1A_2\cdots A_n)=P(A_1)P(A_2|A_1)P(A_3|A_1A_2)\cdots P(A_n|A_1A_2\cdots A_{n-1})$,常用于序贯发生的一系列事件的积的概率求解。

\begin{example}
    回忆上一节的“配对问题”。我们有 $P(A_{i_1}A_{i_2}\cdots A_{i_r})=P(A_{i_1})P(A_{i_2}|A_{i_1})\cdots P(A_{i_r}|A_{i_1}\cdots A_{i_{r-1}})=\frac{1}{n}\times\frac{1}{n-1}\times\cdots\times\frac{1}{n-(r-1)}=\frac{(n-r)!}{n!}$。
\end{example}

\begin{proposition}
    对于给定的事件 $B$,$P(\cdot|B):\mathcal{F}\rightarrow\mathbb{R}$ 是概率函数,即 $(\Omega,\mathcal{F},P(\cdot|B))$ 仍是概率空间。
\end{proposition}

对于上述命题的证明,只需验证 $P(\cdot|B)$ 满足概率的三条公理即可。

这提示我们,条件概率也是一种概率,如果我们将 $P(A)$ 称为观察到事件 $B$ 之前 $A$ 的“先验概率”,则 $P(A|B)$ 就是相应的“后验概率”。

一个常见的迷思是:观测到事件 $A$ 已经发生后,是否可以说事件 $A$ 发生的概率 $P(A)=1$?学过条件概率之后,我们知道答案是否定的,实际上是后验概率 $P(A|A)=1$。

\end{document}
