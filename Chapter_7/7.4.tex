\documentclass[../main.tex]{subfiles}
\begin{document}

\begin{definition}\label{def:7.4.1}
    若 $\boldsymbol\beta=(\beta_0,\cdots,\beta_j,\cdots)$ 为状态空间上的概率分布,即 $\beta_j\geq0,\forall j\in S,\sum_{j\in S}\beta_j\equiv1$,且 $\boldsymbol\beta P=\boldsymbol\beta$,则称 $\boldsymbol\beta$ 为该 Markov 链的\emph{平稳分布}。
\end{definition}

若 $\boldsymbol\beta$ 是平稳分布且 $X_0$ 的分布为 $\boldsymbol\beta$,即 $P(X_0=j)=\beta_j,\forall j\in S$,则 $\forall n\in\mathbb N^*$,都有 $X_n$ 的边际分布是 $\boldsymbol\beta$。

\begin{definition}\label{def:7.4.2}
    设 $i$ 为常返态,定义\emph{平均首返时间} $u_i=\sum_{n=1}^\infty nf_{ii}^{(n)}$,若 $u_i<\infty$,则称 $i$ 为\emph{正常返}的,否则称 $i$ 为\emph{零常返}的。
\end{definition}

\begin{theorem}\label{thm:7.4.1}
    若 $i$ 常返,则 $\lim_{n\rightarrow\infty}p_{ii}^{(nd(i))}=\frac{d(i)}{u_i}$。特别地,若 $u_i=+\infty$,则 $\frac{d(i)}{u_i}=0$。
\end{theorem}

\begin{theorem}\label{thm:7.4.2}
    若 $i$ 为零常返或非常返,则 $\lim_{n\rightarrow\infty}p_{ii}^{(n)}=0$。
\end{theorem}

\begin{proof}
    \mbox{}
    \begin{itemize}
        \item 若 $i$ 为零常返,则 $u_i=+\infty$,由定理~\ref{thm:7.4.1} 知 $\lim_{n\rightarrow\infty}p_{ii}^{(nd(i))}=0$,而对于不被 $d(i)$ 整除的 $m\in\mathbb N^*$,恒有 $p_{ii}^{(m)}=0$。故 $\lim_{n\rightarrow\infty}p_{ii}^{(n)}=0$。
        \item 若 $i$ 为非常返,由 $\sum_{n=0}^\infty p_{ii}^{(n)}<+\infty$ 知 $\lim_{n\rightarrow\infty}p_{ii}^{(n)}=0$。
    \end{itemize}
\end{proof}

\begin{theorem}\label{thm:7.4.3}
    若 $i\leftrightarrow j$ 且同常返,则 $i,j$ 同为正常返或同为零常返。
\end{theorem}

\begin{theorem}\label{thm:7.4.4}
    若 $j$ 为零常返或非常返,则 $\forall i\in S,\lim_{n\rightarrow\infty}p_{ij}^{(n)}=0$。
\end{theorem}

\begin{proof}
    \begin{equation*}
        \begin{aligned}
                 & p_{ij}^{(n)}                                                                    \\
            =    & \sum_{k=1}^nf_{ij}^{(k)}p_{jj}^{(n-k)}                                          \\
            =    & \sum_{k=1}^Mf_{ij}^{(k)}p_{jj}^{(n-k)}+\sum_{k=M+1}^nf_{ij}^{(k)}p_{jj}^{(n-k)} \\
            \leq & \sum_{k=1}^Mf_{ij}^{(k)}p_{jj}^{(n-k)}+\sum_{k=M+1}^nf_{ij}^{(k)}
        \end{aligned}
    \end{equation*}
    由于 $\sum_{k=1}^\infty f_{ij}^{(k)}=f_{ij}\leq1$,故 $\forall\epsilon>0,\exists M\in\mathbb N^*,\sum_{k=M+1}^\infty f_{ij}^{(k)}<\epsilon$,如此选取上式中的 $M$,有 $\lim_{n\rightarrow\infty}p_{ij}^{(n)}\leq\lim_{n\rightarrow\infty}\sum_{k=1}^Mf_{ij}^{(k)}p_{jj}^{(n-k)}+\epsilon$。由定理~\ref{thm:7.4.2} 知 $\lim_{n\rightarrow\infty}p_{jj}^{(n-k)}=0,k=1,\cdots,M$,故 $\lim_{n\rightarrow\infty}p_{ij}^{(n)}\leq\epsilon$,由 $\epsilon$ 的任意性知 $\lim_{n\rightarrow\infty}p_{ij}^{(n)}=0$。
\end{proof}

定理~\ref{thm:7.4.4} 的直观理解为,若 $j$ 不是正常返的,则从 $i$ 出发经足够长时间后处于 $j$ 的概率很小。

可以证明,有限链不可能有零常返态,故不可约有限链的所有状态均为正常返的。并且若链有零常返态,则必有无穷多个零常返态。

\begin{definition}\label{def:7.4.3}
    若 $i$ 为正常返的和非周期的,则称 $i$ 为\emph{遍历}(ergodic)的。若链不可约,且所有状态都是遍历的,则称链为\emph{遍历}的。
\end{definition}

\begin{theorem}\label{thm:7.4.5}
    对不可约、非周期的 Markov 链,有以下结论:
    \begin{enumerate}
        \item 若所有状态均为正常返,则 $\pi_j=\lim_{n\rightarrow\infty}p_{ij}^{(n)}(j\in S)$ 是唯一的平稳分布
        \item 若所有状态均为零常返或非常返的,则平稳分布不存在
    \end{enumerate}
\end{theorem}

\begin{proof}
    仅证明结论 1。不妨设 $S=\{0,1,\cdots\}$,则 $\forall M\in\mathbb N^*,\sum_{j=0}^Mp_{ij}^{(n)}\leq\sum_{j=0}^\infty p_{ij}^{(n)}=1$,令 $n\rightarrow\infty$ 有 $\sum_{j=0}^M\pi_j\leq1$,再令 $M\rightarrow\infty$ 有 $\sum_{j=0}^\infty\pi_j\leq1$。又由 $\forall M\in\mathbb N^*,p_{ij}^{(n+1)}=\sum_{k=0}^\infty p_{ik}^{(n)}p_{kj}\geq\sum_{k=0}^Mp_{ik}^{(n)}p_{kj}$,令 $n\rightarrow\infty$ 有 $\pi_j\geq\sum_{k=0}^M\pi_kp_{kj}$,再令 $M\rightarrow\infty$ 有 $\pi_j\geq\sum_{k=0}^\infty\pi_kp_{kj}$。若此不等式对某个 $j$ 取严格大于号,则 $\sum_{j=0}^\infty\pi_j>\sum_{j=0}^\infty\sum_{k=0}^\infty\pi_kp_{kj}=\sum_{k=0}^\infty\pi_k\sum_{j=0}^\infty p_{kj}=\sum_{k=0}^\infty\pi_k$,矛盾。故 $\pi_j=\sum_{k=0}^\infty\pi_kp_{kj},\forall j\in S$,即 $\boldsymbol\pi=\boldsymbol\pi P$。进而,有 $\boldsymbol\pi=\boldsymbol\pi P^n$,即 $\pi_j=\sum_{k=0}^\infty\pi_kp_{kj}^{(n)},\forall j\in S$,令 $n\rightarrow\infty$ 有 $\pi_j=\pi_j\sum_{k=0}^\infty\pi_k,\forall j\in S$,故 $\sum_{k=0}^\infty\pi_k=1$,至此证明了 $\boldsymbol\pi$ 是平稳分布。\\
    下证唯一性。若 $\boldsymbol\beta=(\beta_0,\beta_1,\cdots)$ 为任一平稳分布,有 $\boldsymbol\beta=\boldsymbol\beta P=\cdots=\boldsymbol\beta P^n$,即 $\beta_j=\sum_{k=0}^\infty\beta_kp_{kj}^{(n)},\forall j\in S$,令 $n\rightarrow\infty$ 有 $\beta_j=\sum_{k=0}^\infty\beta_k\pi_j=\pi_j,\forall j\in S$,故 $\boldsymbol\beta=\boldsymbol\pi$。
\end{proof}

\begin{proposition}
    若 $i\leftrightarrow j$ 且 $j$ 为遍历的,则 $\lim_{n\rightarrow\infty}p_{ij}^{(n)}=\frac1{u_j}$。
\end{proposition}

% 对上述命题的一种不严谨的直观理解:$p_{ij}^{(n)}=\sum_{k=1}^nf_{ij}^{(k)}p_{jj}^{(n-k)}$,又由定理~\ref{thm:7.4.1},$\lim_{n\rightarrow\infty}p_{jj}^{(n-k)}=\frac1{u_j}$,故 $\lim_{n\rightarrow\infty}p_{ij}^{(n)}=\frac1{u_j}\sum_{k=1}^\infty f_{ij}^{(k)}=\frac{f_{ij}}{u_j}=\frac1{u_j}$。

\begin{definition}\label{def:7.4.4}
    若 $\pi_j=\lim_{n\rightarrow\infty}p_{ij}^{(n)}$ 存在,则 $\boldsymbol\pi=(\pi_0,\cdots)$ 称为该链的\emph{极限分布}。
\end{definition}

于是定理~\ref{thm:7.4.5} 的结论 1 可以表述为:不可约遍历链的极限分布存在,且为其唯一平稳分布。

若极限分布 $\boldsymbol\pi$ 存在,则对于任意函数 $g$,有 $\frac1N\sum_{n=0}^{N-1}g(X_n)\overset{\mathrm{a.s.}}\rightarrow\mathrm E_{\boldsymbol\pi}(g(X))=\sum_{j=0}^\infty g(j)\pi_j$。由于 $\lim_{n\rightarrow\infty}P^{(n)}=\lim_{n\rightarrow\infty}P^n=
    \left[\begin{matrix}
            \pi_0  & \pi_1  & \cdots & \pi_j  & \cdots \\
            \pi_0  & \pi_1  & \cdots & \pi_j  & \cdots \\
            \vdots & \vdots & \ddots & \vdots & \ddots \\
        \end{matrix}\right]=
    \left[\begin{matrix}
            \boldsymbol\pi \\
            \boldsymbol\pi \\
            \vdots
        \end{matrix}\right]$,故对任意初始分布 $\boldsymbol\beta$,有 $\lim_{n\rightarrow\infty}\boldsymbol\beta P^{(n)}=\boldsymbol\beta\left[\begin{matrix}
            \boldsymbol\pi \\
            \boldsymbol\pi \\
            \vdots
        \end{matrix}\right]=\boldsymbol\pi$。

需要强调,一个链有平稳分布并不意味着其极限分布存在,例如 $P=
    \left[\begin{matrix}
            0 & 1 \\
            1 & 0
        \end{matrix}\right]$,其平稳分布为 $\boldsymbol\pi=(\frac12,\frac12)$,但 $\lim_{n\rightarrow\infty}P^{(n)}$ 不存在。

可以证明:
\begin{enumerate}
    \item 有限链总存在平稳分布
    \item 不可约链的平稳分布若存在则唯一
\end{enumerate}

\begin{example}
    PageRank 是 Page 和 Brin 在 1998 年的论文中提出的对搜索引擎搜索结果中的网页进行排名的一种算法,曾是 Google 搜索引擎的核心算法。设有 $4$ 个网页间的链接关系如下图。
    \begin{center}
        \begin{tikzpicture}[>=stealth,thick, node distance=.5cm,baseline=(current bounding box.north)]
            \tikzstyle{state} = [circle, draw, minimum size=1cm]
            \node[state] (1) at (0,0) {1};
            \node[state] (2) [right=of 1] {2};
            \node[state] (3) [below=of 1] {3};
            \node[state] (4) [below=of 2] {4};
            \draw[->] (1) -- (3);
            \draw[->] (2) -- (4);
            \draw[->] (1) -- (4);
            \draw[->] (3) -- (4);
            \draw[->] ([yshift=2pt]1.east) -- ([yshift=2pt]2.west);
            \draw[->] ([yshift=-2pt]2.west) -- ([yshift=-2pt]1.east);
        \end{tikzpicture}
    \end{center}
    \bigskip
    则构造转移概率矩阵 $P=
        \left[\begin{matrix}
                0   & \frac13 & \frac13 & \frac13       \\
                \frac12   & 0       & 0       & \frac12       \\
                0   & 0       & 0       & 1       \\
                \frac14 & \frac14 & \frac14       & \frac14
            \end{matrix}\right]$,其中第四行为均匀分布,是因为从网页 $4$ 出发没有任何链接,故等概率地跳转到任一网页。一般地,令 $M$ 为页面总数,则 $P$ 为 $M\times M$ 矩阵,若 $\boldsymbol\pi$ 为其平稳分布(假设存在且唯一),则访问概率 $\pi_j$ 可用于衡量页面 $j$ 的重要程度。\\
            但是,$P$ 本身可能可约,这种情况下不清楚平稳分布唯一性。为此,取 $\tilde P=\alpha P+(1-\alpha)\frac JM$,其中 $\alpha\in(0,1)$,而 $J$ 为 $M\times M$ 矩阵且所有元素均为 $1$,即将 $P$ 与平凡的均匀概率转移进行加权平均。则 $\tilde P$ 不可约且为有限链,故所有状态正常返,再加之非周期性(因 $\tilde p_{ii}>0,\forall i\in S$),于是根据定理~\ref{thm:7.4.5},存在唯一平稳分布且为其极限分布,也记作 $\boldsymbol\pi$。\\
            原始论文中建议 $\alpha=0.85$。由于 $M$ 极大,直接求解 $\boldsymbol\pi$ 困难,故可以利用 Markov 链求解,即随机选取一个初始概率分布 $\boldsymbol\beta$,利用 $\boldsymbol\beta\tilde P=\alpha\boldsymbol\beta P+(1-\alpha)\boldsymbol\beta\frac JM$,其中 $\boldsymbol\beta\frac JM=(\frac1M,\cdots,\frac1M)$,迭代计算 $\boldsymbol\beta\tilde P^n$,当 $n$ 足够大时,即近似为 $\boldsymbol\pi$,据此给出网页的排名。
\end{example}

\end{document}
