\documentclass[../main.tex]{subfiles}
\begin{document}

\begin{definition}\label{def:6.4.1}
    $\forall\alpha\in(0,1)$,$\hat\theta_i=\hat\theta_i(X_1,\cdots,X_n)(i=1,2)$ 为统计量,若 $P(\hat\theta_1<\theta<\hat\theta_2)\geq1-\alpha$,则称 $(\hat\theta_1,\hat\theta_2)$ 为 $\theta$ 的一个 $(1-\alpha)$-置信的\emph{(双侧)区间估计}。
\end{definition}

$(1-\alpha)$ 称为\emph{置信水平},\emph{置信系数}或\emph{置信度}是指置信水平中的最大者,这三个术语都是针对方法而言的。$\alpha$ 通常取 $0.05,0.01,0.1$ 等。

通常用 $\mathrm E(\hat\theta_2-\hat\theta_1)$ 来刻画区间估计的精度。我们遵循可靠度优先原则,即先保证置信水平,然后再提升精度。

\begin{example}
    设 $X_1,\cdots,X_n$ 独立同分布,$X_i\sim N(\mu,\sigma^2)$,$\mu$ 未知,$\sigma^2$ 已知,则由 $\bar X\sim N(\mu,\frac{\sigma^2}n)$,有 $\bar X-\mu\sim N(0,\frac{\sigma^2}n)$。为给出 $\mu$ 的区间估计,我们的目标是寻找 $c_1,c_2$ 使得 $P(\bar X-c_1<\mu<\bar X+c_2)\geq1-\alpha$,这等价于 $P(-c_2<\bar X-\mu<c_1)\geq1-\alpha$。设 $\alpha_1=P(\bar X-\mu\leq-c_2),\alpha_2=P(\bar X-\mu\geq c_1)$,一个自然的选择是令 $\alpha_1=\alpha_2=\alpha/2$(事实上这也是能够使精度最高的选择)。记 $z_{\frac\alpha2}$ 为 $N(0,1)$ 的上 $\frac\alpha2$-分位数,即 $\Phi(z_{\frac\alpha2})=1-\frac\alpha2$,则 $P(\left|\frac{\bar X-\mu}{\sigma/\sqrt n}\right|\leq z_{\frac\alpha2})=1-\alpha$,从而 $P(\bar X-z_{\frac\alpha2}\frac\sigma{\sqrt n}<\mu<\bar X+z_{\frac\alpha2}\frac\sigma{\sqrt n})=1-\alpha$,故 $(\bar X-z_{\frac\alpha2}\frac\sigma{\sqrt n},\bar X+z_{\frac\alpha2}\frac\sigma{\sqrt n})$ 是 $\mu$ 的一个 $(1-\alpha)$-置信的区间估计。
\end{example}

若 $\alpha=0.05$,则 $z_{\frac\alpha2}\approx1.96\approx2$。

上述区间估计的一种理解是:若用 $\bar X$ 来估计 $\mu$,则绝对误差 $|\bar X-\mu|$ 在 $(1-\alpha)$-置信下不超过 $z_{\frac\alpha2}\frac\sigma{\sqrt n}$。

区间的半长度为 $z_{\frac\alpha2}\frac\sigma{\sqrt n}$,如果给定精度,例如取 $\epsilon>0$,要求 $z_{\frac\alpha2}\frac\sigma{\sqrt n}\leq\epsilon$,则 $n\geq(\frac{z_{\frac\alpha2}\sigma}\epsilon)^2$,即样本容量至少为 $(\frac{z_{\frac\alpha2}\sigma}\epsilon)^2$ 时有 $(1-\alpha)$-置信使绝对误差不超过 $\epsilon$。这一推理可以理解为 $(\alpha,\epsilon,n)$ 三个变量之间存在的关系。

\begin{example}
    设 $X_1,\cdots,X_n$ 独立同分布,$X_i\sim N(\mu,\sigma^2)$,$\mu,\sigma^2$ 未知,首先估计 $\sigma^2$。注意到,$\frac{(n-1)S^2}{\sigma^2}=\sum_{i=1}^n\left(\frac{X_i-\bar X}{\sigma}\right)^2=\sum_{i=1}^n\left(\frac{X_i-\mu}{\sigma}\right)^2-\left(\frac{\bar X-\mu}{\frac\sigma{\sqrt{n}}}\right)^2\sim\chi^2(n-1)$,同样令 $\alpha_1=\alpha_2=\alpha/2$,有 $(\frac{(n-1)S^2}{\chi^2_{\frac\alpha2}(n-1)},\frac{(n-1)S^2}{\chi^2_{1-\frac\alpha2}(n-1)})$ 是 $\sigma^2$ 的一个 $(1-\alpha)$-置信的区间估计,其中 $\chi^2_{\frac\alpha2}(n-1)$ 和 $\chi^2_{1-\frac\alpha2}(n-1)$ 分别为 $\chi^2(n-1)$ 的上 $\frac\alpha2$-分位数和下 $\frac\alpha2$-分位数。\\
    接下来估计 $\mu$,可以证明,$\frac{\bar X-\mu}{\frac\sigma{\sqrt{n}}}\sim N(0,1)$ 且与 $\frac{(n-1)S^2}{\sigma^2}$ 独立,从而 $\frac{\frac{\bar X-\mu}{\frac\sigma{\sqrt{n}}}}{\sqrt{\frac{\frac{(n-1)S^2}{\sigma^2}}{n-1}}}=\frac{\bar X-\mu}{\frac S{\sqrt{n}}}\sim t(n-1)$,故 $(\bar X-t_{\frac\alpha2}(n-1)\frac S{\sqrt{n}},\bar X+t_{\frac\alpha2}(n-1)\frac S{\sqrt{n}})$ 是 $\mu$ 的一个 $(1-\alpha)$-置信的区间估计,其中 $t_{\frac\alpha2}(n-1)$ 为 $t(n-1)$ 的上 $\frac\alpha2$-分位数。
\end{example}

\begin{example}
    若 $X\sim N(\mu_1,\sigma^2),Y\sim N(\mu_2,\sigma^2)$,且 $X,Y$ 独立,下面估计均值差 $\mu_1-\mu_2$。设随机样本为 $X_1,\cdots,X_n$ 和 $Y_1,\cdots,Y_m$,则 $\bar X-\bar Y\sim N(\mu_1-\mu_2,\frac{\sigma^2}n+\frac{\sigma^2}m)$,有 $\frac{(\bar X-\bar Y)-(\mu_1-\mu_2)}{\sigma\sqrt{\frac1n+\frac1m}}\sim N(0,1)$。同时,由 $\frac{\sum_{i=1}^n(X_i-\bar X)^2}{\sigma^2}=\frac{(n-1)S_1^2}{\sigma^2}\sim\chi^2(n-1)$ 和 $\frac{\sum_{i=1}^m(Y_i-\bar Y)^2}{\sigma^2}= \frac{(m-1)S_2^2}{\sigma^2}\sim\chi^2(m-1)$,且 $\frac{(n-1)S_1^2}{\sigma^2}$ 与 $\frac{(m-1)S_2^2}{\sigma^2}$ 独立,有 $\frac{(n-1)S_1^2}{\sigma^2}+\frac{(m-1)S_2^2}{\sigma^2}\sim\chi^2(n+m-2)$,故 $\frac{\frac{(\bar X-\bar Y)-(\mu_1-\mu_2)}{\sigma\sqrt{\frac1n+\frac1m}}}{\sqrt{\frac{\frac{(n-1)S_1^2}{\sigma^2}+\frac{(m-1)S_2^2}{\sigma^2}}{n+m-2}}}=\frac{(\bar X-\bar Y)-(\mu_1-\mu_2)}{S\sqrt{\frac1n+\frac1m}}\sim t(n+m-2)$,其中 $S^2=\frac{(n-1)S_1^2+(m-1)S_2^2}{n-m+2}$,于是 $(\bar X-\bar Y-t_{\frac\alpha2}(n+m-2)S\sqrt{\frac1n+\frac1m},\bar X-\bar Y+t_{\frac\alpha2}(n+m-2)S\sqrt{\frac1n+\frac1m})$ 是 $\mu_1-\mu_2$ 的一个 $(1-\alpha)$-置信的区间估计。
\end{example}

类似点估计,区间估计也有对应的大样本方法,即所谓\emph{渐近置信区间}。

\begin{example}
    (选举问题)\\
    设 $p$ 为未知的真实支持率,样本容量 $n=1200$,其中有 $684$ 人支持,即观测比例为 $\frac{684}{1200}=0.57$,下面给出 $p$ 的一个 $1-\alpha=95\%$ 置信的区间估计。\\
    记 $X_i$ 为第 $i$ 个人的态度,$1$ 表示支持,$0$ 表示不支持,$X_i\sim B(p)(i=1,2,\cdots,n)$ 且独立,记观测比例 $P_n=P_n(X_1,\cdots,X_n)=\frac1n\sum_{i=1}^nX_i=\bar X$,有 $\mathrm E(P_n)=p,\mathrm{Var}(P_n)=\frac{p(1-p)}n$,由 CLT,近似有 $\frac{P_n-p}{\sqrt{\frac{p(1-p)}n}}\sim N(0,1)$。但是,由于 $p$ 未知,则分母上的标准误未知,故我们无法直接利用这一分布给出置信区间。记 $\sigma^2=p(1-p)$,下面采用几种不同方法给出其估计 $\hat\sigma^2$。
    \begin{enumerate}
        \item 用 $S^2=\frac1{n-1}\sum_{i=1}^n(X_i-\bar X)^2$ 估计 $\sigma^2$,于是近似有 $\frac{P_n-p}{\sqrt{\frac{S^2}n}}\sim N(0,1)$,对应的置信区间为 $(P_n-z_{\frac\alpha2}\sqrt{\frac{S^2}n},P_n+z_{\frac\alpha2}\sqrt{\frac{S^2}n})\approx(0.542,0.598)$。
        \item 用 $m_2=\frac1n\sum_{i=1}^n(X_i-\bar X)^2=P_n(1-P_n)$ 估计 $\sigma^2$,于是近似有 $\frac{P_n-p}{\sqrt{\frac{P_n(1-P_n)}n}}\sim N(0,1)$,对应的置信区间为 $(P_n-z_{\frac\alpha2}\sqrt{\frac{P_n(1-P_n)}n},P_n+z_{\frac\alpha2}\sqrt{\frac{P_n(1-P_n)}n})\approx(0.542,0.598)$。
        \item 用 $p(1-p)$ 的最大值 $\frac14$ 来估计 $\sigma^2$,于是近似有 $\frac{P_n-p}{\frac12\sqrt{\frac1n}}\sim N(0,1)$,对应的置信区间为 $(P_n-z_{\frac\alpha2}\frac1{2\sqrt{n}},P_n+z_{\frac\alpha2}\frac1{2\sqrt{n}})\approx(0.542,0.598)$。
    \end{enumerate}
\end{example}

注意我们这里采用了近似分布,因此只能说置信水平近似是 $(1-\alpha)$,且近似的程度取决于总体分布和样本容量 $n$ 的大小。

下面介绍利用 MLE 构建置信区间的方法。

设总体分布的 PDF 或 PMF 为 $f(x;\theta)$,有随机样本 $X_1,\cdots,X_n$,则似然函数 $L(\theta)=\prod_{i=1}^nf(X_i;\theta)$,对数似然函数 $\ell(\theta)=\log L(\theta)=\sum_{i=1}^n\log f(X_i;\theta)$。

\begin{theorem}\label{thm:6.4.1}
    若 $f$ 满足一定的光滑性条件,$\theta^*$ 为 $\theta$ 的 MLE,则存在 $\sigma_n>0$,使得 $\frac{\theta^*-\theta}{\sigma_n}\rightarrow N(0,1)$。
\end{theorem}

根据 Taylor 展开,对于 $\theta^*$ 附近的 $\theta$,有 $0=\ell^\prime(\theta^*)=\ell^\prime(\theta)+\ell^{\prime\prime}(\theta)(\theta^*-\theta)+o(\theta^*-\theta)$,从而 $\theta^*-\theta\approx-\frac{\ell^\prime(\theta)}{\ell^{\prime\prime}(\theta)}$,即 $\sqrt n(\theta^*-\theta)\approx\frac{\frac1{\sqrt n}\ell^\prime(\theta)}{-\frac1n\ell^{\prime\prime}(\theta)}$。

由 $\frac1{\sqrt n}\ell^\prime(\theta)=\frac1{\sqrt n}\sum_{i=1}^n\frac{\partial\log f(X_i;\theta)}{\partial\theta}=\frac1{\sqrt n}\sum_{i=1}^n\frac{f_\theta(X_i;\theta)}{f(X_i;\theta)}$,其中 $f_\theta$ 表示 $f$ 对 $\theta$ 的偏导数,记 $Y_i=\frac{f_\theta(X_i;\theta)}{f(X_i;\theta)}$,则 $Y_1,\cdots,Y_n$ 独立同分布,且 $\mathrm E(Y_i)=\mathrm E\left(\frac{f_\theta(X_i;\theta)}{f(X_i;\theta)}\right)=\int_{-\infty}^{+\infty}\frac{f_\theta(x;\theta)}{f(x;\theta)}f(x;\theta)\,\mathrm dx=\int_{-\infty}^{+\infty}f_\theta(x;\theta)\,\mathrm dx=\frac{\partial}{\partial\theta}\int_{-\infty}^{+\infty}f(x;\theta)\,\mathrm dx=\frac{\partial}{\partial\theta}1=0$,$\mathrm{Var}(Y_i)=\mathrm E(Y_i^2)=\mathrm E\left(\left(\frac{\partial\log f(X_i;\theta)}{\partial\theta}\right)^2\right)$ 记作 $I(\theta)$。根据 CLT,我们有 $\frac1{\sqrt n}\ell^\prime(\theta)=\frac1{\sqrt n}\sum_{i=1}^nY_i\rightarrow N(0,I(\theta))$。

一般地,我们称 $I_n(\theta)=\mathrm E((\ell^\prime(\theta))^2)=\mathrm E\left(\left(\sum_{i=1}^n\frac{\partial\log f(X_i;\theta)}{\partial\theta}\right)^2\right)$ 为\emph{Fisher 信息量},展开得 $I_n(\theta)=\sum_{i=1}^n\mathrm E\left(\left(\frac{\partial\log f(X_i;\theta)}{\partial\theta}\right)^2\right)+\sum_{i\neq j}\mathrm E\left(\frac{\partial\log f(X_i;\theta)}{\partial\theta}\frac{\partial\log f(X_j;\theta)}{\partial\theta}\right)$,由于 $X_1,\cdots,X_n$ 独立同分布,有 $\mathrm E\left(\frac{\partial\log f(X_i;\theta)}{\partial\theta}\frac{\partial\log f(X_j;\theta)}{\partial\theta}\right)=\mathrm E\left(\frac{\partial\log f(X_i;\theta)}{\partial\theta}\right)\mathrm E\left(\frac{\partial\log f(X_j;\theta)}{\partial\theta}\right)=0$,从而 $I_n(\theta)=nI(\theta)$。

注意到 $\frac{\partial^2\log f(X_i;\theta)}{\partial\theta^2}=\frac\partial{\partial\theta}\left(\frac{f_\theta(X_i;\theta)}{f(X_i;\theta)}\right)=\frac{f_{\theta\theta}(X_i;\theta)f(X_i;\theta)-f_\theta(X_i;\theta)f_\theta(X_i;\theta)}{f^2(X_i;\theta)}=\frac{f_{\theta\theta}(X_i;\theta)}{f(X_i;\theta)}-\left(\frac{f_\theta(X_i;\theta)}{f(X_i;\theta)}\right)^2$,故 $\mathrm E\left(\frac{\partial^2\log f(X_i;\theta)}{\partial\theta^2}\right)=\mathrm E\left(\frac{f_{\theta\theta}(X_i;\theta)}{f(X_i;\theta)}-\left(\frac{f_\theta(X_i;\theta)}{f(X_i;\theta)}\right)^2\right)$,其中 $\mathrm E(\frac{f_{\theta\theta}(X_i;\theta)}{f(X_i;\theta)})=\int_{-\infty}^{+\infty}\frac{f_{\theta\theta}(x;\theta)}{f(x;\theta)}f(x;\theta)\,\mathrm dx=\int_{-\infty}^{+\infty}f_{\theta\theta}(x;\theta)\,\mathrm dx=\frac{\partial}{\partial\theta}\int_{-\infty}^{+\infty}f_\theta(x;\theta)\,\mathrm dx=\frac{\partial}{\partial\theta}0=0$,即 $\mathrm E\left(\frac{\partial^2\log f(X_i;\theta)}{\partial\theta^2}\right)=-\mathrm E\left(\left(\frac{f_\theta(X_i;\theta)}{f(X_i;\theta)}\right)^2\right)=-I(\theta)$。则根据弱大数定律有 $-\frac1n\ell^{\prime\prime}(\theta)=-\frac1n\sum_{i=1}^n\frac{\partial^2\log f(X_i;\theta)}{\partial\theta^2}\overset{P}{\rightarrow}I(\theta)$。

至此,有结论 $\sqrt n(\theta^*-\theta)\approx\frac{\frac1{\sqrt n}\ell^\prime(\theta)}{-\frac1n\ell^{\prime\prime}(\theta)}\rightarrow N(0,\frac1{I(\theta)})$,即 $\frac{\theta^*-\theta}{\sqrt{\frac1{nI(\theta)}}}\rightarrow N(0,1)$,即定理~\ref{thm:6.4.1} 中的 $\sigma_n=\sqrt{\frac1{nI(\theta)}}$。$\theta$ 是未知的,但构造置信区间时 $I(\theta)$ 可以用 $I(\theta^*)$ 估计,即 $\frac{\theta^*-\theta}{\sqrt{\frac1{nI(\theta^*)}}}\rightarrow N(0,1)$。

对选举问题,$f(x;p)=p^x(1-p)^{1-x},I(p)=\mathrm E\left(\left(\frac{\partial\log f(X_i;p)}{\partial p}\right)^2\right)=\mathrm E\left(\left(\frac{X_i-p}{p(1-p)}\right)^2\right)=\frac1{p(1-p)}$,于是 $\frac{P_n-p}{\sqrt{\frac1{nI(P_n)}}}=\frac{P_n-p}{\sqrt{\frac{P_n(1-P_n)}n}}\rightarrow N(0,1)$,据此构造的置信区间与前面的第二种方法相同。

最后介绍一个近似估计两正态总体的均值差的例子。

\begin{example}
    设总体分布为 $X\sim N(\mu_1,\sigma_1^2),Y\sim N(\mu_2,\sigma_2^2)$,$X,Y$ 独立,$\mu_1,\mu_2,\sigma_1^2,\sigma_2^2$ 均未知,随机样本 $X_1,\cdots,X_n;Y_1,\cdots,Y_m$,则 $\frac{(\bar X-\bar Y)-(\mu_1-\mu_2)}{\sqrt{\frac{\sigma_1^2}n+\frac{\sigma_2^2}m}}\sim N(0,1)$。由于 $\sigma_1^2,\sigma_2^2$ 未知,我们用 $S_1^2,S_2^2$ 分别估计之,于是近似有 $\frac{(\bar X-\bar Y)-(\mu_1-\mu_2)}{\sqrt{\frac{S_1^2}n+\frac{S_2^2}m}}\sim N(0,1)$,对应 $\mu_1-\mu_2$ 的置信区间为 $(\bar X-\bar Y-z_{\frac\alpha2}\sqrt{\frac{S_1^2}n+\frac{S_2^2}m},\bar X-\bar Y+z_{\frac\alpha2}\sqrt{\frac{S_1^2}n+\frac{S_2^2}m})$。
\end{example}

\end{document}
