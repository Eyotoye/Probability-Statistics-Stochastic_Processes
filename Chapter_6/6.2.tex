\documentclass[../main.tex]{subfiles}
\begin{document}

设 $(X_1,\cdots,X_n)$ 的联合分布(PMF 或 PDF)为 $f(x_1,\cdots,x_n;\theta)$,其中 $\theta$ 为未知参数(标量或向量)。

对于观测 $X_1,\cdots,X_n$,我们定义\emph{似然函数}为 $L(\theta)=f(X_1,\cdots,X_n;\theta)$。

对于离散情形,$L(\theta)$ 就是出现观测 $(X_1,\cdots,X_n)$ 的概率。

随机变量 $X_1,\cdots,X_n$ 的一个实现通常称为\emph{观测值},记为 $x_1,\cdots,x_n$。

若 $X_1,\cdots,X_n$ 独立同分布,来自总体 $f_1(x;\theta)$(PMF 或 PDF),则 $L(\theta)=\prod_{i=1}^nf_1(x_i;\theta)$。

\begin{example}
    设 $X_1,\cdots,X_n$ 独立同分布,$X_i\sim N(\mu,\sigma^2)$,$\mu$ 和 $\sigma^2$ 未知,则似然函数 $L(\mu,\sigma^2)=\prod_{i=1}^n\frac1{\sqrt{2\pi}\sigma}\e^{-\frac{(x_i-\mu)^2}{2\sigma^2}}$。
\end{example}

\begin{definition}\label{def:6.2.1}
    $\theta^*=\underset{\theta}{\operatorname{argmax}} L(\theta)$ 称为 $\theta$ 的\emph{极大似然估计}(MLE)。
\end{definition}

注意 $\theta^*=\theta^*(X_1,\cdots,X_n)$ 是一个随机变量,因为它是 $X_1,\cdots,X_n$ 的函数。

\begin{example}
    设 $X_1,\cdots,X_n$ 独立同分布,$X_i\sim U(0,\theta),\theta>0$ 未知,$L(\theta)=\left\{
        \begin{aligned}
            \frac1{\theta^n} & , & X_i\in (0,\theta),\forall i, \\
            0                & , & \text{其他},
        \end{aligned}\right.$,则 $\theta^*=\max\{X_1,\cdots,X_n\}$。
\end{example}



\end{document}
