\documentclass[../main.tex]{subfiles}
\begin{document}

\begin{definition}\label{def:4.2.1}
    设 $X$ 为连续型随机变量,若 $P(X\leq m)=F(m)=1/2$,则称 $m$ 为 $X$ 的\emph{中位数}。
\end{definition}

和均值一样,中位数也是随机变量集中趋势的一种刻画。中位数不一定唯一。

若 $m$ 是连续型随机变量 $X$ 的中位数,则 $P(X<m)=P(X>m)=1/2$。

以下给出更一般的中位数定义。

\begin{definition}\label{def:4.2.2}
    对随机变量 $X$,若 $P(X<m)\leq 1/2$,且 $P(X>m)\leq 1-1/2=1/2$,则称 $m$ 为 $X$ 的\emph{中位数}。
\end{definition}

\begin{example}
    设离散型随机变量 $X$ 的分布表为

    \bigskip
    \begin{tabular}{|c|c|c|c|c|}
        \hline
        $X$ & 1   & 2   & 3    & 4    \\
        \hline
        $P$ & 1/3 & 1/2 & 1/12 & 1/12 \\
        \hline
    \end{tabular}
    \bigskip

    则其中位数为 $2$。
\end{example}

\begin{definition}\label{def:4.2.3}
    对随机变量 $X$,$\forall\alpha\in(0,1)$,若 $P(X<a)\leq\alpha$ 且 $P(X>a)\leq1-\alpha$,则称 $a$ 为 $X$ 的\emph{(下侧)$\alpha$-分位数}。
\end{definition}

上述定义的 $\alpha$-分位数是不唯一的。为了唯一性,考虑定义 $F^{-1}(\alpha)=\inf\{x|F(x)\geq\alpha\}$。

我们给出\emph{众数}(mode)的方便定义:$f(x)$ 的最大值点,其中 $f(x)$ 为 PMF 或 PDF。由于 PDF 可在任意零测集上修改取值,故这一定义并非严谨的。

\end{document}
