\documentclass[../main.tex]{subfiles}
\begin{document}

\begin{definition}\label{def:6.3.1}
    称一个随机过程 $\{N(t),t\geq0\}$ 为\emph{计数过程},若其满足:
    \begin{enumerate}
        \item $N(t)\in\mathbb N$
        \item $\forall t>s\geq0,N(t)\geq N(s)$
        \item $N(t)-N(s)$ 为 $(s,t]$ 时间内发生的事件数
    \end{enumerate}
\end{definition}

\begin{definition}\label{def:6.3.2}
    称一个计数过程 $\{N(t),t\geq0\}$ 为 \emph{Poisson 过程},若其满足:
    \begin{enumerate}
        \item $N(0)=0$
        \item $\{N(t),t\geq0\}$ 有平稳增量性和独立增量性
        \item $\exists\lambda>0$,当 $h\rightarrow0$ 时,$P(N(h)=1)=\lambda h+o(h)$
        \item 当 $h\rightarrow0$ 时,$P(N(h)\geq2)=o(h)$
    \end{enumerate}
\end{definition}

利用平稳增量性可知,$\forall t\geq0,P(N(t+h)-N(t)=1)=\lambda h+o(h),P(N(t+h)-N(t)\geq2)=o(h)$。

考虑 $(0,t]$ 时间段,将其分成 $n$ 个长度为 $\frac tn$ 的子区间,当 $n\gg1$ 时,每个小区间上发生 $2$ 次及以上事件的概率趋于 $0$,而发生 $1$ 次事件的概率 $p\approx \lambda\frac tn$,因此近似有 $N(t)\sim B(n,p)$,而我们知道当 $n\rightarrow\infty$ 时,$B(n,p)$ 会趋向于 Poisson 分布,即 $N(t)\sim P(\lambda t)$。稍后我们将严格证明这一结论。

称 $\lambda$ 为 Poisson 过程的\emph{强度}或\emph{到达率}。

\begin{theorem}\label{thm:6.3.1}
    若 $\{N(t),t\geq0\}$ 是 Poisson 过程,则 $\forall t\geq0,N(t)\sim P(\lambda t)$。
\end{theorem}

\begin{proof}
    记 $P_n(t)=P(N(t)=n)\ (n\in\mathbb N)$,则 $P_0(t+h)=P(N(t+h)=0)=P(N(t+h)-N(t)=0,N(t)=0)=P(N(t+h)-N(t)=0)P(N(t)=0)=(1-\lambda h+o(h))P_0(t)$,即 $\frac{P_0(t+h)-P_0(t)}h=-\lambda P_0(t)+\frac{o(h)}{h}$。当 $h\rightarrow0$ 时,$\frac{o(h)}{h}\rightarrow0$,因此 $\frac{P_0(t+h)-P_0(t)}{h}\rightarrow-\lambda P_0(t)$,即 $P_0'(t)=-\lambda P_0(t)$,由边界条件 $P_0(0)=P(N(0)=0)=1$ 解得 $P_0(t)=e^{-\lambda t}=\frac{(\lambda t)^0}{0!}e^{-\lambda t}$。同理,
    \begin{equation*}
        \begin{aligned}
              & P_n(t+h)                                                       \\
            = & P(N(t+h)=n)                                                    \\
            = & P(N(t+h)-N(t)=0,N(t)=n)+P(N(t+h)-N(t)=1,N(t)=n-1)              \\
            + & P(N(t+h)-N(t)\geq2,N(t+h)=n)                                   \\
            = & P(N(t+h)-N(t)=0)P(N(t)=n)+P(N(t+h)-N(t)=1)P(N(t)=n-1)          \\
            + & P(N(t+h)-N(t)\geq2,N(t+h)=n)                                   \\
            = & P(N(h)=0)P_n(t)+P(N(h)=1)P_{n-1}(t)+o(h)                       \\
              & (\because P(N(t+h)-N(t)\geq2,N(t+h)=n)\leq P(N(t+h)-N(t)\geq2) \\
              & =P(N(h)\geq2)=o(h))                                            \\
            = & (1-\lambda h+o(h))P_n(t)+(\lambda h+o(h))P_{n-1}(t)+o(h)       \\
            = & P_n(t)+\lambda h(P_{n-1}(t)-P_n(t))+o(h)
        \end{aligned}
    \end{equation*}
    因此 $\frac{P_n(t+h)-P_n(t)}h=\lambda(P_{n-1}(t)-P_n(t))+\frac{o(h)}h$,当 $h\rightarrow0$ 时,$\frac{P_n(t+h)-P_n(t)}h\rightarrow\lambda(P_{n-1}(t)-P_n(t))$,即 $P_n'(t)=\lambda(P_{n-1}(t)-P_n(t))$。由数学归纳法可证明 $P_n(t)=\frac{(\lambda t)^n}{n!}e^{-\lambda t},\forall n\in\mathbb N$,即 $N(t)\sim P(\lambda t)$。
\end{proof}

于是我们有 $\mathrm E(N(t))=\lambda t,\mathrm{Var}(N(t))=\lambda t$。

下面给出 Poisson 过程的一个等价定义。

\begin{definition}\label{def:6.3.3}
    称一个计数过程 $\{N(t),t\geq0\}$ 为 \emph{Poisson 过程},若其满足:
    \begin{enumerate}
        \item $N(0)=0$
        \item $\{N(t),t\geq0\}$ 有独立增量性
        \item $\exists\lambda>0,\forall t>s\geq0,n\in\mathbb N,P(N(t+s)-N(s)=n)=\frac{(\lambda t)^n}{n!}e^{-\lambda t}$,即 $N(t+s)-N(s)\sim P(\lambda t)$
    \end{enumerate}
\end{definition}

记 $T_1$ 为首次到达时刻,则 $P(T_1\leq t)=1-P(T_1>t)=1-P(N(t)=0)=1-e^{-\lambda t}$,因此 $T_1\sim Exp(\lambda)$。再记 $T_2$ 为第二次到达时刻,则在 $T_1=t_0$ 条件下,$T_2-T_1$ 的分布 $P(T_2-T_1\leq t)=P(T_2-t_0\leq t)=1-P(T_2-t_0>t)=1-P(N(t_0+t)-N(t_0)=0)=1-P(N(t)=0)=1-e^{-\lambda t}$,因此 $T_2-T_1\sim Exp(\lambda)$,且 $T_2-T_1$ 与 $T_1$ 相互独立。一般地,记 $T_i$ 为第 $i$ 次到达时刻,$W_i=T_i-T_{i-1}$ 为相邻两次到达间隔时间,约定 $T_0=0$,则 $\{W_i\}_{i=1}^\infty$ 相互独立且 $W_i\sim Exp(\lambda)$,而 $T_k=\sum_{i=1}^kW_i$,称 $T_k$ 服从参数为 $k$ 和 $\lambda$ 的 \emph{Gamma 分布},记作 $T_k\sim \Gamma(k,\lambda)$,$\mathrm E(T_k)=\frac k\lambda,\mathrm{Var}(T_k)=\frac k{\lambda^2}$。$T_k$ 服从的分布又称之为 \emph{Erlang 分布}。事实上,Erlang 分布是 Gamma 分布在 $k\in\mathbb N^*$ 时的特例。$\Gamma(1,\lambda)$ 就是 $Exp(\lambda)$。

利用 $\forall t\geq0,P(T_k\leq t)=P(N(t)\geq k)=1-\sum_{n=0}^{k-1}\frac{(\lambda t)^n}{n!}e^{-\lambda t}$,求导可得 $T_k$ 的 PDF 为 $f_{T_k}(t)=\frac{\lambda^kt^{k-1}}{(k-1)!}e^{-\lambda t}(t\geq 0)$。

Poisson 过程还有一个等价定义如下。

\begin{definition}\label{def:6.3.4}
    称一个计数过程 $\{N(t),t\geq0\}$ 为 \emph{Poisson 过程},若其满足:
    \begin{enumerate}
        \item $N(0)=0$
        \item $\exists\lambda>0$,各相邻两次到达间隔时间 $\{W_i\}_{i=1}^\infty$ 相互独立且 $W_i\sim Exp(\lambda)$
    \end{enumerate}
\end{definition}

\begin{example}
    拨打服务热线时,被告知除了正在接受服务的人以外,前面还有 $55$ 人在等待。假设呼叫者离开服从 Poisson 过程,$\lambda=2\text{ 人/min}$,则平均等待时间为 $T_{56}=\sum_{i=1}^{56}W_i$,其中 $W_i\sim Exp(2)$ 且相互独立,因此平均等待时间为 $\mathrm E(T_{56})=\frac{56}{2}=28\text{ min}$,且 $\mathrm{Var}(T_{56})=\frac{56}{4}=14\text{ min}^2$。等待时间超过 $30$ 分钟的概率为 $P(T_{56}>30)=\int_{30}^{+\infty}f_{T_{56}}(t)\mathrm dt$。根据 CLT,近似有 $T_{56}\sim N(28,14)$,故有 $P(T_{56}>30)\approx P(Z>\frac{30-28}{\sqrt{14}})=1-\Phi(\frac2{\sqrt{14}})$,其中 $Z\sim N(0,1)$。
\end{example}

\end{document}
