\documentclass[../main.tex]{subfiles}
\begin{document}

离散型和连续型随机变量的期望分别参见定义~\ref{def:2.2.2} 和定义~\ref{def:2.4.2}。

对于随机向量,期望自然推广定义为 $\mathrm E((X_1,\cdots,X_n))=(\mathrm E(X_1),\cdots,\mathrm E(X_n))$。

\begin{proposition}
    期望有如下性质:
    \begin{enumerate}
        \item 离散型和连续型随机向量的函数的期望 $\mathrm E(g(X_1,\cdots,X_n))$ 分别等于\\
              $\sum_{x_i\in\{X_i(\omega)|\omega\in\Omega\},\forall i\in\{1,\cdots,n\}}g(x_1,\cdots,x_n)f(x_1,\cdots,x_n)$\\
              和 $\int_{\mathbb R^n}g(x_1,\cdots,x_n)f(x_1,\cdots,x_n)\mathrm dx_1\cdots\mathrm dx_n$,\\
              其中 $g$ 为可测函数,$f$ 分别为联合 PMF 与联合 PDF
        \item $\mathrm E(aX+bY)=a\mathrm E(X)+b\mathrm E(Y),\forall\text{ 常数}\ a,b\in\mathbb R$
        \item 若 $X_1,\cdots,X_n$ 相互独立,则 $\mathrm E(X_1\cdots X_n)=\mathrm E(X_1)\cdots\mathrm E(X_n)$
    \end{enumerate}
\end{proposition}

\end{document}
