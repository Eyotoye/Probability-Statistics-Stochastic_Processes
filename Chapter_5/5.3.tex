\documentclass[../main.tex]{subfiles}
\begin{document}

\begin{theorem}\label{thm:5.3.1}
    设随机变量 $X_1,\cdots,X_n$ 独立同分布,均值 $\mathrm E(X_i)=\mu$,方差 $\mathrm{Var}(X_i)=\sigma^2>0$,则 $\forall x\in\mathbb R,\lim_{n\rightarrow+\infty}P\left(\frac{\bar X-\mu}{\sigma/\sqrt n}\leq x\right)=\Phi(x)$,其中 $\Phi(x)$ 为标准正态分布的 CDF。或等价地,$\lim_{n\rightarrow+\infty}P\left(\frac{X_1+\cdots+X_n-n\mu}{\sigma\sqrt n}\leq x\right)=\Phi(x)$。
\end{theorem}

\begin{proof}
    只对 $X_i$ 的 MGF 存在的情形给出证明。\\
    不失一般性,假设 $\mu=0,\sigma^2=1$,令 $M(t)=\mathrm E(e^{tX_i})$,则 $M(0)=1,M'(0)=0,M''(0)=1$,于是 $\mathrm E(e^{t\frac{X_1+\cdots+X_n}{\sqrt n}})=M^n\left(\frac t{\sqrt n}\right)$,而根据 Taylor 展开,$M\left(\frac t{\sqrt n}\right)=1+0+\frac12\left(\frac t{\sqrt n}\right)^2+o\left(\frac{t^2}n\right)$,故 $\mathrm E(e^{t\frac{X_1+\cdots+X_n}{\sqrt n}})=(1+\frac{t^2}{2n}+o(\frac{t^2} n))^n\rightarrow e^{t^2/2}(n\rightarrow+\infty)$,此为 $N(0,1)$ 的 MGF,这说明 $\frac{X_1+\cdots+X_n}{\sqrt n}$ 的分布趋近于 $N(0,1)$。
\end{proof}

上述定理通常称为 Lindeberg-Lévy CLT,可推广至不同分布的情形。

如果将定理中的 $\frac{\bar X-\mu}{\sigma/\sqrt n}$ 理解为标准化的过程,则不难得出 $\bar X$ 近似服从 $N(\mu,\frac{\sigma^2}n)$,$X_1+\cdots+X_n$ 近似服从 $N(n\mu,n\sigma^2)$。

\begin{example}
    (De Moivre-Laplace CLT)\\
    设 $X_i\sim B(p)$,则 $\sum_{i=1}^nX_i\sim B(n,p)$,当 $n$ 充分大时,可以近似地认为 $\sum_{i=1}^nX_i\sim N(np,np(1-p))$,于是我们可近似计算 $P(t_1\leq \sum_{i=1}^nX_i\leq t_2)=P\left(\frac{t_1-np}{\sqrt{np(1-p)}}\leq\frac{\sum_{i=1}^nX_i-np}{\sqrt{np(1-p)}}\leq\frac{t_2-np}{\sqrt{np(1-p)}}\right)\approx \Phi(y_2)-\Phi(y_1)$,其中 $y_1=\frac{t_1-np-\frac12}{\sqrt{np(1-p)}},y_2=\frac{t_2-np+\frac12}{\sqrt{np(1-p)}}$,其中 $\frac12$ 是连续性修正项。
\end{example}

\begin{definition}\label{def:5.3.1}
    (依分布收敛)\\
    我们称 $Y_n$ \emph{依分布收敛}于 $Y$,记作 $Y_n\overset{d}\rightarrow Y$,如果 $\lim_{n\rightarrow+\infty}F_{Y_n}(x)=F_Y(x),\forall x\in\mathbb R$。
\end{definition}

用上述定义,CLT 可以表述为 $\frac{\bar X-\mu}{\sigma/\sqrt n}\overset{d}\rightarrow Z$,其中$Z\sim N(0,1)$,或简记为 $\frac{\bar X-\mu}{\sigma/\sqrt n}\rightarrow N(0,1)$。

\begin{example}
    (选举问题)\\
    设 $p$ 为选民真实支持度(未知),随机抽样调查 $n$ 人(假设 $n$ 远远小于总人数 $N$,可以近似有放回抽样),样本支持比例 $P_n=\frac1n\sum_{i=1}^nX_i=\bar X$,其中 $X_i\sim B(p)$ 且独立,表示第 $i$ 个人是否支持。\\
    设置精度 $\epsilon=0.03$,置信度 $1-\alpha=95\%$,则至少需要 $n$ 为多少,才能保证 $P(|P_n-p|<\epsilon)\geq1-\alpha$?\\
    根据 CLT,我们有 $P(|P_n-p|\geq\epsilon)\approx2\left(1-\Phi(\frac\epsilon{\sqrt{p(1-p)/n}})\right)\leq\alpha$,于是 $n\geq\frac{z_{\alpha/2}^2p(1-p)}{\epsilon^2}$,其中 $z_{\alpha/2}$ 为标准正态分布的上 $\alpha/2$ 分位数,代入最大值点 $p=\frac12$,我们得到 $n\geq\frac{z_{\alpha/2}^2}4\epsilon^2$,代入 $\epsilon=0.03,\alpha=0.05$,得到 $n\geq1068$。这一结果与 $N$ 无关!
\end{example}

\end{document}
