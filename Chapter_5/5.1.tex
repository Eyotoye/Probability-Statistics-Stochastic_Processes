\documentclass[../main.tex]{subfiles}
\begin{document}

\begin{theorem}\label{thm:5.1.1}
    (Markov 不等式)\\
    若随机变量 $Y\geq 0$,则 $\forall a>0$,有 $P(Y\geq a)\leq \frac{\mathrm E(Y)}a$。
\end{theorem}

\begin{proof}
    令 $I=\left\{
        \begin{aligned}
            1 & , & Y\geq a, \\
            0 & , & Y<a,
        \end{aligned}\right.$
    则 $I\leq Y/a$,故 $P(Y\geq a)=\mathrm E(I)\leq \mathrm E(Y/a)=\mathrm E(Y)/a$。
\end{proof}

一般地,对于任意随机变量 $X$,则 $\forall a>0$,有 $P(|X|\geq a)\leq\frac{\mathrm E(|X|)}a$。

\begin{example}
    令 $X$ 表示从人群中随机抽的一个人的年收入,$X\geq 0$,则 $P(X\geq k\mathrm E(X))\leq 1/k$。\\
    直观理解是,人群中随机抽到一个其年收入是人均收入 $k$ 倍及以上的人的概率不超过 $1/k$。
\end{example}

\begin{theorem}\label{thm:5.1.2}
    (Chebyshev 不等式)\\
    若随机变量 $Y$ 的方差 $\mathrm{Var}(Y)$ 存在,则 $\forall a>0$ 有 $P(|Y-\mathrm E(Y)|\geq a)\leq \frac{\mathrm{Var}(Y)}{a^2}$。
\end{theorem}



\end{document}
