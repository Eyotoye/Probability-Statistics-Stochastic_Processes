\documentclass[../main.tex]{subfiles}
\begin{document}

\emph{回归问题}可以表述为:已知存在变量关系 $Y=f(X_1,\cdots,X_n)+\epsilon$,其中 $Y$ 称为\emph{因变量}或\emph{响应变量},$X_1,\cdots,X_n$ 称为\emph{自变量}、\emph{回归变量}或\emph{预测变量},而 $\epsilon$ 为随机误差,可能无法测量或不重要,或在建模时尚不清楚。

如果假定 $\mathrm E(\epsilon|X_1,\cdots,X_n)=0$,则 $\mathrm E(Y|X_1,\cdots,X_n)=f(X_1,\cdots,X_n)$,称为 $Y$ 对 $X_1,\cdots,X_n$ 的\emph{(均值)回归函数}。如果有 $(X_1,\cdots,X_n,Y)$ 的样本数据,则可以通过有监督的学习,推断出 $f$ 的信息。

模型中的 $X_1,\cdots,X_n$ 可以是随机的,也可以是非随机的所谓\emph{控制变量}。以下一律约定 $X_1,\cdots,X_n$ 为非随机的。

最后,假设 $\mathrm E(\epsilon)=0,\mathrm{Var}(\epsilon)=\sigma^2$,其中 $\sigma^2$ 未知。影响 $\sigma^2$ 大小的因素包括:
\begin{enumerate}
    \item 模型中的重要因素是否完全
    \item $f$ 的形式是否准确
\end{enumerate}

\end{document}
