\documentclass[../main.tex]{subfiles}
\begin{document}

设 $X_1,\cdots,X_n$ 为独立同分布的样本,我们定义\emph{样本矩}如下:
\begin{enumerate}
    \item $k$ 阶样本原点矩 $\mu_k=\frac1n\sum_{i=1}^nX_i^k$
    \item $k$ 阶样本中心矩 $m_k=\frac1n\sum_{i=1}^n(X_i-\bar X)^k$
\end{enumerate}

根据大数定律,$\mu_k\rightarrow\mathrm E(X^k)$。

矩估计就是用样本矩去估计参数。

\begin{example}
    设 $X_1,\cdots,X_n$ 独立同分布,$X_i\sim N(\mu,\sigma^2)$,则 $\mu=\mathrm E(X)\approx \mu_1=\bar X$,$\sigma^2=\mathrm{Var}(X)\approx m_2=\frac1n\sum_{i=1}^n(X_i-\bar X)^2$。
\end{example}

\begin{example}
    设 $X_1,\cdots,X_n$ 独立同分布,$X_i\sim Exp(\lambda)$,则 $\lambda=\mathrm E(X)^{-1}\approx \mu_1^{-1}=\frac1{\bar X}$,或 $\lambda=\mathrm{Var}(X)^{-1/2}\approx m_2^{-1/2}$。
\end{example}

我们发现上例中 $\lambda$ 可以有两种不同的矩估计,一个基本原则是尽量用低阶矩。

% \begin{example}
%     设 $X_1,\cdots,X_n$ 独立同分布,定义\emph{经验分布函数}为 $F_n(x)=\frac1n\sum_{i=1}^nI_{\{X_i\leq x\}}$,则 $I_{\{X_i\leq x\}}\sim B(p)$,其中 $p=F(x)$,从而 $F_n(x)\overset{\mathrm{a.s.}}\rightarrow F(x)$。
% \end{example}

% 注意到,$F_n(x)$ 的矩就是 $X_1,\cdots,X_n$ 的样本矩。

\end{document}
