\documentclass[../main.tex]{subfiles}
\begin{document}

\begin{definition}\label{def:7.4.1}
    当原假设为真时,称“出现观测值以及更极端的观测的概率”为该检验的 \emph{p 值}。
\end{definition}

其中,所谓“更极端”的具体含义是由 $H_1$ 决定的,这也进一步决定了拒绝域的形状。

\begin{example}
    (选举问题)\\
    设 $p$ 为未知的真实支持率,样本容量 $n=1200$,其中有 $684$ 人支持,即观测比例为 $p_n=\frac{684}{1200}=0.57$。给定 $p_0\in(0,1)$,检验 $H_0:p=p_0,H_1:p>p_0$,取检验统计量为 $P_n=\bar X$,其中 $\{X_i\}_{i=1}^n$ 独立同分布,$X_i\sim B(p)$。由 CLT,近似有 $\frac{P_n-p}{\mathrm{Se}(P_n)}\sim N(0,1)$,其中 $\mathrm{Se}(P_n)=\sqrt\frac{p(1-p)}n$。当 $H_0$ 为真时,$\mathrm{Se}(P_n)=\sqrt\frac{p_0(1-p_0)}n$,故 $p$ 值为 $P(P_n\geq p_n|H_0)=P(\frac{P_n-p_0}{\mathrm{Se}(P_n)}\geq\frac{p_n-p_0}{\mathrm{Se}(P_n)}|H_0)\approx P(Z\geq\frac{p_n-p_0}{\mathrm{Se}(P_n)})$,其中 $Z\sim N(0,1)$。代入 $p_0=0.5$,得 $\mathrm{Se}(P_n)\approx0.014$,$p$ 值 $\ll0.001$;代入 $p_0=0.55$,得 $\mathrm{Se}(P_n)\approx0.014$,$p$ 值 $\approx0.082$。
\end{example}

根据 $p$ 值的定义,容易发现设计检验准则时,拒绝 $H_0$ 的条件为 $p$ 值 $\leq\alpha$。非正式地,$p$ 值通常作为拒绝 $H_0$ 的证据强弱的度量。但需要强调,$p$ 值与 $P(H_0|\text{观测值})$ 完全是两回事。若 $p$ 值不小,则不拒绝 $H_0$,这可能有多种原因,既可能是 $H_0$ 为真,也可能是 $H_0$ 不真,但检验的功效不够大。$p$ 值也不等于“错误拒绝 $H_0$ 的概率”。

上例中,若 $H_0$ 改为 $p\leq p_0$,则 $p$ 值应当修正为 $\sup\limits_{p\leq p_0}P(P_n\geq p_n)$。此时当 $H_0$ 为真时,对 $\mathrm{Se}(P_n)$ 的估计 $\widehat{\mathrm{Se}}(P_n)=\sqrt\frac{P_n(1-P_n)}n$,$p$ 值的计算公式也相应修正为 $P(P_n\geq p_n|H_0)=P(\frac{P_n-p}{\widehat{\mathrm{Se}}(P_n)}\geq\frac{p_n-p}{\widehat{\mathrm{Se}}(P_n)}|H_0)\approx P(Z\geq\frac{p_n-p}{\widehat{\mathrm{Se}}(P_n)})$,其中 $Z\sim N(0,1)$。

对 $p$ 值的定义给出一更具体的修订。

\begin{definition}\label{def:7.4.2}
    若检验准则为拒绝 $H_0:\theta\in\Theta_0$ 当且仅当 $T(X_1,\cdots,X_n)\geq C$,则检验的 \emph{$p$ 值}为 $\sup\limits_{\theta\in\Theta_0}P(T(X_1,\cdots,X_n)\geq T(x_1,\cdots,x_n))$。其中 $T(x_1,\cdots,x_n)$ 为检验统计量的观测值。
\end{definition}

\end{document}
