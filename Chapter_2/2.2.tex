\documentclass[../main.tex]{subfiles}
\begin{document}

\begin{definition}\label{def:2.2.1}
    离散随机变量 $X$ 的\emph{概率质量函数}(Probability Mass Function, PMF)$f$ 是指该随机变量取各个可能值的概率,即 $f(x)=P(X=x),\forall x\in\mathbb{R}$。可以用\emph{分布表}的形式展示各个可能取值与概率的对应关系。
\end{definition}

\begin{proposition}
    如果离散随机变量 $X$ 的所有可能取值为 $\{x_i\}$,则 $X$ 的 PMF 具有如下性质:
    \begin{enumerate}
        \item $f(x_i)=p_i\geq 0,\forall i$
        \item $\sum_i p_i=1$
        \item $F(x)=\sum_{x_i\leq x}f(x_i)$
    \end{enumerate}
\end{proposition}

\begin{definition}\label{def:2.2.2}
    离散随机变量 $X$ 的\emph{期望}定义为 $\mathrm{E}(X)=\sum_ix_ip_i$。\\
    我们称 $X$ 的期望\emph{存在},当且仅当 $\sum_i|x_i|p_i<+\infty$。\\
    当期望存在时,其\emph{方差}定义为 $\mathrm{Var}(X)=\sum_i(x_i-\mathrm{E}(X))^2p_i=\mathrm{E}((X-\mathrm{E}(X))^2)=\mathrm{E}(X^2)-\mathrm{E}^2(X)$。\\
    当方差有限时,称其算术平方根为 $X$ 的\emph{标准差},记作 $\mathrm{SD}(X)$。
\end{definition}

注意,通常我们所说的一个随机变量的\emph{均值}指的就是期望。

\emph{标准化}指的是对 $X$ 作线性变换 $\frac{X-\mu}{\sigma}$,其中 $\mu$ 和 $\sigma$ 分别为 $X$ 的期望和标准差,得到均值为 $0$,标准差为 $1$ 的随机变量。

对于可测函数 $g$,$g(X)$ 也是随机变量,其期望 $\mathrm{E}(g(X))=\sum_ig(x_i)p_i$。

期望反映了随机变量的集中趋势,而方差反映了其分散程度。

\end{document}
