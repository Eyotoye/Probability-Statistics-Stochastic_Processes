\documentclass[../main.tex]{subfiles}
\begin{document}

设随机变量 $X_1,\cdots,X_n$ 独立同分布,均值 $\mathrm E(X_i)=\mu$,方差 $\mathrm{Var}(X_i)=\sigma^2>0$,则样本均值 $\bar X=\frac1n\sum_{i=1}^nX_i$,其均值 $\mathrm E(\bar X)=\mu$,方差 $\mathrm{Var}(\bar X)=\frac{\sigma^2}n\rightarrow0(n\rightarrow+\infty)$。

% 这告诉我们,当样本量足够大时,样本均值 $\bar X$ 的分布会趋近于总体均值 $\mu$。

\begin{theorem}\label{thm:5.2.1}
    (Khinchin 弱大数定律)\\
    设随机变量 $X_1,\cdots,X_n$ 独立同分布,均值 $\mathrm E(X_i)=\mu$,方差 $\mathrm{Var}(X_i)=\sigma^2>0$,则 $\forall\epsilon>0$,有\\
    $\lim_{n\rightarrow+\infty}P(|\bar X-\mu|\geq\epsilon)=0$,或等价地,$\lim_{n\rightarrow+\infty}P(|\bar X-\mu|<\epsilon)=1$。
\end{theorem}

\begin{proof}
    由 Chebyshev 不等式,$P(|\bar X-\mu|\geq\epsilon)\leq\frac{\mathrm{Var}(\bar X)}{\epsilon^2}=\frac{\sigma^2}n\frac1{\epsilon^2}\rightarrow0(n\rightarrow+\infty)$。
\end{proof}

$\forall\epsilon>0,\forall\alpha>0$,如果我们将 $\epsilon$ 和 $1-\alpha$ 分别称为\emph{精度}和\emph{置信度},则根据 Khinchin 弱大数定律,$\exists N\in\mathbb N^+$,当 $n\geq N$ 时,$P(|\bar X-\mu|<\epsilon)\geq1-\alpha$,即 $\bar X$ 至少以概率 $1-\alpha$ 落在区间 $(\mu-\epsilon,\mu+\epsilon)$ 内。

换句话说,当样本量足够大时,有很大的概率 $\bar X\approx\mu$,其中 $\mu$ 为未知的总体均值。

我们将 $X_i\sim B(p)$ 这一特例称之为 \emph{Bernoulli 大数定律}。

通过更进一步的讨论可以证明,上述定理中关于方差的条件可以去掉,结论仍正确。

此外,我们还有对 Khinchin 弱大数定律的若干推广,如
\begin{enumerate}
    \item 要求 $X_i$ 两两不相关,$\mathrm{Var}(X_i)$ 一致有界,我们就得到了 Chebyshev 大数定律;
    \item 要求 $\mathrm{Var}(\bar X)\rightarrow0(n\rightarrow+\infty)$,我们就得到了 Markov 大数定律。
\end{enumerate}

\begin{definition}\label{def:5.2.1}
    我们称 $Y_n$ \emph{依概率收敛}于 $Y$,记作 $Y_n\overset{P}{\rightarrow}Y$,如果 $\forall\epsilon>0$,有 $\lim_{n\rightarrow+\infty}P(|Y_n-Y|\geq\epsilon)=0$。
\end{definition}

用上述定义,弱大数定律可以表述为 $\bar X\overset{P}{\rightarrow}\mu$。

\begin{theorem}\label{thm:5.2.2}
    (Kolmogorov 强大数定律)\\
    设随机变量 $X_1,\cdots,X_n$ 独立同分布,均值 $\mathrm E(X_i)=\mu$,则 $P(\lim_{n\rightarrow+\infty}\bar X=\mu)=1$。
\end{theorem}

考虑 $X_i\sim B(p)$ 的特殊情形,则 $\bar X$ 称之为频率,根据 Kolmogorov 强大数定律,$P(\lim_{n\rightarrow+\infty}\bar X=p)=1$,这说明概率的频率解释是合理的。

\begin{definition}\label{def:5.2.2}
    我们称 $Y_n$ \emph{以概率 $1$ 收敛}于 $Y$,又称\emph{几乎必然收敛}于 $Y$,记作 $Y_n\overset{\mathrm{a.s.}}{\rightarrow}Y$,如果 $P(\lim_{n\rightarrow+\infty}Y_n=Y)=1$。
\end{definition}

用上述定义,强大数定律可以表述为 $\bar X\overset{\mathrm{a.s.}}{\rightarrow}\mu$。

\begin{example}
    (Monte Carlo 积分)\\
    设我们要计算 $g(x)>0$ 在区间 $[a,b]$ 上的定积分,首先取一个适当的 $c>\sup\{g(x)|x\in[a,b]\}$,设 $(X_i,Y_i)$ 独立且服从区域 $[a,b]\times[0,c]$ 上的均匀分布,记 $I_i=\left\{
        \begin{aligned}
            1 & , & Y_i\leq g(X_i), \\
            0 & , & Y_i>g(X_i),
        \end{aligned}
        \right.$,则 $I_i\sim B(p)$,其中 $p=\frac{\int_a^b g(x)\mathrm{d}x}{c(b-a)}$,于是 $\bar I=\frac1n\sum_{i=1}^nI_i\approx p$,从而 $\int_a^b g(x)\mathrm{d}x\approx c(b-a)\bar I$。
\end{example}

\begin{example}
    我们通过一个例子来考察一下上面介绍的两种收敛性的区别。\\
    设概率空间 $(\Omega,\mathcal F,P)$,其中 $\Omega=[0,1]$,$\omega$ 在 $\Omega$ 上均匀分布。定义随机变量序列 $\forall\omega\in\Omega,Y_1(\omega)=\omega+I_{[0,1]}(\omega),Y_2(\omega)=\omega+I_{[0,1/2]}(\omega),Y_3(\omega)=\omega+I_{[1/2,1]}(\omega),Y_4(\omega)=\omega+I_{[0,1/3]}(\omega),Y_5(\omega)=\omega+I_{[1/3,2/3]}(\omega),Y_6(\omega)=\omega+I_{[2/3,1]}(\omega),\cdots$,则 $Y_n(\omega)$ 依概率收敛于 $Y(\omega)=\omega$,但不以概率 $1$ 收敛于 $Y(\omega)$,因为 $\forall \omega_0\in\Omega$,$Y_n(\omega_0)$ 无极限。
    % 另外,如果考察随机变量序列 $Z_n(\omega)=\omega^n$,则 $Z_n(\omega)$ 以概率 $1$ 收敛于 $Z(\omega)=0$。
\end{example}

\end{document}
