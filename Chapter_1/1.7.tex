\documentclass[../main.tex]{subfiles}
\begin{document}

\begin{definition}\label{def:1.7.1}
    若 $P(AB)=P(A)P(B)$,则称事件 $A,B$ 相互\emph{独立}。
\end{definition}

如果 $P(B)>0$,我们注意到 $A,B$ 独立等价于 $P(A|B)=P(A)$。

\begin{proposition}
    若 $A,B$ 独立,则 $A^c,B$ 独立。
\end{proposition}

\begin{definition}\label{def:1.7.2}
    若 $P(ABC)=P(A)P(B)P(C)$,且 $A,B,C$ 两两独立,则称事件 $A,B,C$ \emph{独立}。
\end{definition}

注意,仅有 $A,B,C$ 两两独立,不能推出三者独立。

\begin{definition}\label{def:1.7.3}
    若对于事件列 $\{A_i\}_{i=1}^\infty$,任意取有限个事件 $A_{i_1},A_{i_2},\cdots,A_{i_r}$,都有 $P(A_{i_1}A_{i_2}\cdots A_{i_r})=P(A_{i_1})P(A_{i_2})\cdots P(A_{i_r})$,则称 $\{A_i\}_{i=1}^\infty$ 相互\emph{独立}。
\end{definition}

\begin{example}
    每周开奖的彩票,各次中奖率均为 $10^{-5}$ 且独立,问连续十年(520 周)不中奖的概率?\\
    令事件 $A_i$ 为第 $i$ 周不中奖,则 $P(A_i)=1-10^{-5}$,故 $P(A_1\cdots A_{520})=(1-10^{-5})^{520}\approx 0.9948$。
\end{example}

\begin{definition}\label{def:1.7.4}
    若事件 $A,B,E$ 满足 $P(AB|E)=P(A|E)P(B|E)$,则我们称 $A,B$ 关于 $E$ \emph{条件独立}。
\end{definition}

注意,条件独立性和独立性之间没有蕴涵关系。

\end{document}
