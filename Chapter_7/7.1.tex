\documentclass[../main.tex]{subfiles}
\begin{document}

\begin{example}
    某女士声称自己可以区分奶茶的制作方法是先加奶还是先加茶。为检验她的话是否为真,Ronald Fisher 设计了如下实验:分别用两种方法制作各 $4$ 杯奶茶,以随机顺序让女士品尝并鉴别(女士知道两种奶茶各有 $4$ 杯),发现她全部说对了。用 $H$ 表示“该女士无鉴别能力”这一假设,则在 $H$ 成立的前提下,该女士只能随机猜测哪 $4$ 杯是先加奶的,能全猜对的概率为 $\frac1{\tbinom 84}=\frac 1{70}$。根据\emph{小概率事件原理},即小概率的事件不易发生,于是我们相信 $H$ 不成立,即该女士有鉴别能力。
\end{example}

那么一个自然而然的问题是:概率要多小才算小呢?通常,我们结合实际情况选取阈值 $\alpha=0.05,0.01,0.1$ 等,称之为\emph{显著性水平}。

上例中,若女士只说对了 $3$ 杯,那么 $H$ 成立的前提下,能猜对至少 $3$ 杯的概率为 $\frac{17}{70}\approx0.243$。形象地说,这一概率即“出现比实际结果更极端的结果的概率”,称为 \emph{$p$ 值}。由于 $p>\alpha$,因此不能轻易否定 $H$,即不能轻易认为女士有鉴别能力。

这种方法称为 \emph{Fisher 显著性检验}。注意到,若我们认可某组观测(样本)的效力,则用它来证实和证伪某个理论(断言)具有天然的不对等,因为即使 $p$ 值不小,我们也不能断言该理论(断言)成立,只能说该理论(断言)在这组观测下没有被证伪。因此,用 Fisher 显著性检验证伪比证实更容易。

通过这个例子我们看到,可以将假设 $H$ 模型化,计算出 $H$ 成立的前提下的各种情况的概率,如记女士猜对的杯数为随机变量 $X$,则 $P(X=k)=\frac{\tbinom 4k\tbinom 4{4-k}}{\tbinom 84}(k\in\{0,1,2,3,4\})$。

历史上,先后提出了 Fisher 显著性检验、Neyman-Pearson 检验和零假设显著性检验(NHST)。

统计学上的假设(\emph{统计假设})是对一个或多个总体的某种断言或猜测,分为 $H_0$ 和 $H_1$,分别称之为\emph{原假设}或\emph{零假设}(Null Hypothesis)和\emph{备择假设}(Alternative Hypothesis)。原假设 $H_0$ 是被检验的假设,而备择假设 $H_1$ 是拒绝 $H_0$ 后可供选择的假设。

一种常见情形是假设可表示为参数形式,即 $H_0:\theta\in\Theta_0,H_1:\theta\in\Theta_1,\Theta_0\cap\Theta_1=\varnothing$,且 $\Theta_0\cup\Theta_1$ 为 $\theta$ 的所有可能取值之集合。

\begin{example}
    \mbox{}\\
    设总体分布为 $X\sim N(\mu,\sigma^2)$,其中 $\sigma^2$ 已知,以下是一些原假设与备择假设的例子:
    \begin{enumerate}
        \item $H_0:\mu=\mu_0,H_1:\mu\neq\mu_0$
        \item $H_0:\mu=\mu_0,H_1:\mu>\mu_0$
        \item $H_0:\mu\leq\mu_0,H_1:\mu>\mu_0$
    \end{enumerate}
    设总体分布为 $X\sim N(\mu_1,\sigma^2),Y\sim N(\mu_2,\sigma^2)$,$X,Y$ 独立,$\sigma^2$ 已知,则一组可能的原假设与备择假设为:$H_0:\mu_1=\mu_2,H_1:\mu_1\neq\mu_2$。
\end{example}

我们称只对应一个总体的假设为\emph{简单假设},对应多个总体的假设为\emph{复合假设}。例如上例中的 $H_0:\mu=\mu_0$ 为简单假设,$H_0:\mu\leq\mu_0$ 为复合假设。注意,若上例中的 $\sigma^2$ 未知,则 $H_0:\mu=\mu_0$ 等价于 $H_0:\mu=\mu_0,\sigma^2>0$,这是一个复合假设。

依据样本(观测)对假设进行决策(\emph{拒绝} $H_0$ 或不拒绝 $H_0$)的过程,称为\emph{假设检验}。一个具体的\emph{检验(准则)},就是做出决策的一个具体法则,即在何种情况下拒绝 $H_0$。根据小概率事件原理,若在原假设 $H_0$ 为真的前提下,所观测的样本出现的概率很小,则意味着样本提供了拒绝 $H_0$ 的证据。

考虑所有可能出现的观测之集合 $\{(X_1(\omega),\cdots,X_n(\omega))|\omega\in\Omega\}$,其中样本量 $n$ 固定,则可以按照检验准则将之分为两部分 $R$ 和 $R^c$,其中 $R$ 称为\emph{拒绝域}或\emph{临界域},当样本落在 $R$ 中时,拒绝原假设 $H_0$。一种常见的拒绝域形式为 $R=\{(X_1,\cdots,X_n)|T(X_1,\cdots,X_n)\geq c\}$,其中 $T(X_1,\cdots,X_n)$ 称为\emph{检验统计量},$c$ 称为\emph{临界值}。若对于某个 $\alpha\in(0,1)$,有 $\forall\theta\in\Theta_0,P_\theta(T(X_1,\cdots,X_n)\geq c)\leq\alpha$,则称($R$ 对应的)检验是(显著性)水平为 $\alpha$ 的检验。

\begin{example}
    \mbox{}\\
    设总体分布为 $X\sim N(\mu,\sigma^2)$,其中 $\sigma^2$ 已知。考虑以下两个假设检验。
    \begin{enumerate}
        \item $H_0:\mu=\mu_0,H_1:\mu\neq\mu_0$,这是一个\emph{双侧检验}。对于给定的 $\alpha\in(0,1)$,设检验准则为当 $|\bar X-\mu_0|\geq c$ 时拒绝 $H_0$。这要求当 $H_0$ 为真时,$P_{H_0}(|\bar X-\mu_0|\geq c)\leq\alpha$。由于 $\bar X-\mu\sim N(0,\frac{\sigma^2}n)$,即 $H_0$ 为真时 $\frac{\bar X-\mu_0}{\frac\sigma{\sqrt n}}\sim N(0,1)$,要求为 $P_{H_0}(\frac{|\bar X-\mu_0|}{\frac\sigma{\sqrt n}}\geq\frac c{\frac\sigma{\sqrt n}})\leq\alpha$,因此取 $\frac c{\frac{\sigma}{\sqrt n}}=z_{\frac\alpha 2}$,即 $c=z_{\frac\alpha 2}\frac\sigma{\sqrt n}$。据此确定检验准则:若 $|\bar X-\mu_0|\geq z_{\frac\alpha 2}\frac\sigma{\sqrt n}$,则拒绝 $H_0$。
        \item $H_0:\mu\geq\mu_0,H_1:\mu<\mu_0$,这是一个\emph{单侧检验}。对于给定的 $\alpha\in(0,1)$,设检验准则为当 $\bar X\leq c$ 时拒绝 $H_0$。这要求当 $H_0$ 为真时,$P_{H_0}(\bar X\leq c)\leq\alpha$。当 $H_0$ 为真时 $\frac{\bar X-\mu}{\frac\sigma{\sqrt n}}\sim N(0,1)$,要求为 $P_{\mu\geq\mu_0}(\frac{\bar X-\mu}{\frac\sigma{\sqrt n}}\leq\frac{c-\mu}{\frac\sigma{\sqrt n}})\leq\alpha$,因此 $\frac{c-\mu}{\frac{\sigma}{\sqrt n}}\leq-z_\alpha$,由于要对所有 $\mu\geq\mu_0$ 成立,取 $c=\mu_0-z_\alpha\frac\sigma{\sqrt n}$。据此确定检验准则:若 $\bar X\leq\mu_0-z_\alpha\frac\sigma{\sqrt n}$,则拒绝 $H_0$。
    \end{enumerate}
    本例有时也称为 $Z$-检验。
\end{example}

上例中,若 $\sigma^2$ 未知,则要根据 $\frac{\bar X-\mu}{\frac\sigma{\sqrt n}}\sim t(n-1)$ 来构造检验准则,称为 $t$-检验。

\end{document}
