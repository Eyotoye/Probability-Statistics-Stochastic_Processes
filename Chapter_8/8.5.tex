\documentclass[../main.tex]{subfiles}
\begin{document}

本节介绍关于 Brown 运动的积分。

首先通过一个例子引入形如 $\int_a^bf(B(t))\mathrm dt$ 的均方积分。

\begin{example}
    $\int_0^tB(u)\mathrm du$ 称之为\emph{积分 Brown 运动}。设有分划 $0=t_0<t_1<\cdots<t_n=t$,记 $\lambda_n=\max_{k=1,\cdots,n}\{t_k-t_{k-1}\}$,则 $\lim_{n\rightarrow\infty}\lambda_n=0$。定义 $S_n=\sum_{k=1}^nB(u_k)(t_k-t_{k-1})(u_k\in[t_{k-1},t_k])$,若 $\lim_{n\rightarrow\infty}S_n$ 存在,记为 $Y(t)$,则定义 $\int_0^tB(u)\mathrm du=Y(t)$。
\end{example}

需要注意,定义中出现的极限是均方极限,以下详述。

\begin{definition}\label{def:8.5.1}
    可以证明,对零均值随机变量 $X,Y$,有 $\langle X,Y\rangle=\mathrm E(XY)$ 是内积。记 $||X||=\sqrt{\langle X,X\rangle}$ 为其诱导的范数,则定义 $X$ 和 $Y$ 的\emph{均方距离}为 $||X-Y||$。若一个随机过程 $\{X(t),t\geq0\}$ 满足 $\mathrm E(X^2(t))=||X(t)||^2<+\infty,\forall t\geq0$,则称其为\emph{二阶矩过程}。标准 Brown 运动是二阶矩过程。对二阶矩存在的随机变量列 $\{X_n\}$,若存在二阶矩存在的随机变量 $X$ 满足 $\lim_{n\rightarrow\infty}||X_n-X||=0$,则称 $X$ 为 $\{X_n\}$ 的\emph{均方极限}。
\end{definition}

可以证明,$\{Y(t),t\geq0\}$ 仍为 Gauss 过程,且 $\mathrm E(Y(t))=0,\mathrm{Var}(Y(t))=\mathrm E(Y^2(t))=\mathrm E(\int_0^t\int_0^tB(u)B(s)\mathrm ds\mathrm du)=\frac{t^3}3$,但无独立增量性。

接下来介绍形如 $\int_0^tg(s)\mathrm dB(s)$ 的积分。

\begin{definition}\label{def:8.5.2}
    若均方极限 $\lim_{\lambda_n\rightarrow0}\sum_{k=1}^ng(t_{k-1})(B(t_k)-B(t_{k-1}))$ 存在,则定义 $\int_0^tg(s)\mathrm dB(s)$ 为该极限,称之为 $g(t)$ 关于 $\{B(t),t\geq0\}$ 在 $[0,t]$ 上的 \emph{It\^o 积分}。
\end{definition}

注意,上述定义中 $g(t_{k-1})$ 为每个小区间的左端点处取值,若换成其他点的值,则均方极限不一定存在。特别地,若换成 $\frac12(g(t_k)+g(_{k-1}))$,则均方极限存在,称之为 \emph{Stratonovich 积分}。

\begin{example}\label{Ito}
    考虑 $\int_0^t2B(s)\mathrm dB(s)$。由 $S_n=\sum_{k=1}^n2B(t_{k-1})(B(t_k)-B(t_{k-1}))=\sum_{k=1}^n(B^2(t_k)-B^2(t_{k-1}))-\sum_{k=1}^n(B(t_k)-B(t_{k-1}))^2=B^2(t)-B^2(0)-\sum_{k=1}^n(B(t_k)-B(t_{k-1}))^2$,其中
    \begin{equation*}
        \mathrm E(\sum_{k=1}^n(B(t_k)-B(t_{k-1}))^2)=\sum_{k=1}^n(t_k-t_{k-1})=t,
    \end{equation*}
    而
    \begin{equation*}
        \begin{aligned}
                 & \mathrm{Var}(\sum_{k=1}^n(B(t_k)-B(t_{k-1}))^2)                                                               \\
            =    & \mathrm E((\sum_{k=1}^n(B(t_k)-B(t_{k-1}))^2-t)^2)                                                            \\
            =    & \mathrm E((\sum_{k=1}^n((B(t_k)-B(t_{k-1}))^2-(t_k-t_{k-1})))^2)                                              \\
            =    & \mathrm E(\sum_{k=1}^n((B(t_k)-B(t_{k-1}))^2-(t_k-t_{k-1}))^2)                                                \\
            +    & \sum_{i\neq j}\mathrm E((B(t_i)-B(t_{i-1}))^2-(t_i-t_{i-1}))\mathrm E((B(t_j)-B(t_{j-1}))^2-(t_j-t_{j-1}))    \\
            =    & \sum_{k=1}^n(\mathrm E((B(t_k)-B(t_{k-1}))^4)-2(t_k-t_{k-1})\mathrm E((B(t_k)-B(t_{k-1}))^2)+(t_k-t_{k-1})^2) \\
            =    & \sum_{k=1}^n(3(t_k-t_{k-1})^2-2(t_k-t_{k-1})^2+(t_k-t_{k-1})^2)                                               \\
            =    & 2\sum_{k=1}^n(t_k-t_{k-1})^2                                                                                  \\
            \leq & 2\lambda_nt\rightarrow0(n\rightarrow\infty),
        \end{aligned}
    \end{equation*}
    故 $\lim_{n\rightarrow\infty}\sum_{k=1}^n(B(t_k)-B(t_{k-1}))^2=t$(这称之为 Brown 运动的\emph{二次变差},而可微函数的二次变差总是 $0$),于是 $S_n$ 的均方极限为 $B^2(t)-t$,即 $\int_0^t2B(s)\mathrm dB(s)=B^2(t)-t$。
\end{example}

上例的结果可形式上写为 $2B(t)\mathrm dB(t)=\mathrm dB^2(t)-\mathrm dt$。若在每个小区间选择右端点处的值,即 $\tilde S_n=\sum_{k=1}^n2B(t_k)(B(t_k)-B(t_{k-1}))=S_n+\sum_{k=1}^n2(B(t_k)-B(t_{k-1}))^2$,则 $\tilde S_n$ 的均方极限为 $B^2(t)+t$。Stratonovich 积分的结果为 $\frac12(S_n+\tilde S_n)$ 的均方极限,等于 $B^2(t)$。

可类似地根据定义计算 It\^o 积分 $\int_0^t\mathrm dB(s)=B(t)-B(0)=B(t)$。

一个常用的性质是 $(\mathrm dB(t))^2\sim\mathrm dt$(只差一个 $\mathrm dt$ 的高阶项)。

由例~\ref{Ito} 知 $\mathrm dB^2(t)=2B(t)\mathrm dB(t)+\mathrm dt$,为回答更一般的问题,即 $\mathrm df(t,B(t))$ 的表达式,引入以下定义。

\begin{definition}\label{def:8.5.3}
    若 $\forall t\geq0,\int_0^t\mathrm dX(s)=X(t)-X(0)=\int_0^t\alpha(s)\mathrm ds+\int_0^t\beta(s)\mathrm dB(s)$,其中 $\int_0^t|\alpha(s)|\mathrm ds<+\infty,\mathrm E(\int_0^t\beta^2(s)\mathrm ds)<+\infty$,则称 $\{X(t),t\geq0\}$ 为 \emph{It\^o 过程},记为 $\mathrm dX(t)=\alpha(t)\mathrm dt+\beta(t)\mathrm dB(t)$。
\end{definition}

\textbf{\textcolor{brown}{TODO: $\mathrm dB^2(t)=2B(t)\mathrm dB(t)+\frac1{2!}\cdot2(\mathrm dB(t))^2+\cdots$ 怎么理解?}}

\begin{theorem}\label{thm:8.5.1}
    设 $f(t,x)$ 关于 $t,x$ 分别有一阶、二阶连续偏导,则
    \begin{equation*}
        \mathrm df(t,B(t))=\left(\frac{\partial f}{\partial t}(t,B(t))+\frac12\frac{\partial^2f}{\partial x^2}(t,B(t))\right)\mathrm dt+\frac{\partial f}{\partial x}(t,B(t))\mathrm dB(t),
    \end{equation*}
    上式称之为 \emph{It\^o 微分公式}。
\end{theorem}

\begin{proof}
    仅给出描述性简略证明。由 $\mathrm df=\frac{\partial f}{\partial t}\mathrm dt+\frac{\partial f}{\partial x}\mathrm dx+\frac12\frac{\partial^2f}{\partial x^2}(\mathrm dx)^2$,其中 $\mathrm dx=\mathrm dB(t)$,故 $\mathrm df=\left(\frac{\partial f}{\partial t}(t,B(t))+\frac12\frac{\partial^2f}{\partial x^2}(t,B(t))\right)\mathrm dt+\frac{\partial f}{\partial x}(t,B(t))\mathrm dB(t)$。
\end{proof}

\textbf{\textcolor{brown}{TODO: 上面这个证明这样就可以吗?}}

一般地,若 $\mathrm dX(t)=\alpha(t)\mathrm dt+\beta(t)\mathrm dB(t),Y(t)=f(t,X(t))$,则有
\begin{equation*}
    \mathrm dY(t)=\left(\frac{\partial f}{\partial t}(t,X(t))+\frac{\partial f}{\partial x}(t,X(t))\alpha(t)+\frac12\frac{\partial^2f}{\partial x^2}(t,X(t))\beta^2(t)\right)\mathrm dt+\frac{\partial f}{\partial x}(t,X(t))\beta(t)\mathrm dB(t).
\end{equation*}

\begin{example}
    设 $t$ 时刻的股价为 $S(t)$,其服从几何 Brown 运动 $\mathrm dS(t)=S(t)(u\mathrm dt+\sigma\mathrm dB(t))$。取 $f(t,S(t))=\log S(t)$,则 $\frac{\partial f}{\partial t}=0,\frac{\partial f}{\partial x}=\frac1x,\frac{\partial^2f}{\partial x^2}=-\frac1{x^2}$,故 $\mathrm d\log S(t)=\left(u-\frac12\sigma^2\right)\mathrm dt+\sigma\mathrm dB(t)$,即 $\log S(t)=\log S(0)+\left(u-\frac12\sigma^2\right)t+\sigma B(t)$,故 $S(t)=S(0)e^{\left(u-\frac12\sigma^2\right)t+\sigma B(t)}$。
\end{example}

\begin{example}
    设 $t$ 时刻的股价为 $S(t)$,一种金融产品称之为欧式看涨期权(European Call Option),拥有该期权的人可以在 $T$ 时刻选择是否“行权”,即是否以某预先指定的执行价格 $K$ 买入股票。显然,应当行权当且仅当 $S(T)>K$。期权在 $t$ 时刻的价格定义为 $g(t,S(t))$,则 $g(T,S(T))=\max\{S(T)-K,0\}$。假设股价服从几何 Brown 运动 $\mathrm dS(t)=S(t)(u\mathrm dt+\sigma\mathrm dB(t))$,则由 It\^o 微分方程有 $\mathrm dg(t,S(t))=\left(\frac{\partial g}{\partial t}+\frac{\partial g}{\partial S}S(t)u+\frac12\frac{\partial^2g}{\partial S^2}S^2(t)\sigma^2\right)\mathrm dt+\frac{\partial g}{\partial S}S(t)\sigma\mathrm dB(t)$。称 $P(t)=g(t,S(t))-\delta S(t)$ 为一个投资组合(Portfolio),其中 $-\delta S(t)$ 项表示同时做空该股票,于是 $\mathrm dP(t)=\mathrm dg(t,S(t))-\delta\mathrm dS(t)=\left(\frac{\partial g}{\partial t}+\left(\frac{\partial g}{\partial S}-\delta\right)S(t)u+\frac12\frac{\partial^2g}{\partial S^2}S^2(t)\sigma^2\right)\mathrm dt+\left(\frac{\partial g}{\partial S}-\delta\right)S(t)\sigma\mathrm dB(t)$。为进行风险对冲以消除随机波动的影响,取 $\delta=\frac{\partial g}{\partial S}$,则 $\mathrm dP(t)=\left(\frac{\partial g}{\partial t}+\frac12\frac{\partial^2g}{\partial S^2}S^2(t)\sigma^2\right)\mathrm dt$。假设无套利机会,则 $\mathrm dP(t)=rP(t)\mathrm dt$,其中 $r$ 为无风险利率,故 $\frac{\partial g}{\partial t}+rS\frac{\partial g}{\partial S}+\frac12\frac{\partial^2g}{\partial S^2}S^2\sigma^2=rg$,称之为 \emph{Black-Scholes 方程},其终值条件为 $g(T,S(T))=\max\{S(T)-K,0\}$。\\
    该模型中涉及的主要假设包括:股票服从几何 Brown 运动、无套利机会、标的资产不支付利息、可做空、无交易税费、$\sigma$ 保持不变、$r$ 为已知常数、交易连续等。
\end{example}

\textbf{\textcolor{brown}{TODO: 为什么无套利机会可以推出 $\mathrm dP(t)=rP(t)\mathrm dt$?}}

\textbf{\textcolor{brown}{TODO: 历史注记}}

\end{document}
