\documentclass[../main.tex]{subfiles}
\begin{document}

\begin{definition}\label{def:2.3.1}
称一个随机变量 $X$ 服从 \emph{Bernoulli 分布},若 $\exists p\in(0,1)$,$X$ 的取值集合为 $\{0,1\}$,且 $P(X=1)=p,P(X=0)=1-p$。记作 $X\sim B(p)$。
\end{definition}

$B(p)$ 中的 $p$ 称为该 Bernoulli 分布的\emph{参数}。后续介绍的其他分布同理。

我们常将两种取值分别称为“成功”和“失败”。

计算可得,若 $X\sim B(p)$,则 $\mathrm{E}(X)=p,\mathrm{Var}(X)=p(1-p)$。

\begin{definition}\label{def:2.3.2}
称一个随机变量 $X$ 服从\emph{二项分布},若 $\exists N\in\mathbb{N}^*,\ p\in(0,1)$,$X$ 的取值集合为 $\{0,1,\cdots,N\}$,且 $P(X=k)=\binom{N}{k}p^k(1-p)^{N-k}(k\in\{0,1,\cdots,N\})$。记作 $X\sim B(N,p)$。
\end{definition}

我们常将 $k$ 理解为“$N$ 次独立 Bernoulli 试验中的成功次数”。

计算可得,若 $X\sim B(N,p)$,则 $\mathrm{E}(X)=Np,\mathrm{Var}(X)=Np(1-p)$。

\begin{definition}\label{def:2.3.3}
称一个随机变量 $X$ 服从 \emph{Poisson 分布},若 $\exists\lambda>0$,$X$ 的取值集合为 $\mathbb{N}$,且 $P(X=k)=\frac{\lambda^ke^{-\lambda}}{k!}(k\in\mathbb{N})$。记作 $X\sim P(\lambda)$。
\end{definition}

计算可得,若 $X\sim P(\lambda)$,则 $\mathrm{E}(X)=\lambda,\mathrm{Var}(X)=\lambda$。

对 Poisson 分布的一种常见理解是“一段时间内某个小概率事件发生的次数”所服从的分布。例如,观察时间 $(0,1]$ 内某路口的交通事故数 $X$,将 $(0,1]$ 区间等分成 $n$ 个小区间,即 $l_i=(\frac{i-1}{n},\frac{i}{n}](i=1,2,\cdots,n)$。考虑到 $n$ 很大时,每个区间的长度很小,我们作如下假设:
\begin{enumerate}
    \item 每段区间内,至多发生一次事故
    \item $l_i$ 上发生一次事故的概率与区间长度($1/n$)成正比,为 $p=\lambda/n$
    \item 各区间内是否发生事故彼此独立
\end{enumerate}

则 $P(X=k)=\tbinom{n}{k}p^k(1-p)^{n-k}\rightarrow \frac{\lambda^ke^{-\lambda}}{k!}(n\rightarrow+\infty)$,即 $X\sim P(\lambda)$。

\begin{example}
设某医院平均每天出生婴儿数为 $\lambda$,则接下来 $t$ 天内出生婴儿数服从参数为 $t\lambda$ 的 Poisson 分布。
\end{example}

对于一般的二项分布 $X\sim B(N,p)$,若 $p$ 很小,$N$ 很大,而 $\lambda=Np$ 不太大,则近似有 $X\sim P(\lambda)$,且近似误差不超过 $\min\{p,Np^2\}$。

进一步,若 $N$ 次 Bernoulli 试验并非严格独立,但满足弱相依条件,则 Poisson 分布仍为一种较好的近似。

\begin{example}
(配对问题)\\
$A_i$ 表示第 $i$ 个人拿到自己的帽子,则 $P(A_i)=1/n,P(A_i|A_j)=\frac{1}{n-1}(j\neq i)$,当 $n$ 很大时,$1/n$ 和 $\frac{1}{n-1}$ 很接近,可以认为满足弱相依条件。\\
记 $X$ 为拿到自己帽子的人数,则 $X$ 近似服从参数为 $\lambda=np=n\cdot\frac{1}{n}=1$ 的 Poisson 分布,即 $P(X=k)\approx\frac{e^{-1}}{k!}$。\\
我们用常规做法检查这种近似是否合理。首先考虑指定的某 $k$ 人,记事件 $E$ 表示这 $k$ 人拿到自己的帽子,事件 $F$ 表示其余 $(n-k)$ 人未拿到自己的帽子,则 $P(EF)=P(E)P(F|E)=\frac{(n-k)!}{n!}\cdot P_{n-k}$,其中 $P_{n-k}$ 为 $(n-k)$ 人随机拿帽子时无人拿对的概率。那么我们有 $P(X=k)=\tbinom{n}{k}P(EF)=\frac{1}{k!}P_{n-k}\rightarrow\frac{e^{-1}}{k!}(n\rightarrow+\infty)$。这说明前述的近似是较好的。
\end{example}

\end{document}
