\documentclass[../main.tex]{subfiles}
\begin{document}

为估计参数 $\beta_0,\beta_1$,定义损失函数 $s(\beta_0,\beta_1)=\sum_{i=1}^n(y_i-(\beta_0+\beta_1x_i))^2$,目标是最小化 $s(\beta_0,\beta_1)$。直接求导可得最小值点为 $\hat\beta_1=\frac{\sum_{i=1}^n(x_i-\bar x)(y_i-\bar y)}{\sum_{i=1}^n(x_i-\bar x)^2}=\frac{\sum_{i=1}^n(x_i-\bar x)y_i}{S_{xx}},\hat\beta_0=\bar y-\hat\beta_1\bar x=\sum_{i=1}^n(\frac1n-\frac{(x_i-\bar x)\bar x}{S_{xx}})y_i$,其中 $S_{xx}=\sum_{i=1}^n(x_i-\bar x)^2$,另记 $S_{xy}=\sum_{i=1}^n(x_i-\bar x)(y_i-\bar y)$。记 $\hat y=\hat\beta_0+\hat\beta_1x$。

拟合直线的方程为 $y=\hat y=\hat\beta_0+\hat\beta_1x$,过 $(\bar x,\bar y)$ 点。

\begin{proposition}
    $\hat\beta_1,\hat\beta_0$ 分别是 $\beta_1,\beta_0$ 的无偏估计。
\end{proposition}

\begin{proof}
    计算有 $\mathrm E(\hat\beta_1)=\frac{\sum_{i=1}^n(x_i-\bar x)\mathrm E(y_i)}{S_{xx}}=\frac{\sum_{i=1}^n(x_i-\bar x)(\beta_0+\beta_1x_i)}{S_{xx}}=\frac{\beta_1\sum_{i=1}^n(x_i-\bar x)x_i}{S_{xx}}=\frac{\beta_1\sum_{i=1}^n(x_i-\bar x)^2}{S_{xx}}=\beta_1$,以及 $\mathrm E(\hat\beta_0)=\mathrm E(\bar y-\hat\beta_1\bar x)=\frac1n\sum_{i=1}^n\mathrm E(y_i)-\bar x\mathrm E(\hat\beta_1)=\frac1n\sum_{i=1}^n(\beta_0+\beta_1x_i)-\bar x\beta_1=\beta_0$。
\end{proof}

进一步计算可以得出 $\mathrm{Var}(\hat\beta_1)=\mathrm{Var}\left(\frac{\sum_{i=1}^n(x_i-\bar x)y_i}{S_{xx}}\right)=\sum_{i=1}^n\frac{(x_i-\bar x)^2}{S_{xx}^2}\mathrm{Var}(y_i)=\frac1{S_{xx}^2}\sum_{i=1}^n(x_i-\bar x)^2\sigma^2=\frac{\sigma^2}{S_{xx}}$,以及 $\mathrm{Var}(\hat\beta_0)=\mathrm{Var}(\sum_{i=1}^n(\frac1n-\frac{(x_i-\bar x)\bar x}{S_{xx}})y_i)=\sum_{i=1}^n(\frac1n-\frac{(x_i-\bar x)\bar x}{S_{xx}})^2\sigma^2=(n\cdot\frac1{n^2}-\frac{\sum_{i=1}^n2(x_i-\bar x)\bar x}{nS_{xx}}+\frac{\sum_{i=1}^n(x_i-\bar x)^2\bar x^2}{S_{xx}^2})\sigma^2=(\frac1n+\frac{\bar x^2}{S_{xx}})\sigma^2=\frac{\sigma^2}{S_{xx}}\frac{\sum_{i=1}^nx_i^2}{n}$。

可对 $y_i$ 进行中心化处理:$y_i=\beta_0+\beta_1\bar x+\beta_1(x_i-\bar x)+\epsilon_i=\tilde\beta_0+\beta_1(x_i-\bar x)+\epsilon_i$,则其中的常数项 $\tilde\beta_0=\beta_0+\beta_1\bar x$ 的估计为 $\hat{\tilde\beta}_0=\hat\beta_0+\hat\beta_1\bar x=\bar y$。

当 $x=x_i$ 时,拟合直线上相应点为 $(x_i,\hat\beta_0+\hat\beta_1x_i)$。记 $\hat y_i=\hat\beta_0+\hat\beta_1x_i$ 为在 $x=x_i$ 处 $y$ 的拟合值,$y_i-\hat y_i$ 称之为\emph{残差}(Residual)或\emph{误差}。误差平方和(Sum of Squared Errors, SSE)或残差平方和(Residual Sum of Squares, RSS)为 $SSE=\sum_{i=1}^n(y_i-\hat y_i)^2=\sum_{i=1}^n(y_i-(\hat\beta_0+\hat\beta_1x_i))^2$。

\begin{proposition}
    $\hat\sigma^2=\frac{SSE}{n-2}$ 是 $\sigma^2$ 的无偏估计。
\end{proposition}

还可以据此给出 $\hat\beta_1,\hat\beta_0$ 的标准差的估计 $\widehat{\mathrm{Se}}(\hat\beta_1)=\frac{\hat\sigma}{\sqrt{S_{xx}}},\widehat{\mathrm{Se}}(\hat\beta_0)=\hat\sigma\sqrt{\frac1n+\frac{\bar x^2}{S_{xx}}}$。

\end{document}
