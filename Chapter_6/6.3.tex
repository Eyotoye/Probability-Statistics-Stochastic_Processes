\documentclass[../main.tex]{subfiles}
\begin{document}

无论是矩估计还是极大似然估计,都是用样本的函数来估计总体的参数,对每个参数给出一个估计值,这样的估计称为\emph{点估计}。

用于估计参数的函数 $\hat\theta=\hat\theta(X_1,\cdots,X_n)$ 称为\emph{估计量},其分布(依赖于 $\theta$)称为\emph{抽样分布},其标准差 $\sqrt{\mathrm{Var}(\hat\theta)}$ 称为\emph{标准误(差)}(Standard error),记为 $\mathrm{Se}=\mathrm{Se}(\hat\theta)$。

在选择估计量时,有若干准则。首先介绍所谓\emph{无偏性}。

我们称 $\mathrm E(\hat\theta-\theta)=\mathrm E(\hat\theta)-\theta$ 为 $\hat\theta$ 的\emph{偏差}(bias)。

\begin{definition}\label{def:6.3.1}
    设 $\hat\theta$ 是 $\theta$ 的估计量,若 $\forall\theta,\mathrm E(\hat\theta-\theta)=0$,则称 $\hat\theta$ 为 $\theta$ 的一个\emph{无偏估计(量)}。
\end{definition}

由上述定义可知,无偏性指的是无系统偏差。

一般地,若 $\hat g(X_1,\cdots,X_n)$ 是对 $\theta$ 的函数 $g(\theta)$ 的估计,且满足 $\forall\theta,\mathrm E(\hat g(X_1,\cdots,X_n))=g(\theta)$,则称 $\hat g(X_1,\cdots,X_n)$ 是 $g(\theta)$ 的一个无偏估计。

对于无偏估计 $\hat g(X_1,\cdots,X_n)$,若进行 $N$ 组抽样,第 $m$ 组样本记作 $X_1^{(m)},\cdots,X_n^{(m)}$,则由大数定律,$\frac1N\sum_{m=1}^N\hat g(X_1^{(m)},\cdots,X_n^{(m)})$ 会收敛到 $\mathrm E(\hat g(\theta))=g(\theta)$。

在实际应用中,无偏的重要性视情况而定。

\begin{example}
    若随机变量 $X$ 的均值 $\mu$ 和方差 $\sigma^2$ 均未知,则由 $\mathrm E(\bar X)=\mu$ 知 $\bar X$ 是 $\mu$ 的无偏估计。\\
    而二阶矩 $m_2= \frac1n\sum_{i=1}^n(X_i-\bar X)^2=\frac1n\sum_{i=1}^n(X_i-\mu)^2-(\bar X-\mu)^2$,有 $\mathrm E(m_2)=\frac{n-1}n\sigma^2\neq\sigma^2$,故 $m_2$ 不是 $\sigma^2$ 的无偏估计(系统偏小)。\\
    样本方差 $S^2=\frac1{n-1}\sum_{i=1}^n(X_i-\bar X)^2$ 中的 $(n-1)$ 是所谓的无偏修正,满足 $\mathrm E(S^2)=\sigma^2$,故 $S^2$ 是 $\sigma^2$ 的无偏估计。
\end{example}

\begin{example}
    若随机变量 $X\sim U(0,\theta)$,则矩估计 $\hat\theta=2\bar X$ 为 $\theta$ 的无偏估计,而 MLE $\theta^*=\max\{X_1,\cdots,X_n\}$,有 $\mathrm E(\theta^*)=\frac n{n+1}\theta$,故 $\theta^*$ 不是 $\theta$ 的无偏估计。
\end{example}

这个例子说明,MLE 不一定是无偏的。

下面介绍\emph{均方误差}准则。

我们定义均方误差(MSE)为 $\mathrm E((\hat\theta-\theta)^2)=\mathrm{Var}(\hat\theta)+\mathrm E^2(\hat\theta-\theta)$,其中等号右边的两项分别反映了\emph{精确度}(precision)和\emph{准确度}(accuracy)。

\begin{definition}\label{def:6.3.2}
    若 $\hat\theta_1,\hat\theta_2$ 为 $\theta$ 的无偏估计,且 $\forall\theta,\mathrm{Var}(\hat\theta_1)\leq\mathrm{Var}(\hat\theta_2)$,且存在一个 $\theta$ 的值使得不等号严格成立,则称 $\hat\theta_1$ 在均方误差意义下优于 $\hat\theta_2$。
\end{definition}

\begin{example}
    若随机变量 $X$ 的均值 $\mu$ 未知,方差为 $\sigma^2$,则 $\bar X,\frac12(X_1+X_2),X_1$ 都是 $\mu$ 的无偏估计,它们各自的方差为 $\frac{\sigma^2}n,\frac{\sigma^2}2,\sigma^2$,故若 $n>2$,则 $\bar X$ 在均方误差意义下优于 $\frac12(X_1+X_2)$,而 $\frac12(X_1+X_2)$ 在均方误差意义下优于 $X_1$。
\end{example}

\begin{definition}\label{def:6.3.3}
    若 $\hat\theta_0$ 是 $\theta$ 的无偏估计,且 $\forall\hat\theta$ 为 $\theta$ 的无偏估计,都有 $\forall\theta,\mathrm{Var}(\hat\theta_0)\leq\mathrm{Var}(\hat\theta)$,则称 $\hat\theta_0$ 是 $\theta$ 的\emph{最小方差无偏估计}(MVUE)。
\end{definition}

\begin{example}
    若 $X\sim N(\mu,\sigma^2)$,则 $\mathrm E(m_2)=\frac{n-1}n\sigma^2,\mathrm E(S^2)=\sigma^2$,但 $\mathrm E((m_2-\sigma^2)^2)<\mathrm E((S^2-\sigma^2)^2)$,故 $m_2$ 在均方误差意义下优于 $S^2$。尽管 $m_2$ 是有偏的,但它有更小的方差,总的来说其 MSE 更小。
\end{example}

接下来介绍一些大样本性质。所谓大样本性质,是指样本容量 $n$ 趋于无穷时 $\hat\theta$ 的性质。

首先是\emph{渐进无偏性}。若 $\lim_{n\rightarrow+\infty}\mathrm E(\hat\theta-\theta)=0$,则称 $\hat\theta$ 具有渐进无偏性。

然后是\emph{相合性}。若 $\forall\epsilon>0,\lim_{n\rightarrow+\infty}P(|\hat\theta-\theta|\geq\epsilon)=0$,则称 $\hat\theta$ 是 $\theta$ 的\emph{相合估计}。

$\hat\theta$ 是 $\theta$ 的相合估计,当且仅当 $\hat\theta\overset{P}\rightarrow\theta$。例如,根据弱大数定律,$\bar X$ 是 $\mu$ 的相合估计。

相合性是良好点估计的自然要求。

\begin{example}
    若随机变量 $X$ 的均值为 $\mu$,方差为 $\sigma^2$,考虑 $m_2=\frac1n\sum_{i=1}^n(X_i-\bar X)^2=\frac1n\sum_{i=1}^n(X_i-\mu)^2-(\bar X-\mu)^2$,由大数定律,$\frac1n\sum_{i=1}^n(X_i-\mu)^2\overset{P}\rightarrow\mathrm E((X_i-\mu)^2)=\sigma^2$,而 $(\bar X-\mu)^2\overset{P}\rightarrow0$,故 $m_2\overset{P}\rightarrow\sigma^2$,即 $m_2$ 是 $\sigma^2$ 的相合估计。同时,$S^2=\frac n{n-1}m_2\overset{P}\rightarrow\sigma^2$,故 $S^2$ 也是 $\sigma^2$ 的相合估计。
\end{example}

最后是\emph{渐进正态性}。若 $\frac{\hat\theta-\theta}{\mathrm{Se}(\hat\theta)}\overset{d}\rightarrow Z\sim N(0,1)$,则称 $\hat\theta$ 是 $\theta$ 的\emph{渐进正态估计}。

例如,根据 CLT,$\bar X$ 是 $\mu$ 的渐进正态估计,且 $\mathrm{Se}(\bar X)=\frac{\sigma}{\sqrt n}$。

若 $\hat\theta$ 是 $\theta$ 的渐进正态估计,则当 $n$ 充分大时,近似有 $\hat\theta\sim N(\theta,\mathrm{Se}^2(\hat\theta))$。

\end{document}
