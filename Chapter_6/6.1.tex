\documentclass[../main.tex]{subfiles}
\begin{document}

设 $X_1,\cdots,X_n$ 为独立同分布的样本,我们定义\emph{样本矩}如下:
\begin{enumerate}
    \item $k$ 阶原点矩 $a_k=\frac1n\sum_{i=1}^nX_i^k$
    \item $k$ 阶中心矩 $m_k=\frac1n\sum_{i=1}^n(X_i-\overline X)^k$
\end{enumerate}

根据大数定律,$a_k\rightarrow\mathrm E(X^k)$,$m_k\rightarrow\mathrm E(X-\mathrm E(X))^k$。

\begin{example}
    设 $X_1,\cdots,X_n$ 独立同分布,$X_i\sim N(\mu,\sigma^2)$,则 $\mu=\mathrm E(X)\approx a_1=\overline X$,$\sigma^2=\mathrm{Var}(X)\approx m_2=\frac1n\sum_{i=1}^n(X_i-\overline X)^2$。
\end{example}

\begin{example}
    设 $X_1,\cdots,X_n$ 独立同分布,$X_i\sim Exp(\lambda)$,则 $\lambda=\mathrm E(X)^{-1}\approx a_1^{-1}=\frac1{\overline X}$,或 $\lambda=\mathrm{Var}(X)^{-1/2}=m_2^{-1/2}$。
\end{example}

我们发现上例中 $\lambda$ 可以有两种不同的矩估计,通常尽量用低阶矩。

\begin{example}
    设 $X_1,\cdots,X_n$ 独立同分布,定义\emph{经验分布函数}为 $F_n(x)=\frac1n\sum_{i=1}^nI_{\{X_i\leq x\}}$,则 $I_{\{X_i\leq x\}}\sim B(p)$,其中 $p=F(x)$,从而 $F_n(x)\overset{\mathrm{a.s.}}{\rightarrow}F(x)$。
\end{example}

注意到,$F_n(x)$ 的矩就是 $(X_1,\cdots,X_n)$ 的样本矩。

\end{document}
