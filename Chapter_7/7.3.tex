\documentclass[../main.tex]{subfiles}
\begin{document}

\begin{example}
    设总体分布为 $X\sim N(\mu,\sigma^2)$,其中 $\sigma^2$ 已知。给定 $\alpha\in(0,1)$,有 $(1-\alpha)$-置信的(双侧)置信区间 $\mu\in(\bar X-z_{\frac\alpha 2}\frac\sigma{\sqrt n},\bar X+z_{\frac\alpha 2}\frac\sigma{\sqrt n})$。而在进行(双侧)假设检验时,$\forall\mu_0$,若 $H_0:\mu=\mu_0,H_1:\mu\neq\mu_0$,检验准则为若 $|\bar X-\mu_0|\geq z_{\frac\alpha 2}\frac\sigma{\sqrt n}$ 则拒绝 $H_0$,即所谓的 \emph{接受域} $R^c=\{(X_1,\cdots,X_n)\mid|\bar X-\mu_0|<z_{\frac\alpha 2}\frac\sigma{\sqrt n}\}$,或接受条件为 $\bar X-z_{\frac\alpha 2}\frac\sigma{\sqrt n}<\mu_0<\bar X+z_{\frac\alpha 2}\frac\sigma{\sqrt n}$。由此可见,$\mu_0$ 落在置信区间 $(\bar X-z_{\frac\alpha 2}\frac\sigma{\sqrt n},\bar X+z_{\frac\alpha 2}\frac\sigma{\sqrt n})$ 等价于用 $\bar X$ 作为检验统计量检验上述 $H_0,H_1$ 时的结果为不拒绝 $H_0$,这体现出区间估计与假设检验存在\emph{对偶关系}。
\end{example}

\end{document}
